\documentclass[titlepage,12pt,a4paper,ngerman]{report}
\usepackage[utf8]{inputenc}
\usepackage[T1]{fontenc}
\usepackage[german]{babel}
\usepackage{graphicx}
\usepackage{wrapfig}
\usepackage{amsmath}
\usepackage{cleveref}
\usepackage{amsfonts}
\usepackage{amssymb}
\usepackage{tikz}
\usetikzlibrary{decorations.pathmorphing,patterns}
\usetikzlibrary{arrows}
\usepackage{nicefrac}
\usepackage{mathtools}
\usepackage[margin=1in]{geometry}
\usepackage{tcolorbox}
\setlength{\parskip}{1em}
\usepackage{enumerate}

\newcommand{\verteq}{\rotatebox{90}{$\,=$}}
\newcommand{\equalto}[2]{\underset{\scriptstyle\overset{\mkern4mu\verteq}{#2}}{#1}}
\newcommand{\equaltoup}[2]{\overset{\scriptstyle\underset{\mkern4mu\verteq}{#2}}{#1}}
\newcommand{\tx}[1]{\textrm{#1}}
\newcommand{\ov}[1]{\overline{#1}}
\newcommand{\ub}[1]{\underbrace{#1}}
\newcommand{\ob}[1]{\overbrace{#1}}
\newcommand{\im}{\tx{im}}
\newcommand{\spa}{\tx{span}}
\newcommand{\adj}{\tx{adj}}
\newcommand{\grad}{\tx{grad}}
\newcommand{\lag}{\mathcal{L}}
\newcommand{\ham}{\mathcal{H}}
\newcommand{\prt}[2]{\frac{\partial #1}{\partial #2}}
\newcommand{\casess}[4]{\left\{ \begin{array}{ll} {#1} & {#2} \\ {#3} & {#4} \end{array} \right.}

\newcommand{\const}{\tx{const.}}
\newcommand{\summ}[2]{\sum_{#1}^{#2}}
\newcommand{\intt}[2]{\int_{#1}^{#2}}


\tcbuselibrary{theorems}

\newtcbox{\fribox}[1]{nobeforeafter,colback=white,colframe=red!75!black,fonttitle=\bfseries,title=#1,sharp corners,tcbox raise base} % mahlt eine box nur um den text mit tittel

\newcommand{\frbox}[2]{\begin{tcolorbox}[colback=white,colframe=red!75!black,fonttitle=\bfseries,title=#1]#2\end{tcolorbox}} % mahlt eine große box um alles mit tittel

\newtcbox{\ribox}{nobeforeafter,colback=white,colframe=red!75!black,sharp corners,tcbox raise base} % mahlt eine box nur um den text

\newcommand{\rbox}[1]{\begin{tcolorbox}[colback=white,colframe=red!75!black]#1\end{tcolorbox}} % mahlt eine große box um alles was drinnen ist

\newcommand{\rmbox}[1]{\tcboxmath[colback=white,colframe=red!75!black]{#1}} % mahlt eine box um mathe innerhalb mathmode

\renewcommand{\boxed}{\rmbox}

\hbadness=99999

\begin{document}

\title{
\Huge Theoretische Physik I \\[1em]
\Large Vorlesung von Prof.Dr. Gerhard Stock im Sommersemester 2018}
\author{Markus Österle \\ Andréz Gockel}
\date{17.04.2018}
\maketitle
\tableofcontents

\chapter*{Einleitung - Theoretische Physik}
\addcontentsline{toc}{chapter}{Introduction}  

\section*{Einführung}
\addcontentsline{toc}{section}{Einführung} 
\begin{itemize}
\item geht von \underline{grundlegenden Naturgesetzen} aus, die als Postulate (=Axiome)
\item benutzt mathematische Methoden um daraus physikalische Aussagen herzuleiten (z.B. $E_{kin}\sim v^2$) 
\item Eine \underline{Theorie} basiert auf definierten (Def.) Annahmen
\begin{itemize}
	\item[$\rightarrow$] gilt innerhalb eines Anwendungsbereiches und muss hier experimentell (Exp.) \underline{verifizierbare} Ergebnisse liefern\\
	z.B. klassische Mechanik funktioniert für
	\begin{itemize}
		\item[-] $v\ll c$
		\item[-] $\int(p)dx \gg \hbar$
	\end{itemize}
\end{itemize}
\item Ein \underline{theoretisches Modell} macht oft \underline{idealisierende Annahmen} um \underline{explizite} Lösungen zu erlauben z.B. harmonischer Oszillator
\item Computational Physics\\
löst theoretische Ansätze \underline{numerisch}
\end{itemize}

\section*{Bedeutung der klassischen Mechanik}
\addcontentsline{toc}{section}{Bedeutung der Mechanik} 
\begin{itemize}
\item zentrale Rolle, da anschauliche Theorie
\item Einführung:
\begin{itemize}
	\item zentrale Größen (z.B. Energie, Drehimpuls, Wirkung)
	\item Methoden (z.B. Variationsrechnung, Störungstheorie)
	\item Modelle (z.B. harmonischer Oszillator, wichtig in Quantenmechanik (QM), Feldtheorie, ...)
\end{itemize}
\item praktische Bedeutung:
\begin{itemize}
	\item Himmelsmechanik
	\item Statik
	\item Molekül-
	\item Chemie- und Biophysik
\end{itemize}
\item nicht lineare Dynamik (z.B. Chaostheorie, Strukturbildung) sind Beispiele aktueller Forschung
\end{itemize}

\chapter{Newtonsche Mechanik}
\section{allgemeine Begriffe}
\begin{itemize}
\item Statik (ruhende Körper)
\item Kinematik (Bewegung, ohne Beschreibung der Wechselwirkungen (WW))
\item Dynamik (Bewegung \underline{mit} Beschreibung der WW)
\end{itemize}

\subsection{Bezugsystem}
Ursprung $O$\\
Basisvektoren $\vec{e}_{i}= \left\{\begin{array}{ll} x,y,z \\1,2,3\end{array}\right. $ \\	
\underline{Ort:} $\vec{r}$ eines Teilchens (Massenpunkte) (Bahnkurve, Trajektorie)\\
$\vec{r}(t) = \sum_{i} \vec{x}_{i}(t)\vec{e}_{i}$\\
\underline{Geschwindigkeit:} $\vec{v}(t) = \frac{d}{dt} \vec{r}(t) = \dot{\vec{r}}(t)$\\
\underline{Impuls:} $ \vec{p}(t) = m\cdot \vec{v}(t)$\\
\underline{Beschleunigung:} $\vec{a}(t) = \frac{d^{2}\vec{r}(t)}{dt^{2}} = \ddot{\vec{r}} (t)$\\
z.B. geradlinige - gleichmäßige Bewegung\\
$\vec{r}(t) = \vec{r}_{0} + \vec{v}_{0} t$\\
$\vec{v}(t) = \dot{\vec{r}}  = \vec{v}_{0}$\\
$\vec{a}(t) = 0$

\section{Die Newtonsche Gesetze (1687)} % gleichheitszeichen allignment
{\large\underline{NG1: Trägheitsgesetz}}\\
Kräftefreie Bewegung ist gleichförmig, d.h. $v = \const$\\
{\large\underline{NG2: Grundgesetz der Mechanik}}\\
$ F = \frac{dp}{dt} = \frac{d}{dt}(mv) = ma$ $m = \const$\\
def. Kraft und Masse, Bewegungsgleichung\\
{\large\underline{NG3: Action = Reactio}}\\
$F_{12} = -F_{21}$\\
Kraft von 1 auf 2 = Kraft von 2 auf 1
{\large \underline{Voraussetzung (Annahme):}}
\begin{itemize}
\item ,,absoluter`` Raum
\item ,,absolute`` Zeit
\item ,,absolute`` Masse
\end{itemize}
$\rightarrow$ nur gültig für $\frac{v}{c} \ll 1$\\
($c = $Lichtgeschwindigkeit)\\
{\large\underline{Diskussion
	{NG1:}} Macht nur Sinn bei Angabe von Bezugsystem z.B. Vergleich rotierendes vs. ruhendes Bezugssystem\\
$\rightarrow$ Ein Bezugssystem, in dem das NG1 gilt heißt ,,Inertialsystem`` (IS)\\
\underline{Bsp:} Hörsaal, relativ zum Fixsternhimmel
Näherung, z.B. wegen Erdrotation $\rightarrow$ Foucaultsches Pendel\\
$\rightarrow$ physikalische Gesetze nehmen in IS eine besonders einfache Form an.\par
\underline{NG1:} In einem IS ist die kräftefreie Bewegung durch $\vec{r}(t) = const.$ beschrieben
\underline{Relativitätsprinzip} (Galilei)\\
Geg. sei IS S mit Bahnkurve $\vec{r}(t)$\\
und IS $\bar{S}$ mit Bahnkurve $\vec{r}(t)$\\
worin $\bar{S}$ um $\vec{r}_{0}$ zu S verschoben sei und sich mit einer Geschwindigkeit $\vec{v}_{0} = const.$ dazu bewege:\\
Dann gilt die \underline{Galilei Transformation} (Trafo)\\
$ r \rightarrow \bar{r}$ mit
\begin{equation*}
\bar{r} (t) = \vec{r}(t) + \vec{r}_{0} + \vec{v}_{0} t \tag{1}
\end{equation*}
$\rightarrow$ \underline{Relativitätsprinzip}\par
Alle IS sind gleichwertig.\par
{\large\underline{NG2:}} $F = \dot{p} = m \cdot \vec{a}$ % tabelle
\begin{itemize}
	\item setzt ebenfalls ein IS voraus
	\item beschreibt Bewegung mittels \underline{Wirkung} $m\cdot a$ und \underline{Ursache} Kraft $F$
	\item Definition der Kraft und der (trägen) Masse $m$
	\item \underline{grundlegendes Postulat} der klassischen Mechanik:\\
	sind alle Kräfte $F_{1}$ bekannt, so beschreibt $$m\cdot a = \sum_{i} F_{i}$$ \underline{vollständig} die Bewegung.
\end{itemize}
%vorlesung 2
Für gegebene Kraft $\vec{F}$ ergibt sich \underline{vollständig} die Bewegung.\par
\underline{NG3:} $$\vec{F}_{12}=-\vec{F}_{21}$$\\
d.h die Kraft ergibt sich als WW zwischen Körpern.\\
$$\rightarrow \frac{d}{dt} p_{1}=-\frac{d}{dt}p_{2} \hspace{1cm} \rightarrow \hspace{1cm} \frac{d}{dt}(p_{1}+p_{2}) =0$$
$\rightarrow$ Impulserhaltung\par
eigentlich auf Grund der ,,Homogenität des Raumes``.\par
\underline{Zusatz:} Kräfte addieren sich wie Vektoren\\
$$ \vec{F}_{tot} = \sum_{i} \vec{F}_{i}$$
\subsection{Bem:} 
\begin{itemize}
	\item vorher Aristotelesche Mechanik: unterschied
	\begin{itemize}
		\item Bewegung auf der Erde
		\item Bewegung der Gestirne
	\end{itemize}
	\item Newton vereinheitlichte beide Bereiche: Theorie gilt sowohl für Bewegung der Planeten als auch für fallenden Apfel
\end{itemize}
$\rightarrow$ immense Abstraktionsleistung!\\
allg. Ansatz: Vereinheitlichung von z.B. elektrischer und magnetischer WW Maxwell
\begin{itemize}
	\item Wesentliches Axiom ist das 2.NG. (das 1.NG definiert IS, 3.NG entspricht Impulserhaltung)
\end{itemize}
\underline{2.NG:} 
$$m\ddot{\vec{r}}(t) = F(\vec{r},\dot{\vec{r}}, t)$$
DGL 2.Ord, Lösung ergibt Bahnkurve $\vec{r}(t)$ ,,Bewegungsgleichung``\\
Integrationskonstanten gegeben durch Anfangsbedingungen: z.B. $\vec{r}(0) , \dot{\vec{r}}(0)$\par
\subsection{Beispiele:}
\begin{itemize}
	\item[1.)] \underline{Konstante Kraft:} $\vec{F} =\vec{F}_{0}$\\
	z.B. $\vec{F}=m\vec{g}$\\
	$\rightarrow \vec{r}(t) = \frac{1}{2m} \vec{F}_{0} \cdot t^2 + \vec{v}_{0} \cdot t+\vec{r}_{0}$
	\item[2.)] \underline{Lineare Kraft:} $\vec{F} \sim \vec{r}$\\
	z.B. bei Federpendel mit $r= \vline \vec{r}\vline $\\
	$m\ddot{r} = - k r$\\
	mit $r(t) = x_{0} \cos{\omega t} + \frac{v_{0}}{\omega} sin{\omega t}$, $\omega^2 = \frac{k}{m}$\\
	\item[3.)] \underline{Zentrale Kraft:} $\vec{F} = \vec{F}(\vec{r}=\vline \vec{r}_{2}-\vec{r}_{1}\vline )$\\
	z.B. Gravitation $\vec{F} = \gamma \frac{m_1 m_2 }{r^2} \frac{\vec{r}}{ r}$
	\item[4.)] \underline{Lorenzkraft:} (geschwindigkeitsabhängig)\\
	$\vec{F}={\vec{r}}(t) = F(\vec{r},\dot{\vec{r}}, t) = e[\hat{E} (\vec{r}, t) + \dot{\vec{r}} \cdot \vec{B} (\vec{r},t)$
	Ladung e im elektrischen Feld $\vec{E}$ und magnetischen Feld $\vec{B}$
	\item[5.)] \underline{Reibungskräfte:}
	\begin{itemize}
		\item Stokesche Reibung $\vec{F} = -\gamma \vec{v}$
		\item Luftreibung: $\vec{F} \sim \vec{v}^{\,2} \frac{\vec{v}}{v}$
	\end{itemize}
\end{itemize}

\section{Erhaltungssätze}
\begin{itemize}
	\item spielen Zentrale Rolle in der Physik
	\item sind allg. gültig, z.B. auch in der QM
	\item reflektieren \underline{Symmetrie} des Systems
\end{itemize}
allg Form eines erhaltungssatzes der Größe $A(\vec{r},\dot{\vec{r}},t)$\\
$\frac{d}{dt} A = 0 \leftrightarrow $ A ist erhalten\par

\subsection{Impulserhaltung}
$\vec{F} = 0 \rightarrow \frac{d\vec{p}}{dt} =0 \rightarrow \vec{p} = \const$

\subsection{Drehimpulserhaltung}
Vektorielle Multiplikation von NG2 mit $\vec{r} (t)$ ergibt:
$$m\vec{r}(t) \times \ddot{\vec{r}}(t) = \vec{r}(t) \times \vec{F}$$
Mit
Drehimpuls:
$$\vec{T} = \vec{r}(t) \times \vec{p}(t) = \vec{r} \times (m\dot{\vec{r}})$$ 
Drehmoment:
$$ \vec{M} = \vec{r} \times \vec{F}$$
ist
$$\frac{d}{dt}\vec{l} = \underbrace{\dot{\vec{r}} \times (m\dot{\vec{r}})}_{ =0} + \vec{r} \times m \ddot{\vec{r}} = \vec{M}$$
$$\frac{d\vec{l}}{dt} = \vec{M}$$
$$\vec{M} = 0 \quad \Leftrightarrow \quad \frac{d \vec{l}}{dt} = 0 , \quad \vec{l} \textrm{ erhalten}$$

\subsubsection{Bsp: Zentralkraft}
$\vec{F} \parallel  \vec{r} \rightarrow \vec{M} = 0 \rightarrow \vec{l}  = const.$
legt man oEdA $\vec{l}$ in z - Richtung \\
$$\vec{l} = l \vec{e}_z = m \vec{r}\cdot \dot{\vec{r}}$$
liegen $\vec{r}$ und $ \dot{\vec{r}}$ in x - y - Ebene.\par

\subsection{Energieerhaltung}
Ein Teilchen, das sich unter $\vec{F}$ von $\vec{r}$ nach $\vec{r} + d\vec{r}$ bewegt, verrichtet die Arbeit:
$$ dW = \vec{F} \cdot d \vec{r}$$
Längs eines eges C von $\vec{r}_{1}$ nach $\vec{r}_{2}$ ist die geleistete Arbeit
$$W = \int_C dW = \int_C d\vec{r}\cdot \vec{F}$$
die von $\vec{r}_1, \vec{r}_2$ und i.h. auch von der Wegführung abhängt-\par
Die pro Zeit verrichtete Arbeit heißt \underline{Leistung} 
$$ P = \frac{dW}{dt} = \frac{\vec{F}\cdot d \vec{r}}{dt} = \vec{F}\cdot \dot{\vec{r}}$$
$$[\oint_C = \textrm{ geschlossenes Wegintegral }]$$
Multiplikation von NG2 mit $\dot{\vec{r}}$ gibt:
$$ m\ddot{\vec{r}} \cdot \dot{\vec{r}} =\vec{F} \cdot \dot{\vec{r}} \quad ( \widehat{=} \tx{ Leistung P })$$
$$\frac{d}{dt} \ub{\frac{m(\dot{\vec{r}})^2}{2}}_{=T} = \frac{dT}{dt} = P = (\vec{F}_{\tx{kons}} + \vec{F}_{\tx{diss}}) \cdot \dot{\vec{r}} $$
Kinetische Energie T\\
konservative Kräfte $\vec{F}_{\tx{kons}}$ und dissipative Kräfte $\vec{F}_{\tx{diss}}$, wobei $\vec{F}_{\tx{kons}}$ alle Anteile mit:
$$\vec{F}_{\tx{kons}} \cdot \dot{\vec{r}} = - \frac{d}{dt} U(\vec{r}) = - \vec{\nabla} U \cdot \dot{\vec{r}}$$
erhält, wobei k das Potential oder die potentielle Energie ist.
Minuszeichen ist Konvention.\par
Zusammen:
\begin{equation*}
\boxed{\frac{d}{dt} \bigg( \frac{m \dot{\vec{r}}^{\,2}}{2} + U(\vec{r})\bigg)= \vec{F}_{\tx{diss}}
	\cdot \dot{\vec{r}}} \tag{2}
\end{equation*}
Konservative Kraft $\Leftrightarrow E = T+U=\tx{const.}$\\
Mit 
$$ \vec{F}_{\tx{kons}} \cdot \dot{\vec{r}} = - \frac{dU(\vec{r}}{dt} = -\frac{\partial U}{\partial x}\frac{\partial x}{\partial t} - \frac{\partial U}{\partial y}\frac{\partial y }{\partial t} - \frac{\partial U}{\partial z} \frac{\partial z }{\partial t}$$
$$ = - \grad(U(\vec{r})) \cdot  \dot{\vec{r}} = - \vec{\nabla} U \cdot \dot{\vec{r}}$$
%vektorschreibweise
$$ \vec{\nabla} = ( \frac{\partial }{\partial x}, \frac{\partial }{\partial y}, \frac{\partial }{\partial z}) ^\top$$
$$ \grad(U(\vec{r})) = ( \frac{\partial U}{\partial x}, \frac{\partial U}{\partial y}, \frac{\partial U}{\partial z}) ^\top$$
folgt :
\begin{equation*}
\boxed{\vec{F}_{\tx{kons}} = - \grad(U(\vec{r}))} \tag{3}
\end{equation*}

\subsubsection{Bsp: Der gedämpfter harmonischer Oszillator}
$$ F = -kx - \gamma \dot{x} = F_{\tx{kons}} + F_{\tx{diss}}$$
mit
$$F_{\tx{kons}} = -\frac{dk}{dx} \rightarrow k(x) = \frac{k}{2} x^2$$
Da $F_{\tx{diss}} \cdot \dot{x}$ quadratisch in $\dot{x} , \frac{dk(x)}{st}$ aber linear in $\dot{x}$ kann $F_{\tx{diss}} \dot{x}$ nicht in der Form $\frac{dk}{dt}$ geschrieben werden.\\\\
\underline{Bedingung} für konservative Kraft ist
$$ \tx{rot}(\vec{F}(\vec{r})) = \vec{\nabla} \times \vec{F}(\vec{r}) = 0 \quad \leftrightarrow \quad \vec{F}(\vec{r}) = -\vec{\nabla} k(\vec{r})$$
Dann ist das Wegintegral:
$$ W = \int_C d\vec{r}\  \vec{F} (\vec{r})$$
\underline{Wegunabhängig}, verschwindet also für jeden geschlossenen Pfad :
$$ \vec{F}_{\tx{kons}} = -\vec{\nabla} U(\vec{r}) \quad \leftrightarrow  \quad \vec{\nabla} \times \vec{F} = 0 \quad \leftrightarrow \quad  \oint_C d\vec{r} \cdot \vec{F}= 0$$
\underline{allg.:} 
$$ \vec{F}_{\tx{kons}} = - \vec{\nabla} U(\vec{r}) + \ub{ \dot{\vec{r}} \times \vec{B} (\vec{r} , t) } _{\tx{z.B. Lorenz-Kraft}}$$
da:
$$P_L = \dot{\vec{r}} \cdot (\dot{\vec{r}} \times \vec{R} ) = 0$$
\underline{Anwendung:} Vereinfachung von Bewegungsgleichungen
$$\vec{F} = m \ddot{\vec{r}} = - \vec{\nabla} U (x)$$
DGL 2. Ordnung\\
\underline{Bsp:} 1 Dimensionales System mit Energieerhaltung
$$ \frac{1}{2} m \dot{x}^2 + U(x) = E$$
$$\rightarrow \dot{x} = \pm \sqrt{\frac{2}{m}(E -U(x))} \qquad \textrm{DGL 1. Ordnung}$$
,,erstes Integral`` ( da eine Integration bereits vollzogen)
$$ dt = \frac{dx}{\sqrt{\frac{2}{m} (E-U(x))}}$$
$$ t-t_0 = \int_{x_0}^x \frac{dx'}{\sqrt{\frac{2}{m} (E-U(x'))}}$$
Qualitative Diskussion der Bewegung:\\
$$E=T+U\ge U$$
%grafik
\underline{Numerische Integration der Bewegungsgleichung:}
\underline{Bsp:} $$f(t) = m \ddot{x}(t) = m \dot{v}(t)$$
\underline{Idee:} Taylor Entwicklung von x zur Zeit 
\begin{align*}
x(t+\Delta t) &= x(t) + \frac{dx}{\underbrace{dt}_{v}} \underbrace{((t+\Delta t) -t)}_{\Delta t} + \frac{1}{2} \frac{d^2x}{\underbrace{dt^2}_{F/m = a}} \Delta t^2 + \frac{1}{3!} \frac{d^3x}{dt^3} \Delta t^3 + \mathcal{O}(t^4)\\
&= x(t) + v(t) \Delta t + \frac{f(t)}{2m} \Delta t^2 + \frac{\Delta t^3}{3!} + \mathcal{O}(t^4)
\end{align*}
z.B. Lösung durch Abbruch in 2.Ordnung (Euler-Algorithmus)\\
\underline{Besser:}
$$x(t-\Delta t) = x(t) -v(t) \Delta t + \frac{f(t)}{2m} \Delta t^2 - \frac{\Delta t^3}{3!} \dddot{x} + \mathcal{O}(\Delta t^4)$$
$$ x(t+\Delta t) + x(t-\Delta t) = 2x(t) + \frac{f(t)}{m} \Delta t^2 + \mathcal{O}(\Delta t^4)$$
bzw.
\begin{equation*}
x(t+\Delta t) \approx 2 x(t) - x(t-\Delta t) + \frac{f(t)}{m} \Delta t^2 \tag{1}
\end{equation*}
Geschwindigkeit über:
$$x(t+ \Delta t) - x(t - \Delta t) = 2 v(t) \Delta t + \mathcal{O}(\Delta t^3)$$
\begin{equation*}
\Rightarrow v(t) \approx \frac{x(t+\Delta t) - x(t- \Delta t)}{2 \Delta t} \tag{2}
\end{equation*}
%Bilder
Gl. (1) und (2) bestimmen den Verlet-Algorithmus
\section{Beschleunigte Bezugsysteme}
Newton Gesetze gelten für Inertialsysteme (IS), Bezugsystem, das relativ zu IS beschleunigt ist, ist kein IS $\rightarrow$ es treten sogenannte Scheinkräfte auf.\\
z.B. Beschleunigung bei linearer Bewegung
\subsection{Rotierendes Bezugsystem:}
\underline{Geg:} IS $S$ mit $\vec{r}(t)$\\
und nicht-IS $S'$ mit $\vec{r}\,'(t)$, das gegenüber S mit einer Winkelgeschwindigkeit
\begin{equation*}
\vec{\omega} = \frac{d\vec{\varphi}}{dt} \tag{1}
\end{equation*}
rotiert, $\vec{\varphi}, \vec{\omega}$ zeigen in Richtung der Drehachse.\\
oEdA: $\vec{\varphi} \sim \vec{e}_z$\\
%zeichnungen
Berechne Vektor $\vec{G}$, der in $S'$ ruht. Änderung $d\vec{G}_{\tx{rot}}$ aufgrund Rotation
$$\vert d \vec{G}_{\tx{rot}} \vert = \vert d\vec{\varphi} \vert \vert \vec{G} \vert \sin \theta $$
$$d\vec{G}_{\tx{rot}} \perp \vec{\omega}  \qquad d\vec{G}_{\tx{rot}} \perp \vec{G}$$
\begin{equation*}
\rightarrow d\vec{G}_{\tx{rot}} = d\vec{\varphi} \times \vec{G} = (\vec{\omega}\; dt) \times \vec{G} \tag{2}
\end{equation*}
Beliebiger Vektor $\vec{G}(t)$, der sich in $S'$ während $dt$ um $d\vec{G}_{S'}$ ändert, ändert sich damit in S um:
$$d\vec{G}_{S} = d\vec{G}_{S'} + d\vec{G}_{\tx{rot}}$$
Damit:
\begin{equation*}
\boxed{\frac{d\vec{G}_{S}}{dt} = \frac{d\vec{G}_{S'}}{dt} + \vec{\omega} \times \vec{G}} \tag{3}
\end{equation*}
Für $ \vec{G} = \dot{\vec{r}}$ ist:
\begin{equation*}
\dot{\vec{r}} = \dot{\vec{r}}\,' + \vec{\omega} \times \vec{r}\,' \tag{4}
\end{equation*}
$$\vec{G = \dot{\vec{r}}} : \frac{d}{dt} \dot{\vec{r}} = \frac{d}{dt} (\dot{\vec{r}}\,' + \vec{\omega} \times \vec{r}\,' ) + \vec{\omega} \times (\dot{\vec{r}} + \vec{\omega} \times \vec{r}\,')$$
$$\ddot{\vec{r}} = \ddot{\vec{r}}\,' + \dot{\omega} \times \vec{\omega} \times \dot{\vec{r}}\,' + \vec{\omega} \times \dot{\vec{r}}\,' + \vec{\omega}\times \vec{\omega}\times \vec{r}\,'$$
Für $\omega= \const$ erhalten wir:
$$\ddot{\vec{r}} = \ddot{\vec{r}}\,' + 2(\vec{\omega} \times \dot{\vec{r}}\,') + \vec{\omega} \times ( \vec{ \omega} \times \vec{r}\,')$$
Für ein in S' kräftefreies Teilchen mit:
$$m\ddot{\vec{r}} = 0$$
erhalten wir dann:
$$m\ddot{\vec{r}}\,'= - \underbrace{2 m (\vec{\omega} \times \dot{\vec{r}}\,')}_{\tx{Corioliskraft}} -\underbrace{ m \vec{\omega}\times (\vec{\omega} \times \vec{r}\,')}_{\tx{Zentrifugalkraft}}$$
$$F_z \sim \omega^2r \textrm{, zeigt von Drehachse weg}$$
$$F_c \sim \omega\dot{r} \textrm{, steht} \perp \textrm{zur Bewegungsrichtung}$$
\underline{Bsp:} 
\begin{itemize}
	\item Erddrehung, Foucaultsches Pendel
	\item Ball auf Drehscheibe
\end{itemize}

\section{Mehr-Körper-Probleme}
Betrachte N Teilchen (Massenpunkte) mit:\\
Ort: $\vec{r}_i$, Masse $m_i$ und die auf sie wirkende Kraft $\vec{F}_i$ 
$$m_i \ddot{\vec{r}}_i = \vec{F}_i \qquad (i = 1, \dots , N)$$
Unterscheidung:\\
\underline{Innere Kräfte:} Kräfte der Teilchen aufeinander. z.B. Coulomb-Kräfte $\vec{F}_{ij}$ von (geladenen) Teilchen i und j\\
\underline{Äußere Kräfte:} $\vec{F}^A_i$ wirken von außen. z.B. Schwerkraft oder externes elektromagnetisches Feld.\\
$$m_i \ddot{\vec{r}}_i = \vec{F}_i^A + \sum _{j\neq i} \vec{F}_{ij}$$
\underline{Schwerpunktbewegung und Impuls}
Ortsvektor des Schwerpunktes
$$\vec{R} = \frac{1}{M} \sum_{i=1}^{N} m_i \vec{r}_i \qquad M = \sum_{i} m_i$$
Bewegungsgleichung für $\vec{R}$:
$$M\ddot{\vec{R}} = \sum_{i} m_i \ddot{\vec{r}}_i = \underbrace{\sum_{i} \sum_{j\neq i} F_{ij}}_{\textrm{3.NG:} F_{ij} = - F_{ji}}$$

% = MIT 3 STRICHEN

$$M \ddot{\vec{R}} = \sum_{i} \vec{F}_i^A + 0 =\vec{F}^A$$

\underline{Schwerpunktsystem:}\\ 
\indent \underline{Schwerpunkt bewegt sich nur gemäß äußerer Kräfte}\\ 
$\Rightarrow$ vergleiche Münchhausen-Trick\\
$\rightarrow$ Rechtfertigung der Idealisierung realer Körper durch Massepunkte

% Vorlesung 4

Auf ein \underline{abgeschlossenes System} wirken keine (oder vernachlässigbare) äußeren Kräfte\\
$$\rightarrow \frac{d}{dt} M \dot{\vec{R}} = 0 \quad \rightarrow \vec{P} = M \cdot \dot{\vec{R}} = \textrm{const}$$
$$\textrm{abgeschlossenes System} \leftrightarrow \textrm{Schwerpunktsystem ist erhalten}$$
\underline{Drehimpuls:} 
Vektorielle Multiplikation des 2. NG mit r$\vec{r}_i$ 
$$ \sum_{i} \vec{r}_i \times m_i\ddot{\vec{r}}_i = \sum_{i} \vec{r}_i \times \vec{F}_i$$
$$( \dot{\vec{r}}_i \times \dot{\vec{r}}_i = 0)$$
$$\frac{d}{dt} \underbrace{\sum_i m_i (\vec{r}_i \times \dot{\vec{r}}_i)} _{\vec{L} = \sum_i \vec{l}_i} = \underbrace{\sum_i \vec{r}_i \times \vec{F}_i^k}_{\vec{M} = \sum_i \vec{M}_i} + \underbrace{\sum_i \vec{r}_i \times \sum _{j\neq i} \vec{F}_{ij}}_{\buildrel ? \over = 0}$$
\begin{align*}
\sum_i \sum_{j\neq i} \vec{r}_i \times \vec{F}_{ij} = \ \  &\frac{1}{2} \sum_{i, j\neq i} (\vec{r}_i \times \vec{F}_{ij} + \vec{r}_j\times \vec{F}_{ji}) \\
\buildrel \textrm{3. NG} \over = &\frac{1}{2} \sum_{i,j\neq i} \underbrace{(\vec{r}_i - \vec{r}_j) \times \vec{F}_{ij}}_{\textrm{Annahme: } \vec{F}_{ij} \parallel \vec{r}_i - \vec{r}_j} = 0
\end{align*}
d.h. innere Kräfte ergeben kein resultierendes Drehmoment.
$$\frac{d\vec{L}}{dt} = \sum_i \vec{r}_i \times \vec{F}_i^A = \vec{M}  $$
$$\textrm{abgeschlossenes System} \leftrightarrow \textrm{Gesamtdrehimpuls } \vec{L} \textrm{ ist erhalten}$$
\underline{Energie:}
Mult. von 2.NG mit $\dot{\vec{r}}_i$ 
\begin{align*}
\sum_i m_i \ddot{\vec{r}}_i \cdot \dot{\vec{r}}_i &= \frac{d}{dt} \underbrace{\sum_i \frac{m_i}{2} \cdot \dot{\vec{r}}_i\,^2}_{T} = \frac{dT}{dt} \\
&= \sum_i (\vec{F}_{i}^{\tx{kons}} + \vec{F}_{i}^{\tx{diss}}) \dot{\vec{r}}_i
\end{align*}
wobei
$$ \sum_i \vec{F}_i^{\tx{kons}} \cdot \dot{\vec{r}}_i = -\frac{dU(\vec{r}_1,\dots , \vec{r}_N)}{dt} = - \sum_i \frac{\partial U}{\partial \vec{r}_i} \cdot \dot{\vec{r}}_i$$
mit
$$\frac{\partial U (\vec{r}_1, \dots , \vec{r}_N)}{\partial \vec{r}_i} = \frac{\partial U}{\partial x_i} \vec{e}_x + \frac{\partial U}{\partial y_i } \vec{e}_y + \frac{\partial U}{\partial z_i } \vec{e}_z$$
\underline{Energiesatz:}
$$ \frac{d}{dt} (T+U) = \sum_i \vec{F}^{\tx{diss}}_i \cdot \dot{\vec{r}}_i$$
$$\textrm{Kräfte konservativ} \leftrightarrow E =  T + U \textrm{ erhalten}$$
Energieerhaltung gilt also auch bei äußeren Kräften, solange sie konservativ sind.\par
\underline{Aufteilung:}
\begin{align*}
\vec{F}^{\tx{kons}}_i &= \vec{F}_i^A (\vec{r}_i) + \sum_{i\neq j} F_{ij}^I (\vec{r}_i,\vec{r}_j)\\
&= -\frac{\partial U^A(\vec{r}_1,\dots , \vec{r}_N)}{\partial \vec{r}_i} - \frac{\partial U^F(\vec{r}_1,\dots , \vec{r}_N)}{\partial \vec{r}_i}
\end{align*}
mit 
\begin{align*}
U^A &= \sum_i U_i (\vec{r}_i), \quad \vec{F}^A_i = -\frac{\partial U_i(\vec{r}_i)}{\partial \vec{r}_i} \qquad &\begin{Large}\substack{\tx{äußere Kräfte wirken} \\ \tx{auf einzelne Teilchen}}\end{Large} \\\\
U^I &= \sum_{i < j} U_{i,j} (\vec{r}_i,\vec{r}_j) \qquad  & {\tx{Annahme: 2 Teilchen WW.}}
\end{align*}
mit $ \frac{\partial U_{ij}}{\partial \vec{r}_i} \buildrel 3.NG \over = -\frac{\partial U_{ij}}{\partial \vec{r}_j} \textrm{ hängt U nur von } \vec{r}_i - \vec{r}_j \textrm{ ab }$\\
Annahme: $\vec{F}_{ij} \parallel \vec{r}_i - \vec{r}_j = \vec{r}_{ij} \quad \rightarrow \textrm{hängt nur von } r_{ij} = | \vec{r}_{ij}| \textrm{ ab }$
$$ \vec{F}_{ij}^I = - \frac{\partial U^I_{ij} (r_{ij} = |\vec{r}_i - \vec{r}_j|)}{\partial \vec{r}_{ij}}$$
$$ U(\vec{r}_1 , \dots , \vec{r}_N) = \sum_i U_i (\vec{r}_i) + \sum _{i<j} U_{ij} (|\vec{r}_i- \vec{r}_j |)$$
\underline{z.B.:} $$U_i = q_i \Phi (\vec{r}_i)$$
$$U_{ij} = \frac{q_i q_j}{|\vec{r}_i - \vec{r}_j |}$$
\underline{Allgemein:} Abgeschlossene N-Teilchen System $(N\ge 2)$ mit ausschließlich konservativen Kräften haben also mindestens 10 Erhaltungsgrößen:\\\\
\begin{tabular}{lll}
	$\bullet$ der Gesamtimpuls &  $\vec{P} = \sum_i m_i \dot{\vec{r}}_i \quad (*)$	& 3 Größen \\ 
	$\bullet$ ein Vektor, & $M\vec{R} - \vec{P} \cdot t\quad $ (Int. von $(*)$)	& 3 Größen \\ 
	\multicolumn{2}{l}{\ \,  der die Schwerpunktsbewegung beschreibt}  \\
	$\bullet$ der Gesamtdrehimpuls & $\vec{L} = \sum_i m_i \vec{r}_i \times \dot{\vec{r}}_i$	& 3 Größen \\ 
	$\bullet$ und die Gesamtenergie & $ E = T+U$	& 1 Größe \\
	\\[-10pt]
	\hline\\[-10pt]
	& & \underline{10 Größen}
\end{tabular} \\\\
Selten mehr: z.B. Lenzscher Vektor im Keplerproblem

\section{Die Hamilton-Funktion (1833)}
\underline{Newton:} Kraft ist zentrale Größe\\
\underline{Hamilton:} Energie ist zentrale Größe\\
Gegeben sei N-Teilchen System mit ausschließlich konservativen Kräften
\begin{equation*}
F_i = - \frac{\partial U(r)}{\partial r_i} = \dot{p}_i \qquad \tx{(3 DGL 2. Ordnung)} \tag{1}
\end{equation*}
\begin{small}
	$$r \equiv (r_1,r_2,\dots ,r_{3N}) \quad p = (p_1, p_2, \dots , p_{3N}) \quad i = (1,\dots, 3N)$$
\end{small}
$$T = \sum_i \frac{m_i}{2} v_i^2 \quad \buildrel p_i = m_i v_i \over = \quad \sum_i \frac{p_i^2}{2m_i} \quad \ \  \textrm{, Potential U} $$
Gesamtenergie $E = T+U$ wird durch die Hamilton-Funktion beschrieben: ,,Hamiltonian``:
\begin{equation*}
\begin{Large}
\rmbox{\begin{array}{ll}
	H &= H(r,p) = T(p) + U(r)\\[15pt]
	&= \sum\limits_{i=1}^{3N} \frac{p_i^2}{2m_i} + U(r)
	\end{array}}
\end{Large}
\tag{2}
\end{equation*}
Mit 
$$\frac{\partial H}{\partial r_i} = \frac{\partial U}{\partial r_i} \buildrel (1) \over = - \dot{P}_i$$
$$ \frac{\partial H}{\partial p_i} = \frac{p_i}{m_i} = \dot{r}_i \quad \ $$
$$m_1 =m_2 = m_3 = \textrm{ Masse des ersten Teilchens} \ \  $$
$$m_4 =m_5 = m_6 = \textrm{ Masse des zweiten Teilchens}$$
$$ \vdots $$
äquivalent zu den Newton Gleichungen: Hamilton-Gleichungen:
Bewegungsgleichungen im Hamilton-Formalismus
\begin{equation*}
\begin{LARGE}
\rmbox{\begin{array}{ll}
	r_i &= \frac{\partial H}{\partial p_i}\\[15pt]
	\dot{p}_i &= -\frac{\partial H}{\partial r_i}
	\end{array}}
\end{LARGE} \tag{3}
\end{equation*}
6N DGL 1. Ordnung (Hamilton)\\
3N DGL 2. Ordnung (Newton)\\
$$\frac{d}{dt} H(r(t),p(t)) = \sum_i \bigg[ \frac{\partial H}{\partial p_i} \underbrace{\dot{p}_i}_{ -\partial H / \partial r_i} + \frac{\partial H}{\partial r_i} \underbrace{\dot{r}_i}_{\partial H / \partial p_i} \bigg] = 0$$
d.h. Energie ist erhalten.
\underline{Bem:}
\begin{itemize}
	\item Im Gegensatz zu vektoriellen Kräften ist der Hamiltonian (die Hamilton Funktion) ein \underline{Skalar}, und damit wesentlich leichter aufzustellen.
	\item Hamilton-Funktion kann für allgemeine Fälle (z.B. geschw. abhängige Potentiale oder zeitabhängige Potentiale (siehe später)) und hat dann nicht notwendigerweise die Bedeutung der Gesamtenergie.
	\item $(r,p)$ bilden den 2.3N dimensionalen \underline{Phasenraum} der die Bewegung \underline{vollständig} beschreibt.
\end{itemize}

\underline{Bsp: harmonischer Oszillator}

% Blatt 2

\underline{gedämpfter harm. Oszillator}

% Blatt 2

\underline{ebenes Pendel mit überschlag}

% Blatt 2

\subsection*{Newtonsche Mechanik: theoretisches Konzept}
\addcontentsline{toc}{subsection}{Newtonsche Mechanik: theoretisches Konzept}
\underline{1.Def:}
\begin{itemize}
	\item Masse, Kraft, Energie
	\item Innertialsystem, beschleunigte Bezugssysteme
	\item konservative/ dissipative, innere/ äußere Kräfte
\end{itemize}
\underline{2. Bew. Gl.:}
N Teilchen, konservative Kräfte mit Pot $U(r)$
$$ r ? (r_1, \dots, 1_{3N})$$
$$m_i\ddot{r}_i = - \frac{\partial U(r)}{\partial r_i}$$ 
gewöhnlich = DGL 2. Ord i. A. nicht linear\\
\underline{3. Erhaltungssätze}
Schwerpunkt $\vec{R}, \vec{P}, M = \sum_i m_i$
\begin{align*}
&\bullet \ \frac{d}{dt} \vec{p} = M \cdot \ddot{\vec{R}} = \underbrace{\vec{F}^A} _{\textrm{äußere Kraft}} \buildrel \textrm{abgeschlossenes System} \over = \quad 0 \leftrightarrow \vec{p} = \textrm{const.} \\
&\indent \textrm{Impulserhaltung}\\[10pt]
&\bullet \ \frac{d}{dt} \vec{L} = \sum_i \vec{r}_ \times \vec{F}_i^A = \vec{M} \rightarrow 0 \leftrightarrow \vec{L} = \textrm{const.} \quad \textrm{Drehimpulserhaltung}\\[10pt]
&\bullet \ \frac{d}{dt} (T + U) = \sum_i \vec{F}_i^{diss} \dot{\vec{r}}_i \buildrel\textrm{Kräftegleichgewicht} \over \longrightarrow 0\\
&\indent E = T+U = \textrm{const.} \quad \textrm{Energieerhaltung}
\end{align*}
\underline{Alternativ:}
Hamilton Funktion mit Impuls $p=(p_1,\dots, p_{3N}) \widehat{=}$ Gesamtenergie
$$H(r,p) = T(p) + U(r) = \sum_i \frac{p_i^2}{dm} + U(r)$$
$$\rightarrow \dot{r_i} = \frac{\partial H}{\partial p_i}, \quad \dot{p_i}= -\frac{\partial H}{\partial r_i} \qquad 2\times 3 N \textrm{DGL 1. Ord}$$
$(r,p)$ bilden den 2 3N dim Phasenraum
\begin{itemize}
	\item Wichtig für Übergang zur QM und statistischen Mechanik
	\item Beschreibt Bewegung vollständig, d.h. geg. Bew. Gl. mit Anfangsbedingungen $r_i(0), p_i(0)$, so ist $r(t)$ und $p(t)$ \underline{für alle Zeiten vollständig} bestimmt. Man sagt die klassische Mechanik ist deterministisch.
\end{itemize}
\subsubsection{Nichtlineare Dynamik und Chaos}
\begin{itemize}
	\item Man kann zeigen: Existieren für ein System mit 2f-dim Phasenraum f Erhaltungsgrößen, so heißt das System Integrabel.\\
	\underline{Bsp.:}
	\begin{itemize}
		\item[1)] Konservative Bewegung in 1D $\rightarrow f=1$ \\
		Energie ist erhalten $\rightarrow$ System ist Integrabel
		\item[2)] 2-Körperproblem: $f=6$ \\
		Erhaltung von Energie, Gesamtimpuls, $\vec{L}^2, L_z \rightarrow$ System ist integrabel
		\item[3)] 3-Körperproblem: $f=9$\\
		6 Erhaltungsgrößen $\rightarrow$ i.A. nicht integrabel, \underline{kann chaotisch} sein (Poincare' um 1900)
	\end{itemize}
	\item \underline{Grund}\\
	Nichtlineare Bwe. Gl. können instabile Lösungen haben d.h. bei \\\underline{geringstfügiger} Änderung der Anfangsbedingungen zeigt System für \\\underline{lange Zeiten} eine qualitativ andere Bewegung: ,,\underline{Schmetterlingseffekt}`` sog. deterministisches Chaos
	\item \underline{Bedingung} für chaotisches Verhalten
	\begin{itemize}
		\item Anzahl der Freiheitsgrade $f \ge2$ 
		\item Nichtlinearität der Kraft
		\item\underline{Bsp:} Vergleiche\\
		harm Oszillator $F\sim r, U(r) \sim r^2$ \\
		mit stabilem Fixpunkt und $H(r)$\\
		Pendel: $F\sim \sin \varphi, U \sim \cos \varphi$ und $ U(\varphi)$ \\
		mit stabilen Fixpunkten in sinus Tälern und instabilen Fixpunkten auf den sinus Bergen
	\end{itemize}
\end{itemize}

\section{Schwingungen}
Harm. Oszillator ist ein zentrales Modell der Physik
\begin{itemize}
	\item analytisch lösbar, auch mit Reibung und Antrieb und in vielen Dimensionen
	\item lineares System
\end{itemize}

1D System mit Harm. Fkt.
$$ H(p,q) = \frac{p^2}{2m} + U(q)$$
das bei $q = q_0$ eine stable Gleichgewichtslage besitzt.\\
Idee(,,harmonische Näherung``) : Taylor Entwicklung von $k$ um $q_0$
$$U(q) = \underbrace{U(q_0)}_{\textrm{oEdA } =\  0} + \underbrace{\frac{dU}{dq} \bigg|_{q_0}}_{\substack{ =\  0 \tx{,da}\\ \textrm{GGWlage}}} (q-q_0) + \frac{1}{2} \underbrace{ \frac{d^2 U}{d q^2} \bigg|}_{= k} (q-q_0)^2 + \dots$$
$$\approx \frac{1}{2} k (q-q_0)^2 \equiv \frac{k}{2} x^2$$
\underline{Bew Gl:} 
$$ \dot{x} = \frac{\partial H}{\partial p }= \frac{p}{m} \qquad; \qquad \dot{p} = - \frac{\partial H}{\partial x} = -kx$$
oder: $$ m\ddot{m} + kx = 0$$
Lösungen sind $\sin \omega t, \cos \omega t$ mit $\omega^2 = \frac{k}{m}$
Die allg. Lösung mit Anfangsbedingungen $x_0 = x(0) , p_0 = 0(0)$
$$x(t) = x_0 \cos \omega t + \frac{p_0}{m \omega} \sin \omega t$$
$$p(t) = p_0 \cos \omega t - m x_0 \omega \sin \omega t$$
Mit $$ e^{ix} = \cos x + i \sin x \qquad (i^2 = -1)$$
$$x(t) = Re(A e^{i\omega t} + B e^{i\omega t} )$$
Im folgenden werden Schwingungen in 1D mit Reibung:
$$ F_R = -\gamma \dot{x}(t)$$
und einer Zeitabhängigen externen Kraft:
$$F_{ext}(t) \quad \textrm{,,Antrieb``}$$
bezeichnet.
\subsection{Gedämpfte Schwingungen}
$$\ddot{x}(t) + \gamma \dot{x}(t) + \omega^2 x(t) = 0 $$
\underline{Grenzfälle:}
\begin{itemize}
	\item $ \gamma = 0 \rightarrow$ harm. Oszillator, $x \sim e^{\pm i \omega t}$
	\item $\omega = 0, \quad \ddot{x} + \gamma\dot{x} = 0, \quad v= \dot{x}$\\
	$\dot{v} = -\gamma v \quad \rightarrow \quad v \sim e^{-\gamma t}$
\end{itemize}
\underline{Ansatz:} $$x(t) = e^{\lambda t} \quad , \quad \lambda = a+ ib \in \mathbb{C}$$
\underline{eingesetzt:} $$(\lambda^2 + \gamma\lambda + \omega^2) e^{\lambda t} = 0$$
$\lambda^2 + \gamma \lambda + \omega^2 = 0 $ charakteristische Gleichung
$$\lambda_{1/2} = - \frac{\gamma}{2} \pm \frac{1}{2} \sqrt{\gamma^2 - 4\omega^2} = -\frac{\gamma}{2} \pm \frac{\sqrt{D}}{2}$$
$\rightarrow$ i.A. 2 Lösungen $x_0 (t) = e^{\lambda_1 t}, x_2(t) = e^{\lambda_2 t}$\\
\underline{$D<0$ :}
d.h. $\gamma < 2 \omega$, komplexe Lösungen
$$\lambda_{1/2} = - \frac{\gamma}{2} \pm \frac{1}{2} \sqrt{ 4\omega^2 - \gamma^2}$$
\underline{allg. Lösung:}
$$x(t) = e^{-\gamma \frac{1}{2} t (c_1 e^{i\Omega t} + c_2 e ^{i \Omega t})}$$
$\gamma$ bewirkt Dämpfung und Änderung der Frequenz\\
\underline{$D>0$ :}
$\gamma > 2\omega$: ,,Überdämpfte Schwingung``
$$\rightarrow x(t) = c_1 e^{i\lambda_2 t} + c_2 e ^{i \lambda_2 t}$$
\underline{$D = 0$:}
$\gamma = 2 \omega$\\
$\rightarrow$ wir erhalten 
$$x(t) = C e^{-\frac{\gamma}{2}t}$$
d.h. nur eine Lsg. anstelle von 2 unabhängigen Lsg.\\
\underline{Variation der Konstanten:}
$$x(t) = C(t) e^{- \frac{\gamma}{2}t}$$
$$\dot{x} = (\dot{C} - \frac{\gamma C}{2}) e ^{-\frac{\gamma}{2}t}$$
$$ \ddot{x} = (\ddot{C} - \gamma\dot{C} + \frac{\gamma^2 C }{4}) e ^{-\frac{\gamma}{2}t}$$
\underline{eingesetzt:}
\begin{align*}
0 & = (\ddot{C} - \gamma \dot{C} + \frac{\gamma^2 C}{4} + \gamma \dot{C} - \frac{\gamma^2 C}{2} + \omega^2 C ) e ^{-\frac{\gamma}{2}t}\\
& = ( \ddot{C} + C\underbrace{(\omega^2 - \frac{\gamma^2}{4})}_{=v} ) e ^{-\frac{\gamma}{2}t}
\end{align*}
$$\rightarrow \ddot{C} = 0 , C(t) = C_0 + C_1 t$$
$\rightarrow$ \underline{allg:} 
$$x(t) = C_0  e ^{-\frac{\gamma}{2}t} + C_1 t  e ^{-\frac{\gamma}{2}t}$$
,,kritische gedämpfte Schwingung``\\
\underline{Bem:}
\begin{itemize}
	\item Allg. \underline{lineare} DGL (mit konstanten Koeffizienten) n-ter Ordnung können durch einen Exponentialansatz gelöst werden.\\
	$\rightarrow$ charakteristische Gl. ist Polynom vom Grad n
	\item Physikalische Bedeutung der Nullstellen von Polynomen:\\
	\begin{tabular}{ccc}
		Komplexe Nullstellen & $\widehat{=}$ & oszillatorische Lösungen \\
		Reelle Nullstellen & $\widehat{=}$ & zerfallende Lösungen
	\end{tabular}
\end{itemize}

\subsection*{Gedämpfte Schwingungen (Wiederholung)}
$$\ddot{x} + \gamma \dot{x} + \omega^2 x = 0$$
$$ x(t) = c_1 e^{\lambda_1 t} + c_2 e^{\lambda_2 t}$$
$$\lambda_{1/2} = \ub{- \gamma/2}_{\tx{Dämpfung}} \pm \frac{1}{2} \underbrace{\sqrt{\gamma^2 - 4 \omega^2}}_{\substack{\textrm{ für } \omega > 2 \gamma \rightarrow \textrm{schwingung} \\ \textrm{sonst Dämpfung}}}$$ 
%bild 1 gedämpft schwingung
%bild 2 überdämpfte schwingung

\subsection{Der getriebene Oszillator}
$$\underbrace{\ddot{x} + \gamma \dot{x} - \omega^2_0 x}_{\textrm{homogene DGL}} = f(t) \quad \substack{\textrm{externer Antrieb } f(t) \\ \textrm{``Inhomogenität'' der DGL}}$$
\underline{Lösung der DGL:}
$$\underbrace{x_{tot}}_{\textrm{bereits bekannt}} = \underbrace{x_{hom}(t)}_{\textrm{partikuläre Lösung}} + x (t)$$
Betrachten periodischen Antrieb
$$f (t) = f \cos(\omega t) = \frac{f}{2} ( e^{i \omega t} + e^{-i \omega t})$$
bzw.
$$f(t) = fe^{\pm i \omega t}$$
mit \underline{Exp. ansatz:} $x(t) = A \pm e^{\pm i \omega t}$\\
\underline{Eingesetzt:} [$\omega \pm i \omega \gamma + \omega_0^2$] $A \pm \quad = \quad f$ \quad [Ermitteln mit $z^{\star}$]
\begin{align*}
\rightarrow \quad A \pm =& f \frac{\omega^2_0 - \omega^2 \mp i \gamma \omega}{(\omega_0^2 - \omega^2)^2 + \omega^2 \gamma^2}\\[10pt]
\rightarrow \quad x(t) =& \frac{1}{2}(A_+e^{i \omega t}+A_- e^{-i \omega t})\\[10pt] 
=& f\frac{(\omega_0^2 - \omega^2) \cos \omega t + \gamma \omega \sin \omega t}{(\omega_0^2 - \omega^2)^2 + \omega^2 \gamma^2} = a \cos \omega t + b \sin \omega t\\[10pt]
=& A \cos (\omega t - \varphi)
\end{align*}
\begin{align*}
&A = \sqrt{a^2 +b^2} \rightarrow A(\omega) = \frac{f}{\sqrt{(\omega^2_0 - \omega^2)^2 - \omega^2 \gamma^2}} \quad &\textrm{Amplitude}\\[15pt]
&\tan(y) = \frac{a}{b} \rightarrow y(\omega) = \arctan\bigg(\frac{\gamma \omega}{\omega^2_0 - \omega^2}\bigg ) \quad &\textrm{Phase}
\end{align*}


%bild 3 A/(f/m) 0 1 2 graph // bild 4 q/pi graph


Resonanz bei $\omega = \omega_0$, $A$ groß für $\gamma$ klein
$\gamma \to 0$ und $\omega \to \omega_0$ 	„Resonanzkatastrophe“\\
z.B. Brücke, Tacoma Narrows Bridge 1940\\
Gesamtlösung $$x_{tot}^{(t)} = x_{hom}^{(t)} - x (t)$$
$$\buildrel{t \to \infty}\over{\rightarrow} x(t) \quad \textrm{,,Stationäre Lösung``} $$
Resonanz wichtig: \begin{itemize}
	\item[-] Schwingende Karrosserieteile
	\item[-] Schwingkreis (E-Dynamik)
	\item[-] Molekülschwingungen etc.
\end{itemize}

%bild 5 2 gekoppelte Oszillatoren und wänden


%markus arrival

\subsection{gekoppelte Oszillatoren}
$$
\begin{tikzpicture}
\node[circle,fill=black,inner sep=1.5mm] (a) at (0,0) {};
\node at (0,.5){$m$};
\node at (2,.5){$m$};
\node at (-1,-0.8){$c$};
\node at (3,-0.8){$c$};
\node at (1,-0.8){$k$};
\draw[thick][->](2,-1)--(2.5,-1);
\draw[thick] (2,-1.25)--(2,-.75);
\node at (2.25,-1.5){$x_2$};
\draw[thick][->](0,-1)--(0.5,-1);
\draw[thick] (0,-1.25)--(0,-.75);
\node at (0.25,-1.5){$x_1$};
\node[circle,fill=black,inner sep=1.5mm] (b) at (2,0) {};
\draw[decoration={aspect=0.3, segment length=3mm, amplitude=3mm,coil},decorate] (-2,0)--(a);
\draw[decoration={aspect=0.3, segment length=3mm, amplitude=3mm,coil},decorate] (b)--(4,0);
\draw[decoration={aspect=0.3, segment length=3mm, amplitude=3mm,coil},decorate] (b)--(a);
\filldraw[pattern=north east lines] (-2.5,-1) rectangle (-2,1);
\filldraw[pattern=north east lines] (4,-1) rectangle (4.5,1);
\end{tikzpicture}
$$
$m_1$ und $m_2$ sind aneinander und an zwei wänden mit federn gekoppelt (äußere federn mit c und innere feder mit k)
\begin{align*}
m\ddot{x}_1 &= - cx_1 + k (x_2 - x_1) \tag{1}\\
m\ddot{x}_2 &= - cx_2 - k (x_2 - x_1) \tag{2}
\end{align*}
System von gekoppelten DGL.
Mit $$ y_1 = \frac{1}{\sqrt{2}}(x_1+x_2) \qquad y_2 = \frac{1}{\sqrt{2}} (x_1-x_2)$$
ist $(1) + (2)$: $$m\ddot{y}_1 = -c y_1$$
$(1) - (2)$: $$m\ddot{y}_2 = -(c+2k) y_2$$
$\rightarrow$ entkoppelte DGL\\
mit Frequenzen $$ \omega_1 = \sqrt{\frac{c}{m}} \quad ,\quad \omega_2 = \sqrt{\frac{c+2k}{m}}$$
Anfangsbedingungen: z.B. $X_2(0) = a, \quad x_1(0) = \dot{x}_1(0) = \dot{x}_2(0) = 0$
$$y_1(t) = \frac{a}{\sqrt{2}} \cos \omega_1 t \quad , \quad y_2 = -\frac{a}{\sqrt{2}} \cos \omega_2 t \textrm{ Eigenschwingungen }$$
eingesetzt:
\begin{align*}
x_1(t) = \frac{a}{2} ( \cos \omega_1 t - \cos \omega_2 t)\\
x_2(t) = \frac{a}{2} ( \cos \omega_1 t + \cos \omega_2 t)
\end{align*}
\underline{Zusammenhang mit Eigenwertproblem}
$$ m\vec{\ddot{x}} = - V \vec{x}$$
$$ \vec{x} = \begin{pmatrix}
x_1 \\
x2
\end{pmatrix} \quad , \quad V = \begin{pmatrix}
c+k & -k \\
-k & c+k
\end{pmatrix}$$
Ansatz:
$$ x_1(t) = a_1 e^{i\omega t}$$
$$ \ddot{x}_i (t) = - \omega^2 a_i e^{i \omega t}$$
eingesetzt:
$$ -m \omega^2 e ^{ i \omega t} \begin{pmatrix}
a_1\\
a_2
\end{pmatrix} = - V \begin{pmatrix}
a_1\\
a_2
\end{pmatrix} e ^{i \omega t}$$
mit $\lambda = m \omega^2$
$$ 1 = \begin{pmatrix}
1 & 0 \\
0 & 1
\end{pmatrix} \qquad \vec{a} = \begin{pmatrix}
a_1\\
a_2
\end{pmatrix} \qquad (V- \lambda 1 ) \vec{a} = 0$$
Lösung für $$\det (V - \lambda 1 ) = 0$$

% einrahmen und umschreiben v a = lambda a

$$ (c+k-\lambda)^2 - k^2 = 0$$
$$ \lambda^2 - 2(c+k)\lambda + (c-k)^2 - k^2 = 0$$
,,Eigenfrequenzen`` Eigenwerte
\begin{align*}
\rightarrow \lambda_1 &= c \quad \omega_1 = \sqrt{\frac{c}{m}} \\
\lambda_2 &= c+2k \quad \omega_2 = \sqrt{\frac{c+2k}{m}}
\end{align*}
Lösung der Eigenwertgleichung (1) für die Eigenwerte ergibt die ,,Eigenvektoren`` hier Eigenschwingungen
$$ \vec{a}_1 = \frac{1}{\sqrt{2}} \begin{pmatrix}
1 \\
1
\end{pmatrix} \qquad \vec{a}_2 = \frac{1}{\sqrt{2}} \begin{pmatrix}
1 \\
-1
\end{pmatrix}$$

$$\rmbox{\underbrace{V}_{\substack{\textrm{Operator} \\ \textrm{Abbildung} \\ \textrm{Matrix}}} \cdot \underbrace{\vec{a}_i}_{\textrm{Eigenvektor}} = \underbrace{\lambda_i}_{\textrm{Eigenwert}} \cdot \underbrace{\vec{a}_i}_{\textrm{Eigenvektor}}}$$

\underline{Eigenschwingungen}\\
z.B. betrachte Molekül mit N Atomen (nicht lineare Mol)(für lineare Mol. $3N-5$ FG)
$$3N - 3 - 3 = 3N - 6$$
-3 Translation -3 Rotation also $3N-6$ Freiheitsgrade für innere Bewegung
$f = 3N-6$ innere FG $x_1, \dots x_f$\\
mit Gleichgewichtslage $x_1^{(0)}, \quad \rightarrow \frac{\partial V}{\partial x_i} \vert _{x_i^{(0)}} = 0$\\
und Potential $V(x_1,\dots, x_f)$\\
Entwickle V um $\vec{x}^{(0)}$\\
$$\underbrace{V(x_1,\dots , x_f)}_{\textrm{oEdA }= 0} = V(x_1^{(0)},\dots , x_f^{(0)}) + \sum_i \underbrace{\bigg(\frac{\partial V}{\partial x_i}\bigg)_{x_i^{(0)}}}_{= 0} (x_i - x_i^{(0)})$$
$$ T = \frac{1}{2} \sum _{i,j} \ub{\bigg(\frac{\partial^2 V}{\partial x_i \partial x_j} \bigg)_0}_{V_{ij}} \ub{(x_i - x_i^{(0)} )}_{\equiv x_i}  \ub{( x_j - x_j^{(0)})}_{\equiv x_j} + \dots $$
Hessematrix: $V = \frac{1}{2} \sum_{i,j} V_{ij} x_i x_j  $ harmonische Näherung
\underline{Bew. GL:} $$m_i\ddot{x}_i = -\frac{\partial V}{\partial x_i} = - \sum_i V_{ij} x_j$$
massenbehaftete Koordinaten: $ q_i = \sqrt{m_i} x_i$
$$\rightarrow \boxed{\ddot{q}_i + \sum_i V_{ij} q_j = 0} \qquad V_{ij} = \frac{V_{ij}}{\sqrt{m_i m_j}}$$
\underline{Exp. Ansatz:} $q_i (t) = q_i r^{i\omega t}$
$$\rightarrow \sum_i V_{ij} a_J - \omega^2 a_i = 0 \textrm{ charakteristische Gl.}$$
Mit $$\mathcal{V} = \{V_{ij}\} \quad , \quad \vec{a} = \begin{pmatrix}
a_1\\
a_2\\
\vdots
\end{pmatrix}$$
ist $(\mathcal{V} - \lambda 1)\vec{a}=0 $ Eigenwertproblem

\subsection{Eigenschwingungen}
$\mathcal{V} = \{v_{ij}\} \qquad i,j = 1, \dots ,N \qquad \vec{a} = (a_1,\dots,a_N)$\\
Eigenwertproblem: $ \boxed{\mathcal{V} \vec{a} = \lambda \vec{a}}  \qquad \lambda = \omega^2$\\
\underline{Eigenwerte $ \lambda_k $} $\det(\mathcal{V}-\lambda1) \buildrel ! \over = 0 \qquad (1)$\\
charakteristische Gleichung., Polynom N-ter Ordung\\
\underline{Eigenvektoren $ \vec{a}_k $:} Lösungen (1) mit $ \lambda = \lambda_k  $\\
orthogonal (bzw. Orthonormal)\\
$$\vec{a}_i \cdot \vec{a}_j = \delta_{ij} = \left\{ \begin{array}{ll}
1 & i=j \\
0 & \tx{sonst}
\end{array}\right.$$
$\rightarrow$ Eigenwertproblem $A = (\vec{a}_{1},\dots , \vec{a}_{N})$ ist orthogonal
$$A^\top = A^{-1} \rightarrow A A^\top = 1 A^\top A$$
A ist diagonalisierbar die Hessematrix $ \mathcal{V} $
$$\boxed{A^{\top}V A = \tx{diag}(\lambda_1,\dots, \lambda_N) = \varLambda}$$
$$\rightarrow \tx{\underline{Lösungen: }} \quad   q_i(t) = a_{ik} e ^{i \omega_k t} \quad (k=1,\dots,N)$$
\underline{Allg. Lsg.:} $$q_i(t) = \sum_k C_k \ q_{ik} \ e^{i\omega_k t}$$
mit Koeffizienten $C_k$ aus Anfangsbedingungen
$$\boxed{q_i(t) = \sum_k a_{ik} \ Q_k(t) \leftrightarrow \vec{q} = A \vec{Q}}$$
$$\boxed{Q_{k}(t) = C_k e^{i\omega_k t}} \quad \tx{,,Eigenschwingung`` oder Normalmoden}$$
\underline{Pot.Energie:}
\begin{align*}
U &= \frac{1}{2} \sum_{i,j} v_{ij} q_i q_j = \frac{1}{2} \vec{q} V \vec{q}\\
&= \frac{1}{2} \vec{Q}^\top \ub{ A^\top V A}_{\varLambda} \vec{Q} = \frac{1}{2} \vec{Q}^\top \varLambda \vec{Q}\\
&= \frac{1}{2} \sum_{n=1}^{N} w_k^2 Q_k^2 \quad \tx{pot. Energie in } \vec{Q}_k \tx{ ist diagonal}
\end{align*}
\begin{align*}
T &= \frac{1}{2} \sum_i \dot{q}_i^2 = \frac{1}{2} \dot{\vec{q}}^\top \dot{\vec{q}} \\
&= \frac{1}{2} \dot{Q}^\top \ub{ A^\top A }_{= 1} \dot{\vec{Q}} = \frac{1}{2} \dot{\vec{Q}}^\top \dot{\vec{Q}}\\
&= \frac{1}{2} \sum_{n=1}^{N} \dot{Q}_k^2 \quad \tx{ kin. Energie ist auch Diagonal}
\end{align*}
Mit $P_k = \dot{Q}_k$ ist damit der Hamiltonian
$$\boxed{H(Q,P) = \frac{1}{2} \sum_{k=1}^{N} ( p^2_k + \omega_k^2 Q^2_k)}$$
der in ein System \underline{unabhängiger} Oszillatoren \underline{seperiert.}
\subsubsection{\underline{Bsp:} Normalmoden von Wasser}
$$
\begin{tikzpicture}
\node[circle,fill=red, draw=black,inner sep=1.5mm] (a) at (0,0) {};
\node[circle,fill=red, draw=black,inner sep=1.5mm] (b) at (5,0) {};
\node[circle,fill=red, draw=black,inner sep=1.5mm] (c) at (10,0) {};
\node[circle,fill=white, draw=black, inner sep=1mm] (a1) at (-1,-1) {};
\node[circle,fill=white, draw=black, inner sep=1mm] (a2) at (1,-1) {};
\node[circle,fill=white, draw=black, inner sep=1mm] (b1) at (4,-1) {};
\node[circle,fill=white, draw=black, inner sep=1mm] (b2) at (6,-1) {};
\node[circle,fill=white, draw=black, inner sep=1mm] (c1) at (9,-1) {};
\node[circle,fill=white, draw=black, inner sep=1mm] (c2) at (11,-1) {};
\draw[thick,-] (a1) -- (a);
\draw[thick,-] (a2) -- (a);
\draw[thick,-] (b1) -- (b);
\draw[thick,-] (b2) -- (b);
\draw[thick,-] (c1) -- (c);
\draw[thick,-] (c2) -- (c);
\node at (0,.5) {O};
\node at (-1.25,-1.5) {H};
\node at (1.25,-1.5) {H};
\node at (5,.5) {O};
\node at (5-1.25,-1.5) {H};
\node at (5+1.25,-1.5) {H};
\node at (10,.5) {O};
\node at (10-1.25,-1.5) {H};
\node at (11.25,-1.5) {H};
\draw [<->,black,thick] (0.5,-1) to [out=-150,in=-30] (-0.5,-1);
\draw [<->,black,thick] (5-0.25-.15,0+.15) to (4-.15,-1+0.25+.15);
\draw [<->,black,thick] (5+0.25+.15,0+.15) to (6+.15,-1+0.25+.15);
\draw [<-,black,thick] (10-0.25-.15,0+.15) to (9-.15,-1+0.25+.15);
\draw [->,black,thick] (10+0.25+.15,0+.15) to (11+.15,-1+0.25+.15);
\node at (0,-3) {Biegeschwingung};
\node at (5,-2.5) {Symmetrische-};
\node at (10,-2.5) {Asymmetrische-};
\node at (7.5,-3.25) {streckschwingung};
\end{tikzpicture}
$$

\noindent
Bilder: Wasser Molekül mit Biegeschwingung und symmetrischer oder asymmetrischer Streckschwingung.\\[5pt]
Anwendung: Wasser absorbiert (infrarot) Licht mit den Eigenfrequenzen: \underline{$\omega_k$ sind exp. observable Größen}\\
$\rightarrow$ Schwingungsspektroskopie


% nummerierung der gleichungen nachtragen


\section{Das Zweikörperproblem}
\begin{itemize}
	\item beschränkt z.B. das Keplerproblem (Erde,Sonne) das H-atom, das 2-atomige Molekül
	\item analytisch lösbar
	\item Anwendung von Symmetrieüberlagerungen
\end{itemize}
2 Körper mit Masssen $ m_i $, Orten $ \vec{r}_i $, Impulsen $ \vec{p}_i \quad (i= 1,2)$ Wechselwirken durch ein Zentralpotential $ U(|\vec{r}_2-\vec{r}_1|) $. abgeschlossenes, konservatives System mit Energie
\begin{equation*}
H = T+U = \sum_{i=1}^{2} \frac{\vec{p}^{\,2}_i}{2 m_i} +  U(|\vec{r}_2-\vec{r}_1|) \tag{1}
\end{equation*}

\underline{Vorgehen:}
\begin{itemize}
	\item Seperation der Schwerpunktsbewegung (Impulserhaltung) \\
	$ \rightarrow $ Reduktion auf Einkörperproblem (3 statt 6 Freiheitsgrade)
	\item Drehimpulserhaltung $ \rightarrow $ 1D Problem
	\item Diskussion des Keplerproblems , $ k \sim 1/r $
\end{itemize}

\underline{(1.) Trafo in Schwerpunkts- und Relativbewegung}
Schwerpunktskoordinaten: $$\vec{R} = \frac{m_1\vec{r}_1 + m_2\vec{r}_2}{m_1+m_2}$$
Relativkoordinaten: $$\vec{r} = \vec{r}_1 - \vec{r}_2$$
$$T = \frac{\vec{p}_1^{\;2}}{2m_1} + \frac{\vec{p}^{\;2}_2}{2m_2} = \frac{m_1}{2} \dot{\vec{r}}_1^{\;2} + \frac{m_2}{2} \dot{\vec{r}}_2^{\;2}$$
\begin{equation*}
\buildrel (2) \over \rightarrow  \ub{\frac{m_1+m_2}{2}}_{M / 2} \dot{\vec{R}}^{\,2} + \frac{1}{2} \ub{\frac{m_1 m_2}{ m_1 + m_2}}_{\mu / 2} \dot{\vec{r}}^{\;2} = \frac{\vec{P}^{\,2}}{2 M} + \frac{\vec{p}^{\;2}}{2 \mu} \tag{3}
\end{equation*}
Gesamtmasse: $ M = m_1 + m_2 $\\
reduzierte Masse 
$$ \mu = \frac{m_1 m_2}{m_1 + m_2} \rightarrow \casess{m_1 \tx{ für } \frac{m_1}{m_2} \ll 1}{\tx{Sonne Erde}}{\frac{m}{2} \tx{ für } m_1 = m_2 = m }{\tx{ 2-atom Modell}}$$

Da Gesamtimpuls erhalten ist
$$\vec{P} = M \vec{R} = \vec{P}_0 = \tx{const.}$$
ist schwerpunktsbewegungs, die unabhängig von Relativbewegung ist
$$\vec{R}(t) = \vec{R}(0) + \frac{\vec{P}(0)}{\mu}t$$
$\rightarrow$ Seperation von SP- und Relativbewegung\\
oder Entkopplung $\rightarrow$ Einkörperproblem mit 3 (statt 6) Freiheitsgraden\\
Beobachte Relativbewegung
$$\mu \ddot{\vec{r}} = - \nabla k(|\vec{r}|)$$
\underline{(2.) Drehimpulserhaltung}
Das Zentralproblem, d.h.
$$\vec{L} = \mu \vec{r} \times \dot{\vec{r}} = \textrm{const.} \quad \tx{also auch Richtung konstant}$$
oEdA sei $\vec{L} = l \vec{e}_z$\\
Mit $\vec{L} \perp \vec{r}$, ist $\vec{r} \perp \vec{e}_z$ und damit $z(t) = \tx{const.} \buildrel \tx{oEdA} \over = 0$\\
d.h. Bewegung findet in x-y-Ebene statt. (nur noch 2 FG)\\
\underline{Polarkoordinaten:}
$$\vec{r} = \begin{pmatrix}
x\\y
\end{pmatrix} = \begin{pmatrix}
r \cos \varphi \\ r \sin \varphi
\end{pmatrix} \quad , \quad \dot{\vec{r}} = \begin{pmatrix}
\dot{x} \\ \dot{y}
\end{pmatrix} = \begin{pmatrix}
\dot{r} \cos \varphi - r \dot{\varphi} \sin \varphi \\ \dot{r} \sin \varphi + r \dot{\varphi} \cos \varphi
\end{pmatrix} $$
$$T = \frac{\mu}{2} \dot{\vec{r}}^2 = \frac{\mu}{2} ( \dot{r}^2 + r^2 \dot{\varphi}^2)$$
mit:
$$ l = \mu(x\dot{y} - y \dot{x}) = \mu r \cos \varphi ( \dot{r} \sin \varphi + r \dot{\varphi} \cos \varphi) - \mu r \sin \varphi ( \dot{r} \cos \varphi - r \dot{\varphi} \sin \varphi)$$
$$l = \mu r^2 \dot{\varphi} = \tx{const.}$$
ergibt sich für die Gesamtenergie
\begin{equation*}
\boxed{ E = \frac{\mu}{2} \dot{r}^2 + \frac{l^2}{2 \mu r^2} + U(r) = \frac{\mu}{2} \dot{r}^2 + U_{\tx{eff}}(r)} \tag{7}
\end{equation*}
effektives Ptential $ U_{\tx{eff}} (r) = U(r)  + \frac{l^2}{2 \mu r^2}$ hat neben dem ,,normalen`` Radialterm $U(r)$ noch den sogenannten Zentrifugalterm $ \frac{l^2}{2 \mu r^2} $ oder Zentrifugalbarriere\\\\
$\rightarrow$ 1D Syste, Bewegungsglechung ist lösbar\\\\
Lösung durch
$$\frac{dr}{dt} = \pm \sqrt{\frac{2}{\mu} (E-U_{\tx{eff}} (r))}$$
ist:
$$\intt{t_0}{t} dt' = t-t_0 = \pm \intt{r_0}{r} \frac{dr'}{\sqrt{\frac{2}{\mu} (E - U_{\tx{eff}} (r'))}}$$
was $r = r(t,E,l,r_0)$ ergibt die Bahnkurve $r(\varphi)$ und wir erhalten
$$\dot{r}(\varphi) = \frac{dr(\varphi)}{d\varphi} = \frac{dr}{d\varphi} \frac{d\varphi}{dt} = \frac{dr}{d\varphi} \dot{\varphi}$$
$$\frac{dr}{d\varphi} = \frac{\dot{r}}{\dot{\varphi}} =  \frac{\pm \sqrt{\frac{2}{\mu} (E - U_{\tx{eff}} (r))}}{\frac{l}{\mu r^2}} $$
$$ \Rightarrow \varphi - \varphi_0 = \pm \frac{l}{\sqrt{\mu}} \int_{r_0}^{r} \frac{dr'}{r'^{\,2} \sqrt{(E - U_{\tx{eff}} (r))}}$$
$ r_0, p_0, E, l $ sind dann Anfangsbedingungen

% blatt 3: 

% neue vorlesung

\subsection{Diskussion des Zweikörperproblems}
$$\varphi - \varphi_0 = \pm \frac{l}{\sqrt{2 \mu}} \int_{r_0}^{r} \frac{dr'}{r'^2 \sqrt{E - U_{\tx{eff}} (r')}}$$
$ r_0, \varphi_0, E, l \  \widehat{=} $ Anfangsbedingungen des 3D Problems\\
($ z(0) = \dot{z}(0)  = 0 $ Anfangsbedingungen des 3D Problems)
\begin{itemize}
	\item wegen $ \dot{\varphi} = \frac{l}{\mu r^2} $ kann $ \dot{\varphi} $ nicht das Vorzeichen wechseln $ \rightarrow $ Drehung immer in selbe Richtung d.h. für $  E = U_{\tx{eff}} $ ist $ r' = 0  \rightarrow $ Umkehrpunkte
	\item Bsp: Sei $ U_{\tx{eff}} (r) = dr^2 + \frac{l^2}{2 \mu r^2}$ \\
	$\alpha > 0 , l \neq 0, E = E_0$
	\item  r oszilliert zwischen $ r_{\tx{min}} $ und $ r_{\tx{max}} $ 
	\item Form der Bahnkurve zwischen je 2 Umkehrpunkten gleich\\
	$ \rightarrow $ Bahn ist durch Teilschleife festgelegt
	\item  Bahn ist nicht notwendigerweise geschlosssen $ \Delta \varphi $ zwischen zwei Umkehrpunkten 
	$$\Delta \varphi = 2 \int_{r_{\tx{min}}}^{r_{\tx{max}}} \frac{\frac{l}{r^2} dr}{\sqrt{ 2\mu (E-U_{\tx{eff}}(r))}}$$
	$ \rightarrow $ geschlossen, wenn nach n Umlaufen $ n \Delta \varphi = m \pi \quad (n,m \in \mathbb{N}) $
	\item Bsp: $ r^2  \tx{ oder } \frac{1}{r}$ Potential $ \Delta \varphi = \pi $\\
	\item  Für $ E = E_{\tx{kin}} $ ist $ r = r_0 = \tx{const.} \rightarrow$ Kreisbahn mit $ \varphi(t)  = \varphi_0 + \frac{l}{\mu r^2_0} t $ 
	\item Für $ l = 0 $ verschwindet Zentrifugalbarriere
	\item $ \dot{\varphi} = 0 , \vec{r} \parallel \dot{\vec{r}} \rightarrow $ Bewegung zentral
	\item  je nach Potential wird auch $ N \rightarrow 0 $ möglich (unelastisch, wegen endlicher Größe der Teilchen)
	\item \underline{Standartfall} $ l \neq 0 $\\
	$$\begin{array}{ll}
	U_{\tx{eff}} \buildrel r \rightarrow 0 \over \rightarrow \infty & \tx{(endliche größe)} \\
	U_{\tx{eff}} \buildrel r \rightarrow \infty \over \rightarrow 0 & \tx{WW nicht ,,gebundener Zustand``} \\
	U_{\tx{eff}} = \tx{min.} & \tx{,,gebundener Zustand``}
	\end{array}$$
	\begin{itemize}
		\item[$ E<0: $] gebundene Bewegung (auch bei H-Atomen, Elektronenstruktur, $ \tx{H}_2 $-Molekül, Kernbewegung)
		\item[$ E>0: $] Streuung d.h. Teilchen kommt aus dem Unendlichen fliegt bis zum Umkehrpunkt $ r_0 $ und verschwindet wieder Abstoßung bei $ r_0 $ wegen Zentrifugalbarriere $ \frac{l^2}{2 \mu r^2} $
	\end{itemize}
	
	bilder zu anziehendem-/abstoßendem Potential
\end{itemize}
\subsection{Keplerproblem}
d.h. $U(r) = - \frac{\alpha}{r} \qquad (\alpha > 0)$\\
$ \alpha = G m_1 m_2 \quad  $  Gravitations-Potential\\
$ \alpha = - \frac{q_1 q_2}{4 \pi \epsilon_0} $ Coulomb-Potential\\
$$U_{\tx{eff}} (r) = -\frac{\alpha}{r} + \frac{l^2}{2 \mu r^2} $$
$$\frac{d U_{\tx{eff}}}{dr} = \frac{\alpha}{r^2} - \frac{2 l^2}{2 \mu r^3} \buildrel ! \over = \qquad r_0 = \frac{l^2}{\alpha \mu}$$
\underline{Mit:}
$$q = \int dr \frac{\frac{l}{r^2}}{\sqrt{2 \mu E + 2 \mu \frac{\alpha}{r} - \frac{l^2}{r^2}}}$$
mit
$$ r = \frac{1}{s} \quad s = \frac{1}{r} \quad \frac{ds}{dr} = - \frac{1}{r^2}$$
$$ q = - \int ds (2\mu \frac{E}{l^2} + 2 \mu \alpha \frac{s}{l^2} - s^2) ^{-\frac{1}{2}}$$
und 
$$ \int dx (c+2bx-x^2)^{-\frac{1}{2}} = - \arccos \frac{x-b}{\sqrt{b^2 + c}}$$
ist
$$ \varphi(r) - \varphi_0 = \arccos \frac{\frac{1}{r} - \frac{\alpha \mu}{l^2}}{\sqrt{\mu^2 \frac{\alpha^2}{l^4} 2 p \frac{E}{l^2}}} = \arccos \frac{\frac{l^2}{\alpha \mu} (\frac{1}{r}) - 1}{\sqrt{1+2l^2 \frac{E}{\mu \alpha^2}}}$$
Mit $ p = \frac{l^2}{\alpha p} $ Abstand $ r_0 $\\
$ \epsilon = \sqrt{1 + 2l^2 \frac{E}{\mu \alpha^2}} \quad$ ,,Exzentrität``\\
$ \varphi_0 = 0 $\\
folgt $  \varphi = \arccos\frac{\frac{p}{r}-1}{\epsilon} $\\
oder:
$$\boxed{ r = \frac{p}{1 + \epsilon \cos \varphi}}$$
Polar-Gleichung für \underline{Kegelschnitte:}
\begin{itemize}
	\item[$ \epsilon > 1 $] $ E>0 $ Hyperbolen
	\item[$ \epsilon = 1 $] $ E = 0 $ Parabel
	\item[$ \epsilon < 1 $] $ E<0 $ Ellipse
\end{itemize}
Bsp:
$$\begin{array}{lll}
\tx{Merkur:} &  \epsilon = 0,206  & \tx{schwer zu beobachten)}\\
\tx{Erde:} & \epsilon = 0,017 \\
\tx{Mars:} & \epsilon = 0,043 & \tx{an ihm entdeckt)}\\
\end{array}$$\\
gebundene bewegung $ \rightarrow $ Keplersche Gesetze
\newpage
\begin{itemize}
	\item[1)] Planetenbewegung sind Ellipsen mit Sonne in einen Brennpunkt\\
	\begin{minipage}{.5\linewidth}
		$$
		\begin{tikzpicture}
		\draw[->] (-2.5,0) -- (3,0) node[anchor=north west] {$x$};
		\draw[->] (.8,0) -- (.8,2) node[anchor=south east] {$y$};
		\draw[thick] (0,0) ellipse (2cm and 1.5cm);
		\draw[thick] (0,0)--(0,-1.5)  (0,-.75) node[right] {$b$};
		\draw[thick] (0,0)--(-2,0);
		\node at (-.75,.3) {$a$};
		\draw[red,thick] (.8+.5,0) arc (0:120:.5);
		\draw[red] (0.8+.25,.25) to [out=60,in=180 ] (2,.8) node[right] {$\varphi$};
		\draw[->,thick,red] (.8,0) -- (100:1.5cm);
		\node[red] at (-.25,.6) {$r$};
		\node at (.8,-.25) {$\epsilon a$};
		\node[circle, fill=black, inner sep=1] at (.8,0) {};
		\end{tikzpicture} 
		$$
	\end{minipage}
	\begin{minipage}{.5\linewidth}
		$$\frac{(x+a\epsilon)^2}{a^2} + \frac{y^2}{b^2} = 1$$
	\end{minipage}
	\item[2)] Flächensatz
	\begin{minipage}{.7\linewidth}
		$$
		\begin{tikzpicture}
		\draw[thick] (1,0) ellipse (2.1cm and 1.5cm);
		\node[circle,fill=black,inner sep=2] (a) at (2,-1.3) {};
		\node at (3,-1.5) {Planet}; 
		\node at (1.2,-0) {Sonne};
		\node at (2,-.5) {\small{Fahrstrahl}};
		\node[circle, fill=black,inner sep=2] (b) at (0,0) {};
		\draw[dashed] (b) -- (a);
		\draw[thick] (b) -- (2.25,1.2);
		\draw[thick] (b) -- (1.7,1.4);
		\draw[thick] (b) -- (-0.85,0.7);
		\draw[thick] (b) -- (-0.85,-0.7);
		\node at (-1.5,0) {$A_1$};
		\node at (2.1,1.8) {$A_2$};
		\end{tikzpicture}
		$$
	\end{minipage}\\[5pt]
	Die vom Fahrstrahl pro Zeit $ dt $ überstrichene Fläche $ dA = r^2 \frac{d\varphi}{2} $ ist konstant\\
	Also $ A_1 = A_2 $ wurden vom Fahrstrahl in der \underline{gleichen} Zeit überfahren.
	\item[3)] Umlaufzeit  T und die große Halbachse a verhalten sich wie
	$$ T^2 = \tx{const.} a^3$$
\end{itemize}
\begin{itemize}
	\item Kepler (1571-1701): aufgrund von Beobachtungen der Planeten
	\item Newton leitete Gravitationsgesetz aus Keplerschen Gesetzen ab
	\item  KG3: $ \frac{r^3}{T^2} = \tx{const.} $\\
	Kreisbahn: $ \omega = \frac{2\pi}{T}  \rightarrow r^3 \omega^2 = \tx{const.} \rightarrow r \omega^2 = \tx{const.} \frac{1}{r^2}$
	$$\vec{r}(t) = r \begin{pmatrix}
	\cos \omega t \\
	\sin \omega t
	\end{pmatrix}$$
	$$\ddot{\vec{r}} (t) = - \omega^2 \vec{r}(t)$$
	$$\vec{F} = m \ddot{\vec{r}} (t) \sim \omega^2 \vec{r}(t) \qquad |F| \sim \frac{1}{r^2} \qquad \tx{Gravitationsgesetz}$$
\end{itemize}

\underline{Streuung:}
\begin{itemize}
	\item $ \epsilon = 1, E = 0 \rightarrow $ Parabel als Grenzfall
	\item $ \epsilon > 1, E > 0 \rightarrow $ \underline{Hyperbeln}
	$$\frac{x^2}{a^2} - \frac{y^2}{b^2} = 1 \leftrightarrow \frac{p}{r} = 1 + \epsilon \cos \varphi$$
\end{itemize}
\underline{attraktives Potential:} $ \alpha > 0 $\\
$ \rightarrow p = \frac{l^2}{\mu \alpha} > 0 $\\
Asymptoten: def Richtung durch $ \cos \varphi_\infty = -\frac{1}{\epsilon} $\\\\
\underline{repulsives Potential:} $ \alpha < 0 \rightarrow p < 0 $\\

% neue vorlesung 

\chapter{Lagrange-Formalismus}
nach J-L Lagrange (1736-1813)\\
1788,,Mechanique Analytique``\\
\underline{\underline{section 0. Motivation}}
\begin{itemize}
	\item Newtonsche Mechanik: alle Kräfte müssen bekannt sein
	$$ m\ddot{r} = \sum_i \vec{F}_i$$
	\item Aber oft nur Wirkung, nicht Kraft selbst bekannt\\
	bsp: Pendel (Abstand fest) \\
	oder Gas im geschlossenen Gefäß (Moleküle ,,gefangen``)
\end{itemize}
\section{Zwangsbedingungen}
$ \rightarrow $ schränken Bewegung des Systems auf einen Unterraum ein (z.B. Achterbahn, Bewegung in 2D)\\
\underline{Bsp.: Fadenpendel}\\
\begin{itemize}
	\item Gravitationskraft $ \vec{F}_a $ wirkt nach unten aber Faden der Länge $ l $ hat Masse $ m $ auf Kreisbahn (allg. Kugelschale)\\
	\begin{equation*}
	\rightarrow \tx{ ZB }y = 0 \qquad x^2 +z^2 = l^2 \tag{1}
	\end{equation*}
	\item übersetzen der ZB Newtonschen Bew. Gl.\\
	$ \rightarrow $ \underline{Zwangskraft $ \vec{Z} $}\\
	\begin{equation*}
	m \ddot{r} = \vec{F}_G \vec{Z} \tag{2}
	\end{equation*}
	\item $ \vec{Z} $ nicht von vornherein bekannt, nur Wirkung (1)
\end{itemize}
\underline{Lösungansätze}\\
\begin{itemize}
	\item $ \vec{Z} $ bestimmern: \underline{Lagrange-Gln 1. Art}
	\item  Zwangsbedingungen durch Wahl geeigneter Koordinaten eliminieren (Bsp: $ \varphi(r) \tx{anstatt} r(t) $)\\
	$ \rightarrow $ Bew.-Gl. für neue Koordinaten\\
	$ \rightarrow $ \underline{Lagrange-Gln 2.Art}
\end{itemize}
\underline{Klassifizierung von ZB}\\
\begin{itemize}
	\item System mit f Freiheitsgraden (N Massenpkt $ f=3N $)\\
	$ x_1,\dots,x_f \rightarrow $ Anzahl ZB $ R < f $
	\item Formulierung der ZB:\\
	\begin{equation*}
	g_\alpha (x_1,\dots,x_f,t) = 0 \quad \alpha = 1, \dots ,R \tag{3}
	\end{equation*}
	Bsp: $ f=3 $\\
	$ g_\alpha(x,y,z,t) = y = 0 $\\
	$ g_\alpha(x,y,z,t) = x^2 + z^2 - l^2 = 0 $
	\item jede ZB reduziert Anzahl der Freiheitsgrade\\
	1 Massenpunkt:\\
	keine ZB: Bew. 3D\\
	erste ZB: Bew. auf Fläche\\
	zweite ZB: Bew. auf Schnitt 2-er Flächen
	\begin{itemize}
		\item ZB der Art (3) heißen \underline{holonom}
		\item ZB die Zeit t explizit enthalten $ \rightarrow $ \underline{rheonom}
		\item ZB die Zeit t \underline{nicht} explizit enthalten $ \rightarrow $ \underline{skleronom}
	\end{itemize}
	\item Bsp für nicht-holonom:\\
	$ g_k (\vec{r}) = r-R < 0 $ (Inneres einer Kugel)\\
	$ g_k(\vec{r},\dot{\vec{r}})  = 0 $ (Geschw. benötigt)
\end{itemize}
\section{Lagrange-Gl 1.Art}
\begin{itemize}
	\item Eine holonome ZB: Beschränkung der Bewegung eines Teilchens auf eine Fläche\\
	$ g_1(\vec{r},t) = y = 0  \quad (xz-Ebene)$\\
	oder\\
	$ g_2(\vec{r},t) = x^2 + z^2 - l^2 = 0 $ (Kugelschale mit Radius l)
	\item \underline{keine weitere Einschränkung} der Bew. innerhalb dieser Fläche durch die ZB\\
	$ \rightarrow \vec{Z} $ kann keiner Komponente tangential zur Fläche haben\\
	$ \rightarrow \vec{Z} $ ist orthogonal zur Fläche, die durch $g$ gegeben ist\\
	$ \rightarrow $ wird erfüllt durch Ansatz\\
	\begin{equation*}
	\vec{Z}(\vec{r},t) = \lambda(t) \quad \nabla g(\vec{r},t) \tag{4}
	\end{equation*}
	mit zeitabhängigem Parameter $ \lambda(t) $\\
	Bsp: $ \nabla g_1(\vec{r},t) = \begin{pmatrix}
	0\\1\\0
	\end{pmatrix} \quad ; \quad \nabla g_2(\vec{r},t) = \begin{pmatrix}
	2x\\0\\2z
	\end{pmatrix} $
	\item Ansatz (4) zwar plausibel, kann aber nicht bewiesen werden\\
	$ \rightarrow $ (4) ost \underline{eigenständiges Axiom der Mechanik}
\end{itemize}
\underline{Bemerkung}\\
\begin{itemize}
	\item[1.] Skalare Fkt 2er Variablen $ f(x,y)  \rightarrow  $ ,,Gebirge`` in 3D\\
	$ \rightarrow $ partielle Abl. zeigen in Richtung des \underline{maximalen Anstiegs}\\
	Bsp: Kreiskegel\\
	$ z = f(x,y) = \sqrt{x^2+y^2} $\\
	$ \begin{pmatrix}
	f_x\\f_y
	\end{pmatrix} = \frac{1}{\sqrt{x^2+y^2}} \begin{pmatrix}
	x\\y
	\end{pmatrix} \sim \begin{pmatrix}
	x\\y
	\end{pmatrix} $\\
	,,Höhenlinien``
	\item[2.] Implizit durch holonome ZB $ F(x,y,z) = z - f(x,y) = 0 $\\
	Bsp: $ F = z^2 - x^2 - y^2 = 0 $\\
	$ \vec{\nabla} F = - \begin{pmatrix}
	2x\\2y\\-2z
	\end{pmatrix} \sim \begin{pmatrix}
	x\\y\\-z
	\end{pmatrix} \rightarrow \tx{ senkrecht auf Kugel}$
	\item[3.] Kraft $ \sim \vec{\nabla}g $ legt nahe, dass ZB $g$ als Art ,,Potential`` verstanden werden kann
\end{itemize}
\begin{itemize}
	\item Aus (4) und (2) $ \rightarrow $ \underline{Lagrange-Gln 1. Art}\\
	für 1 Teilchen unter einer ZB:\\
	\begin{equation*}
	m\ddot{\vec{r}} = \vec{F} + \lambda(t) \vec{\nabla} g(\vec{r},t) \tag{5}
	\end{equation*}
	$$g(\vec{r},t) = 0$$
	(4 Gln für 4 Unbekannte $x,y,z,\lambda$)
	\item Zwei holonome ZB Beschränken die Bewewgung auf Raumkurve
	\begin{itemize}
		\item[$ \rightarrow $] $ \vec{\nabla}g_1 $ und $ \vec{\nabla}g_2 $ unabhängig voneinander, senkrecht auf Kurve
		\item[$ \rightarrow $] $ \vec{Z} (\vec{r},t) = \lambda_1(t) \vec{\nabla}g_1(\vec{r},t) + \lambda_2(t) \vec{\nabla}g_2(\vec{r},t) $\\
		$ \rightarrow $ Senkrecht (häckchen)
	\end{itemize}
	\item Verallgemeinerung auf R ZB und N Teilchen ($ f=3N $) $ x \equiv(x_1,\dots,x_{3N}) $
	$$\boxed{m_n \ddot{x}_n = F_n + \sum_{\alpha = 1}^{R} \lambda_\alpha(t) \frac{\partial g_\alpha(x,t)}{\partial x_n} \qquad n = 1,\dots , 3N}$$
	$$\boxed{g_\alpha (x,t) = g_\alpha (x_1,\dots , x_{3N},t) = 0 \qquad \alpha = 1, \dots , R}$$
	Lagrange-Gln 1.Art für $3N$ Variablen und $R$ holonome ZB ($3N+R$ Gln für 3N+R unbekannte $ x_n, \lambda_\alpha $)
	\item[Bsp:] 2 Teilchen, 1ZB $ g(\vec{r}_1,\vec{r}_2,t) $\\
	\begin{itemize}
		\item $ g = g(\vec{r}_1,t) \rightarrow \vec{Z}_1 = \lambda(t) \vec{\nabla}_1 g(\vec{r}_1,t) $
		\item $ g(\vec{r}_1,\vec{r}_2,t) = g(|\vec{r}_1 - \vec{r}_2|) = |\vec{r}_1 - \vec{r}_2| -l = 0 $\\
		$ \rightarrow \vec{Z}_1 = - \vec{Z}_2 $
	\end{itemize}
\end{itemize}
\underline{Bemerkungen}\\
\begin{itemize}
	\item[1.] Zusätzliches Axiom (4) $ \rightarrow $ nichttriviale Verallgemeinerung der Newton-Axiome
	\item[2.] d`alembertsche Prinzip ( virtuelle Verrückungen)
	\item[3.] LG 1 insbesondere in technischer Mechanik. Physik haupsächlich LG 2. Art
	\item[4.] \underline{Erhaltung} von Impuls, Energie, Drehimpuls wenn Zwangsbedingungen entsprechende \underline{Symmetrie} erhalten
\end{itemize}

% neue vorlesung
\subsection*{Lagrange Formalismus }
Beispiele Zwangsbedingungen:
\begin{itemize}
	\item Körper auf Tisch $ \rightarrow z = 0 $
	\item Fadenpendel $ \rightarrow y = 0 \rightarrow g_1 = y = 0 \quad x^2 + z^2 = l^2 \rightarrow g_2 = x^2 + z^2 - l^2 = 0 $
\end{itemize}
allgemein: R ZB $ g_\alpha ( \vec{r}_1, \dots , \vec{r}_N,t) = 0 \quad \alpha = 1,\dots,R $ \\
\underline{ holonome ZB }\\
Zwangskräfte: $ m\ddot{\vec{r}} = \vec{F}_G + \vec{Z} $\\
\begin{itemize}
	\item bechränkt Bewegung auf eine Fläche
	\item innerhalb Fläche keine Einschränkung
	\item[$ \rightarrow $] $ \vec{Z} $ ist orthogonal zur Fläche (zu Beweisendes axiom)
\end{itemize}
\underline{Ansatz:} $ \vec{Z}(\vec{r},t) = \lambda(t) \grad g(\vec{r},t) $

hier fehlt was vielleicht was

\underline{Bsp:} Tisch: $  \quad g = z = 0 $\\

hier fehlt was noch mehr

% beide umrahmen andres?

\underline{Lagrange-Gl. 1. Art}
$ 3N+R \quad \tx{Gl.} $\\
\underline{Bsp:} \underline{Atwoodsche Fallmaschiene} (1784)\\
massenlose Rolle (Radius R), über die 2 Massen (reibungslos) verbunden sind, d.h. 2 Massen $ \rightarrow $ 6 Freiheitsgrade\\
\underline{ZB} \\
$ y_1 = 0 = y_2 $\\
$ x_1 = - R , x_2 = R $\\
$  g(z_1,z_2) = z_1 + z_2 + l = 0 \qquad l = L- \pi R  l = $ Seillänge\\
$ \rightarrow $ keine Dynamik in $ x_i $ und $ y_i $\\
$ \rightarrow $ es reicht, Bewegungs-Gl. für $ z_i $ zu betrachten\\
Zwangskräfte: $  z_i = \lambda \frac{\partial y}{\partial z_i}  = \lambda $ in Richtung $ z_i $\\
\begin{align*}
m_1 \ddot{z}_1 &= - m_1 g + \lambda \\
m_2 \ddot{z}_2 &= - m_2 g + \lambda \\
\frac{d^2}{dt^2} g(z_1,z_2) &= \ddot{z}_1 + \ddot{z}_2 = 0
\end{align*}
$$ \Rightarrow -g +\frac{\lambda}{m_1} - g + \frac{\lambda}{m_2} = 0 \Rightarrow \lambda 2 g \frac{m_1 m_2 }{m_1 + m_2}$$
$$ \ddot{z}_1 = -g + \frac{2 m_2}{m_1 + m_2} = g \frac{2 m_2 - (m_1 + m_2)}{m_1 + m_2} = g \frac{m_2 - m_1}{m_1 + m_2}$$
$$ z_1(t) = z_1(0) + \dot{z}_1(0) t + \frac{m_1 - m_2}{m_1 + m_2} \frac{g}{2} t^2 $$

\section{Lagrange-Gl. 2. Art}
Ausgangspunkt: Lag. Gl. 1. Art. (1)
\begin{equation*}
m_1 \ddot{x}_1 = F_n + \sum_\alpha \frac{\partial g_\alpha (x)}{\partial x_n} \tag{1}
\end{equation*}
$$ x = (x_1,\dots,x_3N) \quad n = 1,\dots,3N \quad g_\alpha (x) = 0 \quad \alpha = 1,\dots, R $$
Die Anzahl der Freiheitsgrade $ f = 3N - R $\\
\underline{Idee:} Führe ,,generalisierte`` (oder verallgemeinerte) Koordinaten ein $ q = (q_1, \dots, q_t) $

\begin{itemize}
	\item die die Lage aller Teilchen festlegen, d.h.\\
	\begin{equation*}
	x_n = x_n (q,t) = x_n (q_1, \dots , q_t, t) \qquad n = 1,\dots,3N \tag{2}
	\end{equation*}
	\item ZB $ q_\alpha $ sollen für beliebige $ q_i $  
	
	hier fehlt was
	
	%nachtragen
	
	\begin{equation*}
	g_\alpha (x_1(q,t) , \dots , x_{3N} (q,t), t ) = 0 \tag{3}
	\end{equation*}
	\item[$ \rightarrow $] ZB schränken Bewegung der $ q_i $ nicht ein
\end{itemize}
\underline{Bsp:} Ebenes Pendel mit variabler Länge $ l(t) $\\
$ x = l(t) \sin \varphi = x(\varphi,t)$\\
$ z = - l(t) \cos \varphi = z(\varphi,t)$\\
$ y = 0 = y(\varphi,t)$\\
d.h. $ \varphi $ ist verallg. Koord. , die die ZB\\
\begin{align*}
g(\vec{r},t) &= x^2 ( \varphi,t) + y^2 ( \varphi,t ) - l^2(t) \\
&= l^2 \cos^2 \varphi + l^2 \sin^2  - l^2 = 0
\end{align*}
für alle Werte $ \varphi $ erfüllt.\\
\underline{2. Bsp:} Teilchen im Kreiskegel $ \rightarrow $ general. Koord. $ r,\varphi $\\
Zylinderkoordinaten\\
$ x = r \cos \varphi $\\
$ y = r \sin \varphi $\\
$ z = r \cot \alpha $ alpha Azimutalwinkel phi Himmelsrichtungs Winkel\\
\underline{Eliminierung der Zwangskräfte}\\
Ausgangspunkt: Gl. (1)\\
Nach (3) hängen ZB $ g_\alpha $ nicht von $ q_i $ ab
\begin{equation*}
\frac{d g_\alpha}{ d q_k} = \sum_{n=1}^{3N} \frac{\partial g_\alpha}{\partial x_n} \frac{\partial x_n}{q_k} = 0 \quad k = 1, \dots , f \tag{4}
\end{equation*}
Gl. (1) multipliziert mit $ \partial x_n / \partial q_n $ ergibt:
$$\sum_{n} m_n \ddot{x}_n \frac{\partial x_n}{\partial q_k} = \sum_n \bigg[ F_n \frac{\partial x_n}{\partial q_k} + \sum_\alpha \lambda_\alpha \frac{\partial g_\alpha}{\partial x_n } \frac{\partial x_n}{\partial q_k} \bigg]$$
\begin{equation*}
\boxed{\sum_n \bigg[ m _n \ddot{x}_n - F_n \bigg] \frac{\partial x_n}{\partial q_q} = 0 } \tag{5}
\end{equation*}
$ n = 1,  \dots , 3N \quad k = 1, \dots , f$\\
\underline{Bem:}
\begin{itemize}
	\item (5) enthält keine Zwangskräfte, nur f Gl. aber die Transformation $ \frac{\partial x_n}{\partial q_k} $
	\item Durch Einführung der Lagrange-Funktion $ L = T-U $ kann (5) wesentlich vereinfacht werden
\end{itemize} 
Dazu betrachten wir:\\
\begin{equation*}
\dot{x}_n = \frac{d}{dt} x_n (q,t) = \sum \frac{\partial x_n}{\partial q_k} \dot{q}_k + \frac{\partial x_n}{\partial t} = \dot{x}_n (q,\dot{q},t) \tag{6}
\end{equation*}
mit general. Geschw. $ \dot{q}_i $ Es gilt:
\begin{equation*}
\frac{\partial \dot{x}_n (q,\dot{q},t)}{\partial \dot{q}_k} = \frac{\partial x_n (q,t)}{\partial q_k} \tag{7}
\end{equation*}
Mit 
\begin{equation*}
T = T(\dot{x}) = \sum_{n=1}^{3N} \frac{m_n}{2} \dot{x}_n^2 \tag{8}
\end{equation*}
ergibt sich: 
\begin{align*}
T = T(q,\dot{q},t) &= \frac{1}{2} \sum_n m_n \bigg[ \sum_k \frac{\partial x_n}{\partial q_k} \dot{q}_k + \frac{\partial
	x_n}{\partial t} \bigg] \bigg[ \sum_i \frac{\partial x_n}{\partial q_i} \dot{q}_i + \frac{\partial x_n}{\partial t}\bigg]\\
&= \frac{1}{2} \sum_{i,k} \sum_n m_n \frac{\partial x_n }{\partial q_i} \frac{\partial x_n}{\partial q_k} \dot{q}_i \dot{q}_k + \sum_k \ub{\sum_n m_n \frac{\partial x_n}{\partial q_k} \frac{\partial x_n}{\partial t}}_{= m_{ik}} \dot{q}_k \\
& \ \ \ + \frac{1}{2} \sum_n m_n \left(\frac{\partial x_n}{\partial t}\right)^2\\
T(q,\dot{q},t) & \equiv \sum_{i,k} m_{ik} (q,t) \dot{q}_1 \dot{q}_k + \sum_k  b_k (q,t) \dot{q}_k + c(q,t) \tag{9}
\end{align*}
\underline{Bem:} \begin{itemize}
	\item Die Größe T in (8) und (9) bezeichnet verschiedene Funktionen der Argumente, stellt aber die gleiche physikalische Größe dar.
	\item Da $ x_n $ linear in $ \dot{q}_k $ ist (7), ist die kin. Energie maximal quadratisch in den $ \dot{q}_k $
	\item Hängen die $ x_n $ nicht explizit von der Zeit ab, $ x_n = x_n (q) $ so wird Gl. (9)
	$$ T ( q, \dot{q}) = \sum_{i,k} m_{ik} (q) \dot{q}_i \dot{q}_k$$
\end{itemize}
Wir bilden die Ableitung
\begin{equation*}
\frac{\partial T(q,\dot{q},t)}{\partial q_k} \buildrel (8) \over = \sum_n m_n \dot{x}_n \frac{\partial \dot{x}_n}{\partial q_k} \tag{10}
\end{equation*}
\begin{equation*}
\frac{\partial T(q,\dot{q},t)}{\partial \dot{q}_k} = \sum_n m_n \dot{x}_n \frac{\partial \dot{x}_n}{\partial \dot{q}_n} \buildrel (7) \over = \sum_n m_n \dot{x}_n \frac{\partial  x_n}{\partial q_k} \tag{11}
\end{equation*}
\begin{equation*}
\frac{\partial T}{\partial q_k} = \sum n m_n \ddot{x}_n \frac{\partial x_n}{\partial q_k} + \sum_n m_n x_n \frac{\partial \dot{x}_n}{\partial q_k} \tag{12}
\end{equation*}
$$ \bigg[ \frac{d}{dt} \frac{\partial x_n}{\partial q_k} - \sum_l \frac{\partial^2 x_n}{\partial q_l \partial q_k } q_l + \frac{\partial^2 x_n}{\partial t \partial q_k} = \frac{\partial}{\partial q_k} \bigg( \sum_l \frac{\partial x_n}{\partial q_l} q_l + \frac{\partial x_n}{\partial t} \bigg) = \frac{\partial }{\partial q_k} \frac{d x_n}{dt} \bigg]$$
Beachte: Der erste und der zweite Term von (12) kommt auch in (5)\\
\underline{Def:} verallgemeinerung der Kräfte:\\

hier fehlt was\\

% nachtragen

Beachte konservative Kräfte
$$ F_n = - \frac{\partial U(x)}{\partial x_n} $$
Mit Trafo $ x_n = x_n (q,t)  $ ist $ U(q,t) = U ( x_q(q,t),\dots , x_{3N} (q,t)) $\\
Damit ergibt sich die verallgemeinerte Kraft:
\begin{equation*}
Q_k = \sum_n F_n \frac{\partial x_n}{\partial q_k} = - \sum_n \frac{\partial U(x)}{\partial x_n} \frac{\partial x_n}{\partial q_k} = - \frac{\partial U(q,t)}{\partial q_k} \tag{15}
\end{equation*}
Damit wird (14) (da $ \partial U / \partial \dot{q}_k = 0 $)
\begin{equation*}
\frac{d}{dt} \frac{\partial (T-U)}{\partial \dot{q}_k} = \frac{\partial (T-U)}{\partial q_k} \tag{16}
\end{equation*}
Def der sogenannten Lagrange Funktion:

% nachschauen hier fehlt was skript  gl 17 und 18 

% neue vorlesung

\underline{Lagrange-Gleichungen}\\
N Teilchen, $ x = (x_1,\dots,x_{3N}) = \{x_n\}\quad k = 1, \dots , 3N $\\
R Zwangsbedingungen $ g_\alpha (x,t) = 0 \quad \alpha = 1, \dots , R $\\
Lagrange-Gl- 1. Art: 
$$m_n \ddot{x}_n = F_n + \sum_\alpha \lambda_\alpha \frac{\partial g_\perp}{\partial x_n}$$
\underline{Verallg. Koord.:} $ 1 = (q_1,\dots,q_f) = \{ q_k\} \quad k = 1,\dots,f \quad f = 3N - R $\\
mit $ x_n = x_n (q,t) \quad , \quad g_\alpha = g_\alpha(q,t) = 0 $
$$ \rightarrow \frac{\partial g_\alpha}{\partial g_k} = 0 $$
\underline{Eliminierung der Zwangskräfte}
$$\sum_n m_n \ddot{x}_n \frac{\partial x_n}{\partial q_k} = \ub{\sum_n F_n \frac{\partial x_n}{\partial q_k}}_{Q_k} \qquad k = 1, \dots , f \tx{ Gl. ohne Zwangskräfte} $$
$$ \frac{d}{dt} \frac{\partial T}{\partial \dot{q}_k} - \frac{\partial T}{\partial q_k} = Q_k = - \frac{\partial U(q,t)}{\partial q_k}$$
\underline{Lagrange-Fkt.:} $ L(q,\dot{q},t) = T( q, \dot{q},t ) - U(q,t) $
$$ \boxed{\frac{d}{dt} \frac{\partial L}{\partial \dot{q}_k} = \frac{\partial L}{\partial q_k}} \qquad \tx{ Lagrange-Gl (2. Art)}$$
\underline{Diskussion:}\\

\begin{itemize}
	\item[1.)] Gl: (18) stellt ein System von $ f = 3N - R $ DGLn 2. rd. dar d.h. eine Vereinfachung der Lagrange-Gl. 1. Art, aber ohne explizit gegebene Zwangskräfte
	\item[2.)] Da es i.a. unterschiedliche verallg. Koord. $ q_k $ für gegebene Probleme gibt, ist $ L $ nicht eindeutig. Weiterhin sind Zusatztherme zu $ L $ möglich, die die Bew. Gl. nicht ändern (siehe Übungen).\\
	Daher ist $ L $ eine theoretische Größe, im Vergleich zu direkt messbaren Größen wie $ T $ und $ U $.\\
	Die allg. Form der Lagrange-Gl. bleibt aber gleich ,,Forminvarianz``\\
	(Nicht so bei den Nowton-Gl. z.B. in Polar Koord. gilt $ m\ddot{r} = -\frac{\partial U}{\partial r} $ und $ m r^2 \ddot{\varphi} \neq - \frac{\partial U}{\partial r} $)
	\item[3.)] $ L $ ist eine \underline{Skalare} Größe $ \rightarrow $ Leichter aufzustellen als vektorielle Kräfte im $ \mathbb{R}^{3N} $. Zudem ist $ L $ eine einfache Funktion der Variablen.
	\item[4.)] Liegen keine Zwangskräfte vor, so sind die $ q_k $ einfach die kartesischen Koordinaten $ x_n $ und mit
	$$L(x,\dot{x}) = \frac{1}{2} \sum_n^{3N} m_n \dot{x}^2_n - U(x)$$
	$$\frac{d}{dt} \frac{\partial L}{\partial \dot{x}_n} = m_n \ddot{x}_n$$
	$$\frac{\partial L}{\partial x_n} = - \frac{\partial U}{\partial x_n}$$
	$$ \rightarrow m_n \ddot{x}_n = - \frac{\partial U}{\partial x_n} \quad \tx{Newton Bwe- Gl.}$$
	\item[5.)] Bei geschwindigkeitsabhängigem Potential muss die Def. der allg. Kraft erweitert werden, $ U = U(q,\dot{q},t) $
	$$ Q_k = - \frac{\partial U}{\partial q_k} + \frac{d}{dt} \frac{\partial U}{\partial \dot{q}_k}$$
	$ \rightarrow $ führt wieder auf Lagrange-Gl.\\
	wichtigstes Bsp. ist Lorenz Kraft mit Potential
	$$U(\vec{r}, \dot{\vec{r}}, t) = \ub{e \Phi (\vec{r},t)}_{\tx{el. Pol}} - \ub{\frac{e}{c} \vec{A} (\vec{r},t) \cdot \dot{\vec{r}}}_{\tx{Vektorpotential}}$$
	\item[6.)] Wesentlich ist die Wahl der verallg. Koord. $ q_k $, die das betrachtete ,,System`` definieren.\\
	Restliche Freiheitsgrade werden vernachlässigt oder über Reibungstherme oder externe zeitabhängige Funktionen berücksichtigt.
\end{itemize}
\underline{Bsp 1:} Schiefe Ebene mit Steigung $ \alpha $. Achse $ s $ liegt in der schiefen Eben ist also die verallg. Koord
$$ x(t) = s(t) \cos \alpha \qquad z(t) = s(t) \sin \alpha$$
Mit $ T = \frac{m}{2} (\dot{x}^2 + \dot{z}^2) $ und $ U = mgz $\\
ist $$ L(s,\dot{s}) = T-U = \frac{m}{2} \dot{s}^2 - mg \sin \alpha s $$
mit Lagrange-Gl. 
$$\frac{d}{dt} \frac{\partial L}{\partial \dot{s}} = m \ddot{s} = \frac{\partial L}{\partial s} = - mg \sin \alpha$$
mit Lösung:
$$ s(t) = -\frac{g}{2} \sin \alpha t^2 + v_0 t + s_0 $$
\underline{Bsp 2:} Kreiskegel [bild]\\
\underline{kart. Koord:} $ L = T+U = \frac{m}{2} (\dot{x}^2 + \dot{y}^2 + \dot{z}^2) - mgz $\\
\underline{Zylinder Koord:}\\
$ x = r \cos \varphi \quad \dot{x} = \dot{r} \cos \varphi - r \sin \varphi \dot{\varphi} $ \\
$ y = r \sin \varphi \quad \dot{y} = \dot{r} \sin \varphi + r \cos \varphi \dot{\varphi} $\\
$z = r \cot \alpha \quad \dot{z} = \dot{r} \cot \alpha $\\
$ \rightarrow $ verallg. Koord. z.B. $ r,\varphi $\\
$$L = \frac{m}{2} \bigg[r^2 \varphi^2 + \dot{r}^2 ( 1 + \cot ^2 \alpha)\bigg] - mg r \cot \alpha = L(r,\dot{r}, \varphi , \dot{\varphi})$$
\underline{Lagrange-Gl.:}
$$\frac{d}{dt} \frac{\partial L}{\partial \dot{r}} - \frac{\partial L}{\partial r} = \boxed{\frac{m}{2} (1 + \cot^2 \alpha) 2 \ddot{x} - m r \dot{\varphi}^2 + m g \cot \alpha = 0}$$
$$ \frac{d}{dt} \frac{\partial L}{\partial \dot{\varphi}} - \frac{\partial L}{\partial \varphi} = \frac{m}{2} \frac{d}{dt} \bigg[2 r^2 \dot{\varphi}\bigg] = \boxed{m \bigg[2 r \dot{r} \dot{\varphi} + r^2 \ddot{\varphi}\bigg] = 0}$$
\underline{Reibungskräfte}\\
z.B. Stokessche Reibungskraft
$$F_n^R = - \gamma _n \dot{x}_n$$
Reibungskräften kann kein Potential zugeordnet werden\\
Daher zurück zu Gl.(15)
$$ \sum_n m_n \ddot{x}_n \frac{\partial x_n}{\partial q_k} = \sum_n F_n \frac{\partial x_n}{\partial q_k} = Q_k$$
mit verallg. Kräften 
$$ Q_k^R = sum_n F_n^R \frac{\partial x_n}{\partial q_k} $$
\underline{Rayleisghsche Dissipationsfunktion}
$$D(\dot{x}) = \sum_n \frac{\gamma _n}{2} \dot{x}^2_n$$
$$\rightarrow D(q, \dot{q}, t) = \sum_n \frac{\gamma_n}{2} x_n(q,\dot{q},t) $$
$$Q_k^R = - \sum_n \frac{\partial D}{\partial \dot{x}_n} \frac{\partial x_n}{\partial q_k} \buildrel (7) \over =  - \sum_n \frac{\partial D}{\partial \dot{x}_n} \frac{\partial \dot{x}_n}{\partial \dot{q}_k} = \frac{\partial D(q,\dot{q},t)}{\partial \dot{q}_k}$$
$$ (7) : \frac{\partial \dot{x}_n}{\partial \dot{q}_k} = \frac{\partial x_n}{\partial q_k}$$
$ \rightarrow  $ Lagrange-Gl. mit Reibung

\subsection{Lagrange Formalismus}
verallgemeinerte Koordinaten $q = ( q_1, \dots , q_f) \quad f = 3N-R$\\ 
n beschw. $q = (\dot q_1, \dots , \dot q_f)$
$$L ( q, \dot q, t) = T- U$$
$$\frac{d}{dt} \frac{\partial L}{\partial \dot q_k} \quad K = 1, \dots, F$$
\subsection{Energieerhaltung}
Wir betrachten
$$\frac{d}{dt} \sum_k \frac{\partial L}{\partial \dot q_k} \dot q_k  = \sum_k \dot q_k \underbrace{\frac{d}{dt} \frac{\partial L}{\partial \dot q_k}}_{\partial L / \partial q_k} + \sum_k \frac{\partial L}{\partial \dot q_k} \ddot q_k$$
$$\frac{dL}{dt}= \sum_k \frac{\partial L}{\partial q_k} \dot q_k + \sum_k \frac{\partial L}{\partial \dot q_k} \ddot q_k + \frac{\partial L}{\partial t}$$
$$\frac{d}{dt} \bigg( \sum_k \frac{\partial L}{\partial \dot q_k} \dot q_k - L \bigg) = -\frac{\partial L}{\partial t}$$
$\rightarrow$ Erhaltungssatz\\
wenn $\frac{\partial L}{\partial t} = 0$ ist $\sum_k \frac{\partial L}{\partial \dot q_k} \dot q_k - L$ erhalten.\\
$\rightarrow k = U(q) \neq U(q,t)$\\
Hängen zB nicht explizit von der Zeit ab, $x_n = x-n(q) \neq x_n (q,t)$ sowie das Potential $U$ nicht explizit von den geschwind. $U = U(q)$\\
ist $$T = \sum_{k,l} m_{kl}(q) \dot q_k \dot q_l = T (q, \dot q)$$
und $$\sum_{k} \frac{\partial L}{\partial \dot q_k} = \sum_{k} \frac{\partial T}{\partial \dot q_k} \dot q_k = 2T(q,\dot q)$$
Damit folgt mit: $\partial L/ \partial t = 0$
$$\sum_k \frac{\partial L }{\partial \dot q_k}\dot q_k = - L = T+U = E = \tx{ conts.} \quad \tx{Energieerhaltung}$$
\section{Symmetrie und Erhaltungsgrößen}
Lagrange-Formelismus erleichtert das Finden von Erhaltungsgrößen
\begin{description}
	\item [Def:] 1) Zyklische Koordinate $q_k$
	$$\frac{\partial L}{\partial q_k} = 0$$
	$\ \,$2) Verallgemeinerte (oder "kanonisch konjungierter") Impuls
	$$p_k = \frac{\partial L}{\partial \dot q_k}$$
\end{description}
Mit $\frac{d}{dt}\frac{\partial L}{\partial \dot q_k} = \frac{\partial L}{\partial q_k}$\\\
Wenn $q_k$ Zyklisch, ist $p_k$ erhalten
$$\boxed{\frac{\partial L}{\partial q_k} = 0 \rightarrow p_k = \tx{const.}}$$
Bsp: freies Teilchen
$$L = T = \frac{m}{2} \vec{\dot r}^2 \tx{  hängt von }\vec{r} \tx{ ab} \rightarrow \vec{r} \tx{ ist zyklisch Koord.}$$
$\rightarrow$ damit $ \vec{p} = m \vec{\dot r}$ erhalten\\
Bem: Bezeichnung verallg. Impuls p
$$L = \frac{m}{2} \dot r^2 - U(r) \rightarrow p = \frac{\partial L}{\partial \dot r} = m \dot r \tx{ kinematischer Impuls}$$
Die Äquivalenz zwischen verallg. Impuls (2) und kinematischen Impuls $m \dot r$ gilt für Geschw.-unabhängige Potentiale.\\
(gegen)- Bsp: elektormagnetisches Potential mit Vektorpotential $\vec{A}$
\begin{align*}
\vec{p} &= m \vec{\dot r} + q \vec{A} \quad \tx{kan. Impuls}\\
m \vec{\dot r} &= \vec{p} - q \vec{A} \quad \tx{kinem. Impuls}
\end{align*}
Gl. (3) beschreibt Zusammenhang zwischen 
\begin{itemize}
	\item \underline{Symmetrie oder Invarianz}\\
	zB System verändert sich nicht bei Translation in $q_k$ d.g. $L$ kann nicht von $q_k$ abhängen $\frac{\partial L}{\partial q_k} = 0$ und
	\item \underline{Erhaltung}\\
	zugehöriger verallg. Impuls $p_k$ ist erhaltern: $\frac{dp_k}{dt} = 0$
\end{itemize}
\underline{allg. Idee}:\\
\indent geg. Erhaltungsgröße $f,$ mit $\frac{d}{dt} f (q, \dot q,t ) = 0$ bildet eine "Konstante der Bewegung" oder "ertes Integral"\\
$\rightarrow$ erleichtert Lösung der Lagrange-Gl.\\
$\rightarrow$ Wähle verallg. Koord. so, da möglichst viele Erhaltungsgrößen aufgestellt werden, da 
\begin{itemize}
	\item jede Erhaltungsgröße (zB $E, \vec{p},\vec{L})$ verringert die Anzahl der Integrationen der Bew.Gl unter 1.
	\item Erhaltungsgößen sind nützlich bei Interpretation zB Drehimpulserhaltung $\rightarrow$ 2. Keplersche Gesetz
	\item geg. genügende Anzahl von Erhalt. größen\\
	$\rightarrow$ system kann nicht chaotisch sein
\end{itemize}
Bsp: kreiskegel: verallg. Koord. $r, \varphi$ %bild
$$L = \frac{m}{2} [r^2 \dot \varphi^2 + \dot r^2 ( 1+ \cot^2 \alpha)] - mg \cot \alpha\ r$$
Bew.Gl: 
\begin{equation*}
2 \dot r \dot \varphi - r \ddot \varphi = 0 \tag{1}
\end{equation*}
\begin{equation*}
(1+ \cot^2 \alpha) \ddot r - r \dot \varphi^2 - g \cot \alpha = 0 \tag{2}
\end{equation*}
L hängt nicht von $\varphi$ ab $\rightarrow \varphi$ ist zykl. Koord.\\
\begin{equation*}
\rightarrow p_\varphi = \frac{\partial L }{\partial \varphi} = m r^2 \dot \varphi = \const \tag{3}
\end{equation*}
\begin{itemize}
	\item Erhaltung der $z$-Komp. der Drehimpulses
	\item Energieerhaltung
\end{itemize} 
\begin{equation*}
E = T + U = \tx{const.} \tag{4}
\end{equation*}
(3) und (4) sind DGL 1. Ord. während Gl. (1), (2) DGL 2. Ord. sind.\\
$\rightarrow$ leite (3), (4) aus (1), (2) her:
\begin{enumerate}
	\item Mutiplikation: (1), $r$: 	$2r \dot r \dot \varphi + r^2 \ddot \varphi = 0$\\
	$\rightarrow \frac{d}{dt}(r^2 \dot \varphi) = 0 \rightarrow r^2 \dot \varphi=$ const.
	\item Multiplikation von (2) mit $\dot r$: 	 $(1 + \cot^2 \alpha) \dot r \ddot r - \dot r \frac{p_\varphi^2}{m^2 r^3} + g \cot \alpha\ \dot r = 0$ und Gl. (3) 
	$$\rightarrow \frac{d}{dt} [(1 + \cot^2 \alpha) \frac{\dot r^2}{2} + \frac{1}{m^2 r^2} p_\varphi^2 + g \cot \alpha\ r ] = 0$$
	$$ = \frac{d}{dt} \bigg(\frac{E}{m}\bigg) = 0$$
	Intergration von (4) mit
	$$\bigg(\frac{dr}{dt}\bigg)^2 = [E - \frac{m}{2} \frac{p_\varphi^2}{m^2 r^2} - mg \cot \alpha] \frac{2}{m}\frac{1}{(1+\cot^2 \alpha}$$
	$$1 + \cot^2 \alpha = \frac{a\sin^2 \alpha + \cos^2 \alpha}{\sin^2 \alpha}= 1/\sin^2 \alpha$$
	Separation der Variablen:
	\begin{equation*}
	t = \pm \int \frac{dr}{\frac{2}{m} [E- \frac{p_\varphi^2}{2mr^2}-mg \cot \alpha\ r]\sin \alpha}	\tag*{(5)}
	\end{equation*}
\end{enumerate}
und damit $r(t)$\\
Damit kann Gl. (3) integriert werden
\begin{equation*}
\varphi(t<9) = \frac{p_\varphi}{m} \int \frac{dt}{r^2(t)} \tag{6}
\end{equation*}
\underline{Vgl mit Zentralproblem:} effektives Potential
$$U_{\tx{eff}}(r) = \frac{mg \cot \alpha\ r}{\gamma} + \frac{p_\varphi^2}{2mr^2}$$
minimum bei: $\frac{\partial U_{\tx{eff}}}{dr} = \gamma \frac{p_\varphi^2}{2mr^3} = 0$

\subsection{Noether-Theorem}
\setcounter{equation}{0}
\underline{Emmi Noether} (1882 - 1935), deutsche Mathematikerin\\
Verallg. des Zusammenhangs zwischen Invarianz und Erhaltung\\
Geg: $ L = L(q, \dot q, t)$ mit Lösung $q(t) = (q_1(t), \dots, q_f(t))$\\

\frbox{\tx{,,Noether Theorem``}}{Ist $L$ invariant unter der Trafo
	\begin{equation}\label{nt:1}
	q_i (t) \rightarrow q_i (t,\alpha) \tag*{(1)}
	\end{equation}
	also
	\begin{equation}\label{nt:2}
	L (q(t, \alpha), \dot q(t,\alpha),t) = L(q(t),\dot q (t),t) \tag*{(2)}
	\end{equation}
	so ist die größe 
	\begin{equation}\label{nt:3}
	\summ{i=1}{f}\ \frac{\partial L}{\partial\dot q_i}\ \frac{\partial q_i}{\partial \alpha}\bigg|_{\alpha=0} \tag*{(3)}
	\end{equation}
	erhalten.
}
\noindent Bsp. für Trafos sind:\\
\indent - Transformation $q_i = q_i + \alpha$\\
\indent - Rotation um geg. Achse mit Winkel $\alpha$\\
\paragraph{Beweis:}
$$\frac{\partial}{\partial \alpha}L(q(t,\alpha), \dot q(t, \alpha),t)\bigg|_{\alpha=0} \buildrel \ref{nt:2} \over = \frac{\partial}{\partial \alpha}L(q(t)\dot q(t),t)\bigg|_{\alpha=0} = 0$$
$$0 = \summ{i=1}{f}\bigg(\frac{\partial L}{\partial q_i}\ \frac{\partial q_i}{\partial \alpha} + \frac{\partial L}{\partial \dot q_i}\ \frac{\partial \dot q_i}{\partial \alpha}\bigg) \bigg|_{\alpha=0}$$
$$= \sum_i \underbrace{\bigg[\frac{\partial L}{\partial q_i} - \frac{d}{dt} \ \frac{\partial L}{\partial \dot q_i} \bigg]}_{= 0, \tx{ wenn $q_i$ Lösung}} \frac{\partial q_i}{\partial \alpha}\bigg|_{\alpha=0} + \underbrace{\frac{d}{dt} \sum_i \frac{\partial L}{\partial \dot q_i}\ \frac{\partial q_i}{\partial \alpha}\bigg|_{\alpha=0}}_{= 0} = 0$$
\paragraph{Bem:} Beweis gilt für alle $\alpha$, also auch für $\alpha = 0$. Mit $\alpha = 0$ werden oft Ausdrücke einfacher.

% 14.06.18

\paragraph{Invarianz}
\begin{trivlist}
	\item Zyklische Variable $q_a$ mit $\frac{\partial L}{\partial q_a}= 0 $ $\leftrightarrow$ verallg. Impuls $p_a = \frac{\partial L}{\partial \dot q_a} = $ const.
	\item zB: $L = \frac{m}{l}\dot x^2 \rightarrow p = m \dot x = $ const.
	\item Allg: Ist $L(q(t), \dot q(t), t)$ invariant bzgl. $q_i (t) \rightarrow q_i (t, \alpha)$ ist die Größe $$\summ{i = 1}{f} \frac{\partial L}{\partial \dot q_i} \frac{\partial q_i}{\partial \dot \alpha} \bigg|_{\alpha = 0} \tx{ erhalten}$$
\end{trivlist}
\paragraph{Verallg.:} Falls L nicht invariant bzgl. Trafo, aber gilt $$\frac{\partial}{\partial \alpha} L(q(t, \alpha), \dot q (t, \alpha), t) \bigg|_{\alpha = 0} = \underbrace{\frac{\partial}{\partial \alpha} L (q(t), \dot q (t), t)}_{= 0} + \frac{d}{dt}F(q(t), \dot q (t), t)$$
mit beliebiger Funktion $F$, so folgt
\begin{align*}
0 &= \frac{\partial L}{\partial \alpha} + \frac{dF}{dt} = \sum_i \frac{\partial L}{\partial q_i}\ \frac{\partial q_i}{\partial \alpha}\bigg|_{\alpha=0} + \frac{\partial L}{\partial \dot q_i}\ \frac{\partial \dot q_i}{\partial \alpha}\bigg|_{\alpha=0} + \frac{dF}{dt}\\
&= \sum_i \bigg[ \frac{\partial L}{\partial q_i} - \frac{d}{dt} \frac{\partial L}{\partial \dot q_i} \bigg] \frac{\partial q_i}{\partial \alpha} \bigg|_{\alpha = 0} +  \underbrace{\frac{d}{dt} \bigg[ \sum_i \frac{\partial L}{\partial \dot q_i}\ \frac{\partial q_i}{\partial \alpha} + F \bigg]}_{= 0} = 0
\end{align*}
und damit 
\begin{equation*}
\sum_i \frac{\partial L}{\partial \dot q_i}\ \frac{\partial q_i}{\partial \alpha} + F(q, \dot q, t) \tag{$*$}
\end{equation*}
ist erhalten.
\paragraph{Translation:} Lagrange Funktion $L(\vec{r_1}, \dots , \vec{r}_N, \vec{\dot r}_1, \dots, \vec{\dot r}_N, t)$ sei invariant unter Trafo
$$\vec{r}_i (t) \rightarrow \vec{r}_i (t,\alpha) = \vec{r}_i (t) + \alpha\vec{e}$$
wobei $\vec{e}$ ein beliebiger (aber konst.) Einheitsvektor ist.\\
Gilt zB wenn Pot. $U$ nur von Differenzvektoren abhängt
$$\vec{r}_i (t, \alpha) - \vec{r}_j (t, \alpha) = \vec{r}_i (t) - \vec{r}_j (t)$$
Damit ist $$\frac{\partial \vec{r}_i (t,\alpha)}{\partial \alpha} \bigg|_{\alpha = 0} = \vec{e}$$
und $$\sum_i \frac{\partial L}{\partial \vec{\dot r}_i} \vec{e} = \sum_i m_i \vec{\dot r}_i \cdot \vec{e} = \vec{p} \cdot \vec{e} \quad \tx{ ist die erhaltene Größe}$$
Da $\vec{e}$ beliebig ist, ist Gesamt Impuls erhalten. Bei speziellen Vektor $\vec{e}_0$ ist nur $\vec{p} \cdot \vec{e}_0$ Komponente erhalten d.h.\\
\rbox{
	Invarianz bzgl. Translation um $\vec{e} \leftrightarrow$ Impuls $p_e$ ist erhalten.\\
	Symmetrie: ,,Homogenität der Raumes''
}
anschaulich: keine Hindernisse im Raum

\paragraph{Rotationsinvarianz}
Sei $$L(\vec{r}, \vec{\dot r}) = \frac{m}{2}(\dot x^2 + \dot y^2 + \dot z^2) - U(x^2+y^2+z^2)$$
in Zylinderkoord: $$L(r, \varphi, z, \dot r, \dot \varphi, \dot z) = \frac{m}{2}(\dot r^2 + r^2 \dot \varphi^2 + \dot z^2) - U(r^2, z)$$
ist bzgl. der Trafo $\varphi \rightarrow \varphi + \alpha$ invariant\\
Folglich ist $$\frac{\partial L}{\dot \varphi}\ \frac{\partial \varphi}{\partial t} \bigg|_{\alpha = 0} = mr^2\dot \varphi \frac{\partial (\varphi+d)}{\partial \alpha} = mr^2 \dot \varphi = L_z$$ erhalten
\rbox{Invarianz bzgl. Drehung um $\vec{e} \leftrightarrow$ Drehimpuls $L_e$ ist erhalten.\\
	Symmetrie: Isotropie des Raumes}
d.h. keine Richtung ausgezeichnet.

\paragraph{Translation in der Zeit}
$$t \rightarrow t + \alpha. \tx{ d.h. } q_i(t,\alpha) = q_i (t+\alpha)$$
Damit $$\frac{\partial q_i(t,\alpha)}{\partial \alpha} \bigg|_{\alpha = 0} = \frac{\partial q_i}{\partial t}\frac{\partial \alpha}{\partial t}\bigg|_{\alpha = 0} = q_i$$
$$\frac{\partial \dot q_i (t,\alpha)}{\partial \alpha}\bigg|_{\alpha = 0} = \ddot q_i$$
Betrachte 
\begin{align*}
\frac{\partial}{\partial \alpha} L (q(t +\alpha), \dot q (t+ \alpha),t)\bigg|_{\alpha = 0} &= \sum_i \bigg( \frac{\partial L}{\partial q_i}\ \frac{\partial q_i(t+\alpha)}{\partial \alpha} + \frac{\partial L}{\partial \dot q_i}\ \frac{\partial \dot q_i(t+\alpha}{\partial \alpha}\bigg)_{\alpha = 0}\\ 
&= \sum_i \frac{\partial L}{\partial q_i} \dot q_i +\frac{\partial L}{\partial\dot q_i} \ddot q_i
\end{align*}
und $$\frac{dL(q(t),\dot q (t), t)}{dt} = \frac{\partial L}{\partial t} + \sum_i \bigg( \frac{\partial L}{\partial q_i} \dot q_i + \frac{\partial L}{\partial \dot q_i} \ddot q_i \bigg)$$
$$\rightarrow \frac{\partial L }{\partial \alpha}\bigg|_{\alpha = 0} = \frac{d L}{dt} \tx{ falls $L$ nicht explizit von Ziel abhängt d.h. } \frac{\partial L}{\partial t} = 0$$ 
mit $(*)$ und $F(q, \dot q, t) = -L (q, \dot q,t),$ somit ist erhalten
$$\sum_i \frac{\partial L}{\partial \dot q_i}\dot q_i - L (q, \dot q) \tx{ Formel hergeleitet bei Energieerhaltung}$$
$$= T + U = E = \tx{ const. d.h. Energieerhaltung}$$

\rbox{Invarianz unter Zeittranslation $\leftrightarrow$ Energieerhaltung\\
	Symmetrie: Homogenität der Zeit}
anschaulich: Experiment verläuft heute genauso wie morgen
\paragraph{Bem:}
\begin{itemize}
	\item Die durch das Noether Theorem beschriebene Beziehung
	$$\tx{Invarianz/Symmetrie} \leftrightarrow \tx{Erhaltung}$$
	ist fundamental und allg. gültig. Gilt also auch in QM und relativistischer Mechanik.
	\item zB: liefert die Erhaltung von Ladung, Isospin,...\\
	Konstruktionsbedingungen für entsprechende Theorien.
\end{itemize}

\section{Hamiltonsches Prinzip}
\setcounter{equation}{0}
\subsection{Funktionale und Variationsrechnung}
\begin{itemize}
	\item Funktion $x \rightarrow y = f(x)$ ordnet jeder Zahl $x$ eine Zahl $y$ zu.\\
	Extrema durch Nullstellen der Ableitung $\frac{df}{dx}$
	\item Funktional $y = f(x) \rightarrow J[y]$ ordnet einer Funktion $f(x)$ eine Zahl $J$ zu.
\end{itemize}
\paragraph{Bsp1: Kürzeste Wegstrecke} %bild
Wegstrecke: 
\begin{align*}
J = J[y] &= \int_1^2 ds\\
&=\int_{\lambda_1}^{\lambda_2} dx\ \sqrt{1+ y'(x)^2}
\end{align*}
kürzeste Wegstrecke: $J[y] = \tx{ min.}$\\
(Ergebnis aus Kurvenintegral)
\paragraph{Bsp2: Brachistochrone} Bernulli (1696)\\ %bild
Masse $m$ gleitet reibungslos wegen Schwerkraft auf Kurve y(x). Für welcher y(x) ist die Zeit $T$ minimal?\\

$$\left.\begin{array}{l}
\tx{Mit } v = \frac{ds}{dt} \quad dt = \frac{ds}{V}\\
ds = \sqrt{1+y^{12}}\ dx\\
\frac{1}{2}mv^2 = mgy \rightarrow v = \sqrt{2gy}
\end{array}\right\}
\rightarrow T = \int_{x_1}^{x_2} dx\ \sqrt{\frac{1+y^{12}}{2gy(x)}}$$

\subsection{Euler-Lagrange-Gl.}
Problem: Welche Funktion $y(x)$ macht Funktional
\begin{equation}\label{el:1}
J[y] = \int_{x_1}^{x_2} dx\ F(y,y',x) \tag{1}
\end{equation}
minimal, wobei differentierter Funktion $F$ und $y_1=y(x_1), y_2=y(x_2)$ bekannt sind.\\
Sei $y(x)$ die gesuchte Funktion mit $J[y] =$ min.\\
Die \underline{Variation} 
\begin{equation}\label{el:2}
y (x) \rightarrow y(x) + \epsilon \eta (t)  \tag{2}
\end{equation}
mit infinit. $\epsilon$ und beliebigen diff. baren Funktion $\eta(1)$ die die Randbeding. $\eta (x_1) = \eta (x_2) = 0$
\begin{equation}\label{el:3}
J[y+\epsilon \eta] \tx{ ist minimal bei } \epsilon = 0 \quad \forall \eta \tag{3}
\end{equation}
Die Bestimmung von y über (3) wird als \underline{Variationsrechnung} bezeichnet.
$$J[y+\epsilon \eta] = \int_{x_1}^{x_2} dx\ F(y + \epsilon \eta, y' + \epsilon \eta', x)$$
$$= \int_{x_1}^{x_2} dx\ [F(y,y',x) + \frac{\partial F}{\partial y} (y,y',x) \epsilon \eta(x) + \frac{\partial F}{\partial y'} (y,y',x) \epsilon \eta'(x)$$
Damit
$$0 = \frac{dJ(y + \epsilon \eta)}{d \epsilon}\bigg|_{\epsilon = 0} = \int_{x_1}^{x_2}dx\ [\frac{\partial F}{\partial y} \eta(x) + \frac{\partial F}{\partial y'} \eta'(x)]$$
[Par. Int: $\int uv' = \int uv-\int vu'$ für 2. Term]
$$\underbrace{\frac{\partial F}{\partial y'} \eta (t) \bigg|_{x_1}^{x_2}}_{= 0, \tx{ da } \eta(x_1) = \eta(x_2) = 0} + \int_{x_1}^{x_2} dx\ \bigg[ \frac{\partial F}{\partial y} - \frac{d}{dx}\frac{\partial F}{\partial y'} \bigg] \eta(t)$$
Da $\eta$ beliebig sein kann, muss Klammer verschwinden\\

\begin{equation}\label{el:4}
\rightarrow \hspace{20pt} \rmbox{\frac{d}{dx} \frac{\partial F ( y, y',t)}{\partial y'} = \frac{\partial F(y,y',t)}{\partial y}} \tag*{(4)}
\end{equation}

\paragraph{Euler-Lagrange-Gl.}
\begin{itemize}
	\item Hängt nicht von $\eta(x)$ ab
	\item notwendige Bedingung für Extremum
\end{itemize}
Mit variation $\delta_y = \epsilon \eta(x)$ können wir (3) schreiben
\begin{align}\label{el:5}
\delta J &= J[y + \delta y] - J[y] = 0 \tag*{(5)}\\
&= \int_{x_1}^{x_2} dx\ \bigg(\frac{\partial F}{\partial y} \delta y + \frac{\partial F}{\partial y'} \delta y'\bigg) \nonumber\\
&= \int_{x_1}^{x_2} dx\ \bigg(\frac{\partial F}{\partial y} - \frac{d}{dx}\ \frac{\partial F}{\partial y'}\bigg)\delta y \nonumber\\
&= \int_{x_1}^{x_2} dx\ \bigg(\frac{\delta F}{\delta y}\bigg)\delta y = \int_{x_1}^{x_2} dx\ + \delta F(y,y',t) \nonumber
\end{align}
wobei $$\frac{\delta F}{\delta y} = \frac{\partial F}{\partial y} - \frac{d}{dx}\ \frac{\partial F}{\partial y'} \tx{ als \underline{Funktionalableitung} bezeichnet wird}$$
Gl. \ref{el:4} und \ref{el:5} sind äquivalent
\frbox{\tx{,,}Variationsprinzip\tx{``}}{$$\frac{d}{dt}\frac{\partial F}{\partial y} = \frac{\partial F}{\partial y} \leftrightarrow \delta J = 0$$}

% neue vorlesung

\subsection{Variationsrechnung}
\underline{Funktional:}$ y=f(x) \rightarrow J[y] $\\
z.B. Weglänge s der Kurve $ y(x) $
$$J[y] = \int_{x_1}^{x_2} \sqrt{1+y'(x)^2} \, dx$$
\underline{allg:} $$ J[y] = \int_{x_1}^{x_2} dx F(y,y',x) \quad $$
Extrema über \underline{Variation} $ \delta_y = \epsilon \eta (x) $
$$ \frac{\partial J}{\partial t} \bigg|_{\epsilon=0} = 0$$
$$ \delta J = J[y+\delta_y] - J[y] = 0 \leftrightarrow \tx{ Extrema von } J$$
äquivalent zu den Euler-Lagrange-Gl
\frbox{Extremalbedingungen}{$$\frac{d}{dx} \frac{\partial F}{\partial \dot{y}} = \frac{\partial F}{\partial y} \quad \leftrightarrow \quad \delta J = 0 $$}
\underline{Bsp: Kürzeste Verbindung}
$$J = \int ds = \int_{x_1}^{x_2} dx \ub{\sqrt{1 + y'^2}}_{F} = \tx{ min} $$
\underline{Euler-Lagrange-Gl.}
$$\frac{d}{dx} \frac{\partial F}{\partial \dot{y}} = \frac{d}{dx} \frac{2 y'(x)}{2 \sqrt{1 + y'^2}} = \frac{\partial F}{\partial y} = 0 $$
\underline{Integration:}$$\frac{\partial F}{\partial \dot{y}} \int_{x_1}^{x_2} = \frac{y'(x_2)}{\sqrt{1+y'(x_2)^2}} - \frac{y'(x_1)}{\sqrt{1+y'(x_1)^2}} =0 $$
$$\rightarrow y' = \const \quad , \quad y(x)= ax+b$$
$a,b$ aus $ x_1,x_2 $
\paragraph{Verallgemeinerung:}
\begin{enumerate}[(1)]
	\item Hängt F von meheren Funktionen $ y_i \quad (i=1,\dots ,f) $ ab, so erhalten wir $ \frac{\partial J}{\partial \epsilon_i} \big|_{\epsilon_i = 0} = 0 $ und damit f Euler-Lag-Gl.
	$$\rmbox{\frac{d}{dx} \frac{\partial F}{\partial y'_i} = \frac{\partial F}{\partial y_i}}$$
	\item Hängt y von mehreren Argumenten ab, $ y = y(x_1,\dots,x_n) $ so erhalten wir
	$$ J[y] = \int dx_1 \dots \int dx_n F(y,\frac{\partial y}{\partial x_1} , \dots , \frac{\partial y}{\partial x_n} , x) $$
	Mit $ \frac{\partial J}{\partial E} \bigg| _{\epsilon = 0} = 0 $\\
	\underline{Euler-Lagrange-Gl:}
	$$\sum_{i=1}^{n} \frac{\partial }{\partial x_i} \frac{\partial F}{\partial y / \partial x_i} - \frac{\partial F}{\partial y} = 0$$
\end{enumerate}

\section{ 3. Hamiltonsche Prinzip}
Die Korrespondierenz 
\begin{align*}
y_i(x) \quad &\leftrightarrow \quad q_i (t) \\
F(y,y',x) \quad &\leftrightarrow \quad L(q,\dot{q}, t) \\
\frac{d}{dx} \frac{\partial F}{\partial y'_i} = \frac{\partial F}{\partial y} \quad &\leftrightarrow \quad \frac{d}{dt} \frac{\partial L}{\partial \dot{q}_k} = \frac{\partial L}{\partial q_i}
\end{align*}
\begin{small}
	$ y = (y_1,\dots , y_f) \qquad q = (q_1,\dots , q_f) $
\end{small}
\begin{itemize}
	\item[ $ \rightarrow $ ]
	\item Lösungsverfahren der Lagrange-Gl. (z.B. über Erhaltungssätze) können in Variationsrechnung verwendet werden
	\item physikalische Bedeutung der Variationsrechnung
\end{itemize}
Wir ordnen jeder Bahnkurve $ q(t) $ ein Wirkungsfunktional 
$$ \rmbox{S = S[q] = \int_{t_1}^{t_2} dt \ L(q,\dot{q},t) }$$
$S$ wird oft \underline{Wirkung} genannt\\
Gemäß dem Variationsprinzip sind damit die Lagrange-Gleichungen äquivalent zu der Formulierung, dass die Variation der Wirkung gleich Null ist:
\frbox{\tx{,,}Hamiltonsche Prinzip\tx{``}}{$$\delta S[q] = 0$$} 
Bewegung verläuft so, dass Bahnkurve $ q(t) $ die Wirkung $ S $ minimiert: ,, Prinzip der kleinsten Wirkung``\\
\underline{Bem:}
\begin{itemize}
	\item Anstele von \underline{DGL} (wie Newton, Lagrange) kann das Grundgesetz der Mechanik also auch als \underline{Variationsprinzip} formuliert werden
	\item andere Bsp:\\
	\begin{itemize}
		\item \underline{Optik: Fermatsche Prinzip}\\
		Licht nimmt seinen Weg so, dass die Laufzeit\\
		$ t[x] = \frac{1}{C} \int_{x_1}^{x_2} dx \ n(x) \buildrel ! \over = \tx{ min } \quad n(x) $: Brechungsindex
		\item \underline{Thermodynamik:} 2. Hauptsatz\\
		Entropie S nimmt immer zu
		\item auch die QM kann durch ein Variationsprinzip beschrieben werden
	\end{itemize}
\end{itemize}

\subsection{Schwingung einer Seite}
Seite zwischen zwei wänden mit Abstand l auf der x Achse
Auslenkung $ n(x,t) \widehat{=} $ Feld
\begin{itemize}
	\item Bsp für Kontinuumsmechanik, d.h. Dynamik elastischer Körper inklusive Balkenbiegung und Hydrodynamik
	\item Bsp für einfahce klassische Feldtheorie
	\item führt auf Wellengleichung
	\item Analogie zu QM
\end{itemize}

\underline{Bsp für Felder:} el.Feld $ \vec{E}(\vec{r},t) $ , Temperaturfeld $ T(\vec{r},t) $, Wellenfunktion $ \Psi(\vec{r},t) $ zentrale Größe in Feldtheorie\\
Bewegungsgleichung für Felder, sogenannte Feldgleichung, sind partielle DGL\\
\subsubsection{Herleitung der Wellengleichung}
\underline{Ansatz:}
\begin{itemize}
	\item Saite $ \widehat{=} N $ Massenelemente $ \Delta m $ durch Federn verbunden
	\item zuletzt: $ N \rightarrow \infty $ ,,Kontinuumslimes``
	\item $ l= N \Delta x \quad N \gg 1 $
	\item Massen $ \Delta m_i $ bei $ x_i - (i-\frac{1}{2}) \Delta x \quad (i=1,\dots, N) $ Auslenkung $ u_i(t) = u(x_i,t) $ 
\end{itemize}
\begin{equation*}
T = \sum_{i} \frac{\Delta m_i}{2} \bigg( \frac{du_i}{dt} \bigg) ^2 = \sum_i \frac{\rho \Delta x}{2} \dot{u}_i^2 \tag{1}
\end{equation*}
mit Massendichte $ \rho = $ Masse/Länge \\
Abstand zwischen $ i $-ten und $ (i+1) $-ten Massenpunkt
$$\Delta s = \sqrt{\Delta x^2 + (u_{i+1} - u_i)^2} $$
für kleine Auslenkungen
$$ \approx \Delta x [1 + \frac{(u_{i+1} - u_i)^2}{2 \Delta x^2}]$$
(keine wurzel mehr wegen 1. Ordnung Entwicklung ...)\\
In Ruhelage $ \Delta s = \Delta x $ existiert eine Vorspannkraft $ P $ der Seite\\
Beitrag zur pot. Energie $ \sim P \cdot (\Delta s - \Delta x) $
\begin{equation*}
U = \sum_i P \Delta x \frac{(u_{i+1} - u_i)^2}{2 \Delta x^2} \tag{2}
\end{equation*}
$ N \rightarrow \infty : \quad u_i(t) \rightarrow u(x,t) $
\begin{align*}
T &= \lim\limits_{N\rightarrow \infty} \frac{\rho}{2} \sum_i \Delta x \bigg( \frac{\partial u(x_i,t)}{\partial t} \bigg) ^2\\
&= \frac{\rho}{2} \int_{0}^{l} dx \ub{\bigg( \frac{\partial u(x_i,t)}{\partial t} \bigg) ^2}_{=: \dot{u}^2}
\end{align*}
$$ \lim_{\Delta x \to 0} \frac{u_{i+1}(t) - u_i(t)}{\Delta x} = \frac{\partial u(x,t)}{\partial x} = u'$$
\begin{align*}
U &= \lim\limits_{N\rightarrow \infty} \frac{\rho}{2} \sum_i \Delta x \frac{(u_{i+1} - u_i)^2}{\Delta^2 x}\\
&= \frac{\rho}{2} \int_{0}^{l} dx \ub{\bigg( \frac{\partial u(x,t)}{\partial x} \bigg) ^2}_{=: u'^2}
\end{align*}

Damit ist
\begin{equation*}
L(\dot{u},u') = \int_{0}^{l} dx \ \ub{\bigg[ \frac{\rho}{2} \dot{u}^2 - \frac{P}{2} u'^2\bigg]}_{\tx{ \underline{Lagrange-Dichte}} \quad \mathcal{L} (\dot{u},u') } \tag{3}
\end{equation*}
\underline{Hamilton Prinzip}
$$ \delta S = \delta \int_{t_1}^{t_2} dt \int_{0}^{l} dx \mathcal{L} (\dot{u},u') = 0$$
\underline{Euler-Lagrange-Gl.}\\
für 2 Argumente $ x_1 = x, x_2 = t $\\
und $ F = \mathcal{L}, y = u $
$$\frac{\partial}{\partial t} \frac{\partial L}{\partial \dot{u}} + \frac{\partial}{\partial x} \frac{\partial \mathcal{L}}{\partial u'} - \frac{\partial \mathcal{L}}{\partial u} = 0 $$
$$ \rho \ddot{u} + P u'' - 0 = 0 $$
mit $ c = \sqrt{\frac{P}{S}} $ ,, Wellengeschwindigkeit``\\
\frbox{Wellengleichung}{$$\frac{\partial^2 u(x,t)}{\partial x^2} - \frac{1}{c^2} \frac{\partial^2 u(x,t)}{\partial t} = 0 $$}
\noindent
\underline{Randbedingungen:} $  U(0,t) = u(l,t) = 0 $\\
\underline{Anfangsbedingungen:} $ U(x,t_0) = u_0\ , \ \dot{u}(x,t_0) = \dot{u}_0 $\\
\paragraph{Lösung der Wellengleichung:}
\underline{Ansatz:} \\
$ u(x,t) = u_0 e^{\pm i(kx-\omega t)} $\\
Amplitude $ u_0 $, Wellenvektor $ k $, Frequenz $ \omega $\\
eingesetzt: $ u'' = \frac{1}{c^2} \ddot{u} \rightarrow - u k^2 = - \frac{1}{c^2} \omega^2 u $\\
Lösung, falls $ \rmbox{\omega = ck} $ \underline{Dispersionsrelation}\\
allg. Lösung durch Linearkombination im Kontinuumlimit erhalten
$$ u(x,t) = \int_{-\infty}^{\infty} dk \ u(k) e^{\pm (kx - \omega t)} \qquad \tx{ Fourier-Trafo} $$
\underline{Bsp:} Stehende Welle\\
Superposition zwischen rechtslaufenden und linkslaufenden Wellen
\begin{align*}
u &= \cos (kx - \omega t) - \cos(-kx + \omega t) \\
& = \tx{Re } e^{ikx} \ub{ \bigg( e^{-i\omega t} - e^{i \omega t} \bigg) }_{2 i \sin \omega t}\\
& = 2 \tx{ Im } e^{ikx}\sin \omega t = 2 \sin kx \sin \omega t
\end{align*}
d.h. unabhängig von $ t $ erhält man ,,Knoten`` (Nullstellen) für $ \sin kx = 0 $\\
\underline{Randbedingung:} $ u(x=0,t) = u(x=l,t) = 0 $\\
$ \sin kl = 0 \rightarrow k = \pi \frac{n}{l} $\\
s.h. es können nur Wellen mit bestimmten Wellenlängen $ \lambda_n $ auftreten
$$ \lambda_n = \frac{2 \pi }{k} = \frac{2 l}{n}$$
$$ \omega_n = ck = \frac{n \pi c}{l} $$
$ n = 1,2,3\dots $\\
ein Wellenbauch $ \rightarrow $ ,,Grundschwinung``\\
zwei Bäuche (doppelte Frequenz)  ,,1. angeregte Schwingung`` 1. Oberton \\
drei Bäuche (1,5 fache Frequenz) ,,2. angeregte Schwinung`` 2. Oberton \\
Analogie zum QM Teilchen im Kasten\\
$ \rightarrow $ Quantisierung

% neue Vorlesung

\subsection{Der Starre Körper}
\begin{itemize}
	\item System von Massenpunkten mit $ |\vec{r}_i - \vec{r}_j | = \tx{const.} $\\
	z.B.
	\begin{itemize}
		\item starres Molekül
		\item Näherung für kontinuierliche Massenverteilung
	\end{itemize}
	\item Bewegung besteht aus
	\begin{itemize}
		\item Translation, d.h. alle Teilchen haben gleiche Geschwindigkeit
		\item Drehung um einen körperfesten Koordinaten Ursprung $ 0 $
		\item[$ \rightarrow $] $ 2+3=6 $ Freiheitsgrade
	\end{itemize}
\end{itemize}
Raumfestes Inertialsystem (IS) mit $ x,y,z $ und \\
Körperfestes Koord.system (KS) $ x_1,x_2,x_3 $\\
KS: Ursprung bei $ 0 \ \widehat{=} $ im IS $ \vec{r}_0(t) $\\
(z.B. der Schwerpunkt) $ \rightarrow \vec{v}_0 = \frac{d\vec{r}_0}{dt} $\\
KS dreht sich relativ zum IS mit
$$ \vec{\omega}(t) = \frac{d\vec{\varphi}}{dt}$$
$ \vec{r}_n $ seien die Orte des n-ten Teilchens im KS\\
$ \vec{r}_{n,\tx{IS}} $ seien die Orte des n-ten Teilchens im IS
$$ \vec{r}_n = \vec{r}_{r,\tx{IS}} - \vec{r}_0$$
Geschwindigkeit im IS:
$$ \frac{d \vec{r}_{r,\tx{IS}}}{dt} = \ub{\frac{ d \vec{r}_0}{dt}}_{= v_0} + \frac{d\vec{r}_n}{dt}$$
Mit I.4 Beschleunigte Bezugssysteme: \\
für beliebigen Vektor $ \vec{G} $ ist
\begin{equation}
\frac{d\vec{G}_{\tx{IS}}}{dt} = \frac{d \vec{G}_{\tx{KS}}}{dt} + \vec{\omega} \times \vec{G}_{\tx{KS}} \tag{$*$}
\end{equation}
$$ \tx{z.B.} \quad \frac{d \vec{r}_{\tx{IS}}}{dt} = \frac{d \vec{r}}{dt} + \vec{\omega} \times \vec{r}$$
\begin{equation}
\vec{v}_{n,\tx{IS}} = \vec{v}_0 + \vec{\omega}\times \vec{r}_n \tag{1}
\end{equation}
\subsubsection{Kinetische Energie}
\begin{align*}
T &= \sum_n \frac{m_n}{2} \vec{v}_{n,\tx{IS}}^2 = \sum_n \frac{m_n}{2} [\vec{v}_0 + \vec{\omega}\times \vec{r}_n] \qquad \qquad n = 1,\dots , N \\
&= \sum_n \frac{m_n}{2} \vec{v}_0^2 + \ub{\sum_n m_n \vec{v}_0 (\vec{\omega} \times \vec{r}_n)}_{\downarrow \tx{ Spatprodukt}} + \sum_n \frac{m_n}{2} (\vec{\omega} \times \vec{r}_n)^2 \\
&= \frac{M}{2} \vec{v}_0^2 + (\vec{v}_0 \times \vec{\omega}) \sum_n m_n \vec{r}_n + \sum_n \frac{m_n}{2} (\vec{\omega} \times \vec{r}_n)^2 \\
&= T_{\tx{trans}} + T_{\tx{rot-trans}} + T_{\tx{rot}} \tag{2}
\end{align*}
\paragraph{\underline{2 Fälle:}} 
\begin{itemize}
	\item Körper wird in keinem Punkt festgehalten\\
	Mit $ \vec{0} = \vec{R} $ (Schwerpunkt) ist $ \sum\limits_i m_i \vec{r}_i = 0 $
	$$ \rightarrow T = T_{\tx{trans}} + T_{\tx{rot}}$$
	\item Körper wird in mindestens einem Punkt festgehalten\\
	Mit $ \vec{0} = \vec{P} $ ist $ \vec{v}_0 = 0 \qquad $ (z.B. bei Kreisel) 
	$$ \rightarrow T = T_{\tx{rot}}$$
\end{itemize}
Mit $ \vec{\omega} = ( \omega_1,\omega_2,\omega_3) \quad \vec{r}_n = (x_{1n},x_{2n},x_{3n}) $\\
Und der Identität
\begin{align*}
(\vec{\omega} \times \vec{r})^2 &= \omega^2 r^2 - (\vec{\omega} \cdot \vec{r})^2\\
&= \sum_{k=1}^{3} \omega^2_k r^2 - \sum_{i,k}^{3} \omega_ix_i - \omega_k x_k \qquad i,j,k = 1,2,3\\
&= \sum_{i,k} \omega_i \omega_k (r^2 \delta_{ik} - x_i x_k)
\end{align*}
wird
\begin{align*}
T &= \frac{1}{2} \sum_{n=1}^{N} m_1 (\vec{\omega} \times \vec{r}_n)^2\\
&\equiv \frac{1}{2} \sum_{i,k = 1}^{3} I_{ik} \omega_i \omega_k
\end{align*}
mit dem \underline{Trägheitstensor}
\begin{equation*}
\rmbox{I_{ik} = \sum_{n=1}^{N} m_n (r_n^2 \delta_{ik} - x_{ik} x_{kn}) } \tag{3}
\end{equation*}
Im Matrixschreibweise $ I = \{I_{ik}\} $ ist
\begin{equation*}
\rmbox{T_{\tx{rot}} = \frac{1}{2} \vec{\omega} I \vec{\omega} } \tag{4}
\end{equation*}
\paragraph{Bem:}
\begin{itemize}
	\item Begriff Tensor: ursprünglich von Spannungstensor (Physik) \\
	Mathe: Tensor \\
	1. Stufe $ \widehat{=} $ Vektor\\
	2. Stufe $ \widehat{=} $ Matrix
	\item (4) ist eine Bilinearform
	\item Dreht sich Kürper um eine körperfeste Achse (z.B. $ \vec{\omega} = \begin{pmatrix}
	0\\ 0\\ \omega z
	\end{pmatrix} $) \\
	so geht (4) über in
	$$ T_{\tx{rot}} = \frac{I_z}{2} \omega_z^2$$\\
	$ I_z $: Trägheitsmoment des Körpers bzgl. $ \vec{e}_z $
	\item Bei kontinuierlicher Massenverteilung mit Dichte $ \rho(\vec{r}) $ ist 
	$$I_{ij} = \int d \vec{r} \rho(\vec{r}) (\vec{r}^2 \delta_{ij} - rir_j)$$
\end{itemize}
Trägheitstensor ist symm., $ T_{ij} = T_{ji} $ und kann daher mit einer orthogonalen Trafo U\\
$( U^+ = U^{-1}, U^+ U = 1 = U U ^+) $\\
auf Diagonalform gebracht werden
$$U^+ I U = \begin{pmatrix}
I_1& & 0 \\
& I_2 & \\
0 & & i_3
\end{pmatrix} \rightarrow \tx{ Eigenwert Problem ,,Hauptachsen-Trafo``}$$
wobei die Eigenvektoren die Hauptträgheitsachsen und die Eigenwere $ I_i \ \ (i=1,2,3) $ die Hauptträgheitsmomente sind.\\
Symmetrie des Körpers legt die Achse fest z.B. bei Kreisel
\begin{itemize}
	\item \underline{,,Kugelkreisel``}; wenn alle $ I_i $ gleich sind\\
	(Kugel, Würfel, Zylinder mit $ h=\sqrt{3}r $)
	\item \underline{Symmetrischer} Kreisel; $ I_1 = I_2 ,  I_3 \neq I_1 $
	\item \underline{asymmetrischer} Kreisel; alle $I$ verschieden
\end{itemize}
\subsubsection{Drehimpuls:}
in KS ist
\begin{align*}
\vec{L} &= \sum_{n=1}^{N} m_n (\vec{r}_n \times \dot{\vec{r}}_n)\\
&\buildrel (1) \over = \sum_n m_n (\vec{r}_n \times (\vec{\omega} \times \vec{r}_n))
\end{align*}
Mit
\begin{align*}
\vec{r} \times (\vec{\omega} \times \vec{r}) &= \vec{\omega}r^2 - \vec{r} ( \vec{\omega}\cdot \vec{r})\\
&= \sum_{i,k=1}^{3} (r^2 \delta_{ij} - x_i x_n) \omega_k \vec{e}_i
\end{align*}
ist 
$$ \vec{L} = \sum_{i,k = 1}^{3} I_{ik} \omega_k \vec{e}_i = \sum_i L_i \vec{e}_i$$
oder 
\begin{equation*}
\rmbox{\vec{L} = I \vec{\omega}} \qquad \tag{5} 
\end{equation*}
\subsubsection{Eulersche Gleichungen}
$$ \frac{d}{dt} \vec{L} = \vec{M}$$
\begin{align*}
\frac{d}{dt} \vec{L} &= \frac{d}{dt} \sum_{n=1}^N m_n (\vec{r}_n \times \dot \vec{r}_n) = \frac{d}{dt} ( I \cdot \vec{\omega})\\
&= \vec{M} = \sum_{n=1}^{N} \vec{r}_i \times \vec{F}_n^A
\end{align*}
Mit Gl. $ (*) $ ist dann
\begin{equation*}
\frac{d}{dt} (I \vec{\omega})_{\tx{IS}} = \frac{d}{dt} (I \vec{\omega})_{\tx{KS}} + \vec{\omega} \times (I \vec{\omega}) = \vec{M} \tag{6}
\end{equation*}
Ist KS das Haupsachsensystem mit 
$$ I = \begin{pmatrix}
I_1 & & 0 \\
& I_2 & \\
0 & & I_3\\
\end{pmatrix}$$
so ist 
$$ \frac{d}{dt} (I \vec{\omega})_{\tx{KS}} = \begin{pmatrix}
I_1 \dot{\omega}_1\\
I_2 \dot{\omega}_2\\
I_3 \dot{\omega}_3
\end{pmatrix}$$
$$ \begin{pmatrix}
\omega_1\\ \omega_2 \\ \omega_3
\end{pmatrix} \times \begin{pmatrix}
I_1 \omega_1\\ I_2 \omega_2 \\ I_3 \omega_3
\end{pmatrix} = \begin{pmatrix}
\omega_2 \omega_3 (I_3-I_2)\\
\omega_1 \omega_3 (I_1-I_3)\\
\omega_1 \omega_2 (I_2-I_1)
\end{pmatrix}$$
und damit\\ 

% hier rmbox eulergleichung in fav bild einfügen und weiter unten (8) nachtragen: ??? is glaub ich jetzt egal

\frbox{Eulersche Gleichungen}{\begin{align*}
	I_1\dot{\omega}_1 + (I_3-I_2) \omega_2 \omega_3 &= M_1 \\
	I_2\dot{\omega}_2 + (I_1-I_3) \omega_1 \omega_3 &= M_2 \tag{7}\\
	I_3\dot{\omega}_3 + (I_2-I_1) \omega_2 \omega_1 &= M_3 \\
	\end{align*}} 

I.A. Schwer zu Lösen, da nicht linear und M zeitabhängig
\subsubsection{\underline{Bsp: Kräftefreier symmetrischer Kreisel}}
Körper steht auf Schwerpunkt (wie falschrumer Blumentopf) ($ x_3 $ in der Kreiselsymmetrieachse: ,, Figurenachse``)\\
$ I_1 = I_2 \quad I_3 \neq I_1  \rightarrow \tx{ relationssymm. bzgl. } x_3 \tx{ Achse}$\\
$ \vec{M} = 0 $ Damit:\\
\begin{equation*}
I_1 \dot{\omega_1} + (I_3-I_2) \omega_2 \omega_3 = 0 \rightarrow \dot{\omega}_1 - \Omega\omega_2 = 0 \tag{a}
\end{equation*}
\begin{equation*}
I_2 \dot{\omega}_2 + (I_1-I_3) \omega_1 \omega_3 = 0 \rightarrow \dot{\omega}_2 - \Omega\omega_1 = 0 \tag{b}
\end{equation*}
$$I_3 \dot{\omega}_3 = 0 \rightarrow \omega_3 = \tx{ const. } \equiv \omega_0$$
$$ \Omega = \frac{I_1 - I_3}{I_1} \omega_0$$
$\frac{d}{dt} (a)$ mit $ (b) $ : $ \ddot{\omega}_1 + \Omega^2 \omega_1 = 0 $
\begin{align*}
& \rightarrow \omega_1(t) = a \sin (\Omega t + \varphi_0)\tag{8}\\
\omega_2 = \frac{\dot{\omega}_1}{\Omega} \quad & \rightarrow \omega_2(t) = a \cos(\Omega t + \varphi_0)  \tag{8}
\end{align*}
$$ \omega_3 = \omega_0$$
Mit
$$ \vec{\omega}^2 = \omega_1^2(t) + \omega_2^2(t) + \omega_3^2 = \const$$

hat die Projektiion von $ \vec{\omega} $ auf $ x_1 $- $ x_2 $-Ebene die konst. Länge a und rotiert mit $ \Omega $.\\
D.h. im KS rotiert Kreisel auf einem Polkegel (mit $ \omega_0 $ um die eigene symm. Achse)\\
$ \gamma = \arctan \frac{a}{\omega_0} = \tx{ const.} $\\
Zur Betrachtung im IS brauchen wir verallg. Koord. um die Beziehung zwischen KS und IS zu beschreiben\\
$ \rightarrow $ Eulersche Winkel $ \Phi, \Psi, \Theta  \rightarrow 	$ siehe Übung\\
\begin{equation*}
\rightarrow \rmbox{\begin{array}{l}
	\omega_1 = \dot{\Phi} \sin \Theta \sin \Psi + \Theta \cos \Psi\\
	\omega_2 = \dot{\Phi} \sin \Theta \cos \Psi + \Theta \sin \Psi\\
	\omega_3 = \dot{\Phi} \cos \Theta + \dot \Psi
	\end{array}} \tag{9}
\end{equation*}
Einsetzen von (9) in (8), ergibt nach Lösung von (9)
$$ \Psi(t) = \Omega t + \Psi_0 \qquad \Phi(t) = \frac{q}{\sin \Theta_0} + \Phi_0$$
$$\tan\Theta_0 = \frac{q}{\omega_0} \frac{I_1}{I_2}$$

\subsection{Rotation des Starren Körpers}

IS  $ \vec{r}_{IS} $ und KS $ \vec{r} $
$$\frac{d \vec{r}_{IS}}{dt} = \frac{d \vec{r}}{dt} + \vec{\omega} \times \vec{r}$$
\begin{align*}
T_{\tx{rot}} &= \frac{1}{2} \sum_{n=1}^{N} m_n (\vec{\omega} \times \vec{r})^2\\
&= \frac{1}{2} \sum_{i,k=1}^{3} I_{ik} \omega_i \omega_k \\
&= \frac{1}{2} \vec{\omega} I \vec{\omega}
\end{align*}
\begin{equation*}
I _{ik} = \sum_n m_n \bigg( r_n^2 \delta_{ik} - x_{in} x_{kn}\bigg) \quad \tx{\underline{Trägheitstensor}}
\end{equation*}
orthogonale Trafo : $ U^+ I U = \begin{pmatrix}
I_1 & & 0 \\
& I_2 \\
0 & & I_3
\end{pmatrix} $ \\
$I_i \widehat{=} $ Hauptträgheitsmomente, Hauptträgheitsachse
\begin{align*}
\vec{L} &= \sum_n m_n (\vec{r}_n \times \dot{\vec{r}}_n)\\
&= I \omega
\end{align*}
Mit $ \frac{d\vec{L}}{dt} = \vec{M} $ folgen Euler-Gl. (in KS)\\
Kräftefreier Kreisel: $ \vec{M} = 0 \rightarrow $ Präzession
\subsubsection{\underline{Schwerer Kreisel}}
Betrachte symm. Kreisel mit den Hauptträgheitsmomenten $ I_1=I_2 , I_3 \neq I_1 $\\
(fixer Auflagepunkt Rotation um $ x_3 $ Präzession um z mit Winkel $ \theta $ zwischen z und $ x_3 $ und der Schwerpunkt S auf $ x_3 $ höhe s des Kegelkreisels wird mit $ mg $ in Richtung -z gezogen)\\
\subsubsection{\underline{Lagrange-Funktion}}
$$\lag = T-U = \ub{\frac{1}{2} m \dot{\vec{R}}^{\, 2}}_{=0} + \frac{1}{2} I_1 \omega_1^2 + \frac{1}{2} I_2 \omega_2^2 + \frac{1}{2} I_3 \omega_3^2 - mg s \cos \theta$$
Mit Eulerwinkel 
\begin{align*}
\omega_1 &= \dot{\Phi} \sin \Theta \sin \Psi + \dot{ \Theta} \cos \Psi\\
\omega_2 &= \dot{\Phi} \sin \Theta \cos \Psi - \Theta \sin \Psi\\
\omega_3 &= \dot{\Phi} \cos \Theta + \dot{\Psi}
\end{align*}
$ \Phi: $ Drehwinkel um $ \vec{e}_z $, $ \Psi: $ Drehwinkel um $ \vec{e}_{x_3} $\\
\begin{equation*}
\lag = \frac{I_1}{2} \bigg( \dot{\Phi}^2 \sin ^2 \Theta + \Theta ^2 \bigg) + \frac{I_3}{2} \bigg( \dot{\Phi} \cos \Theta + \dot{\Psi}\bigg) ^2 - mgs \cos \Theta \tag{1}
\end{equation*}
\subsubsection{\underline{Symmetrien:}}
\begin{equation*}
\frac{\partial \lag}{\partial t} = 0 \quad , \quad \frac{\partial \lag}{\partial \Theta} = 0 \quad , \quad \frac{\partial \lag}{\partial \Psi} = 0
\end{equation*}
\subsubsection{\underline{Erhaltungsgrößen} ,,erste Integrale``}
Energieerhaltung:
\begin{align*}
E &= T+U \\
&= \frac{I_1}{2} \bigg( \dot{\Phi} ^2 \sin^2 \Theta + \dot{\Theta}^2 \bigg) + \frac{I_3}{2} \bigg( \dot{\Phi} \cos \Theta + \dot{\Psi} \bigg) ^2 + mgs\cos \Theta = \const \tag{2}\\[20pt]
l_z &= \frac{\partial L}{\partial \dot{\Phi}}\qquad  \tx{Drehimpulskomponente } l_z\\
&= I_1 \sin^2 \Theta \dot{\Phi} + I_3 \ub{ \bigg( \cos \Theta \dot{\Psi} + \cos^2 \Theta \dot{\Phi} \bigg) }_{(\dot{\Psi} + \dot{\Phi} \cos \Theta) \cos \Theta} = \const \tag{3}\\[10pt]
l_3 &= \frac{\partial l}{\partial \dot{\Psi}} \qquad \tx{Drehimpulskomponente } l_3 \\
&= I_3 \bigg( \dot{\Psi} + \dot{\Phi} \cos \Theta \bigg) = \const \tag{4}
\end{align*}
Durch Einsetzen von $(3)$ und $(4)$ in $(2)$ werden $ \dot{\Phi} $ und $ \dot{\Psi} $ eliminiert. Mit:
\begin{equation*}
l_z - l_3 \cos \Theta = I_1 \sin^2 \Theta \dot{\Phi} \tag{5}
\end{equation*}
ist 
\begin{equation*}
\frac{(l_z - l_3 \cos \Theta)}{2 I_1 \sin^2 \Theta} = \frac{I1}{2} \sin^2 \Theta \dot{\Phi}^2 \tag*{in (2)}
\end{equation*}
\begin{equation*}
E = \frac{I_1}{2} \dot{\Theta}^2 + \frac{(l_z - l_3 \cos \Theta)}{2 I_1 \sin^2 \Theta} + \frac{l_3^2}{2 I_3} + mgs \cos \Theta = \const \tag{6}
\end{equation*}
Wir schreiben
$$E = \frac{I_1}{2} \dot{\Theta}^2 + U_{\tx{eff}} (\Theta)$$
\begin{equation*}
U_{\tx{eff}} = \frac{(l_z - l_3 \cos \Theta)}{2 I_1 \sin^2 \Theta} + \frac{l_3^2}{2 I_3} + mgs \cos \Theta \tag{7}
\end{equation*}
Ähnlichkeit zum Kepplerproblem ergibt sich hier eine 1D Bewegungsgleichung mit effektive, Potential.\\
Aufgelöst nach $ \frac{d\Theta}{dt} $ und integriert ist:
\begin{equation*}
t = t_0 + \int_{\Theta_0}^{\Theta} d \Theta' \sqrt{\frac{I_1/2}{E - U_{\tx{eff}} (\Theta')}}
\end{equation*}
nicht elementar Lösbar aber Graphisch diskutierbar
\subsubsection{\underline{Graphische Disskusion der Lösung}}
\begin{itemize}
	\item $ U_{\tx{eff}} (\Theta) \buildrel \Theta \rightarrow 0 \over \longrightarrow \infty $ und $ \buildrel \Theta \rightarrow \pi \over \longrightarrow $
	\item dazwischen ein Minimum
\end{itemize}
(Parabel mit 2 Schnittpunkten mit E Energie des Systems an den Winkeln $ \Theta_1 $ und $ \Theta_2 $ und dazwischen Minimum)
aus $ E = U_{\tx{eff}} $ ergeben sich Umkehrpunkte $ \Theta_1, \Theta_2 $ Während die Figurenachse zwischen $ \Theta_1 $ und $ \Theta_2 $ oszilliert ,,Notation``, präzediert sie mit $ (5) $
\begin{equation*}
\dot{\Phi} = \frac{l_z - l_3 \cos \Theta}{I_1 \sin^2 \Theta} \qquad \tx{um die raumfeste z-Achse}
\end{equation*}
Bewegung ist definiert durch Kreiselparameter  $m$, $s$, $ I_1 $, $ I_2 $ und Anfangsbedingungen $ E $, $ l_z $, $ l_3 $\\
Für $ \Theta_1 = \Theta_2 $ verschwindet Notation ,,reguläre Präzession``\\
Im kräftefreien Limes (Grenzfall) $ (g\rightarrow 0) $ ergibt sich $ \Theta = \Theta_0 $, $ \dot{\Phi} = \const $\\

\section{9 Hamilton-Formalismus}
Bereits für konservative Systeme wurde\\
$$ \tx{Hamilton Funktion } \quad \ham(q,p,t) \quad \tx{,,Hamiltonian``}$$
und Hamilton Gleichungen hergeleitet.\\
Ausgehend vom kanonischen Impuls $ p = (p_1,\dots,p_f) $
\begin{equation*}
p_i = \frac{\partial L}{\partial \dot{q}_i} \tag{1}
\end{equation*}
leiten wie nun $ \ham(q,p,t) $ von $ \lag(q,\dot{q},t) $ her.
\subsubsection{Legendre-Trafo}
\begin{equation*}
d(x,y) \rightarrow g(u,y) \quad \tx{mit} \quad u = \frac{\partial f}{\partial x} \tag{2}
\end{equation*}
Ausgehend von
\begin{equation*}
df = \ub{\frac{\partial f}{\partial x}}_{:= u} dx + \ub{\frac{\partial f}{\partial y}}_{:= v} dy = u \, dx + v \, dy
\end{equation*}
definierten wir die Funktion $ g = f - ux $ mit:
\begin{equation*}
dg = df - u\, dx  - v\, du = v \, dy - x \, du \tag{3}
\end{equation*}
Folglich ist g die gewünschte Funktion
\begin{equation*}
dg = \frac{\partial g}{\partial u} du \rightarrow \frac{\partial g}{\partial y} dy \tag{4}
\end{equation*}
\begin{equation*}
v = \frac{\partial g}{\partial y} \quad , \quad x = -\frac{\partial g}{\partial u}
\end{equation*}

% neue Vorlesung

\section{Hamiltonsche Mechanik}
Ausgehend von $ \mathcal{L} = \mathcal{L}(q,\dot{q},t) \qquad q = (q_1,\dots ,q_f) $ und
\begin{equation*}
p_i = \frac{\partial \mathcal{L}}{\partial \dot{q}_i} \tag{1}
\end{equation*}
def. wir den Hamlitonian $ \ham = \ham ( q,p,t) $ über Legendre Trafo:
%$$F(x,y) \to g(u,y) \quad \tx{mit} \quad u = \frac{\partial %f}{\partial x}\quad \tx{und} \quad g = f -ux $$
\underline{Ansatz:} Löse Gl. (1) nach $  \dot{q}_i = \dot{q}_i(q,p,t) $ und def. Hamiltonian $ \ham $ als Legendre-Transformation von $ \lag $
\begin{equation*}
\rmbox{\ham(q,p,t) = \sum_i^{\phantom{i}} \dot{q}_i (q,p,t) p_i - \lag (q, \dot{q}(q,p,t) , t) }
\end{equation*}
Mit
\begin{align*}
d\ham &= \sum_i \bigg[ d \dot{q}_i p_i + \dot{q}_i dp_i - \ub{\frac{\partial \lag}{\partial q_i}}_{\frac{d}{dt} \frac{\partial \lag}{\partial \dot{q}_i} = \dot{p}_i} d q_i - \ub{\frac{\partial \lag}{\partial \dot{q}_i}}_{p_i} d \dot{q}_i \bigg] - \frac{\partial \lag}{\partial t} dt\\
&= \sum_i \bigg[ \dot{q}_i d p_i - \dot{p}_i d q_i \bigg] - \frac{\partial \lag}{\partial t} dt
\end{align*}
Verifizieren wir, dass $ \ham $ von $ q,p,t $ abhängt. Das Totale Differential von $ \ham $:
\begin{equation*}
d\ham = \sum_i \bigg[ \frac{\partial \ham}{\partial q_i} dq_i + \frac{\partial \ham}{\partial p_i} dp_i \bigg] - \frac{\partial \ham}{\partial t} dt
\end{equation*}
liefert durch Koeffizientenvergleich
\frbox{Hamilton-Gleichungen}{\begin{equation*}
	\dot{q}_i = \frac{\partial \ham}{\partial p_i} \quad , \quad \dot{p}_i = - \frac{\partial \ham}{\partial q_i} \tag{1}  
	\end{equation*}}
$ i= 1,\dots , f $ und:
\begin{equation*}
\rmbox{\frac{\partial \ham}{\partial t} = - \frac{\partial L}{\partial t}} \tag{2}
\end{equation*}
\subsubsection{Bemerkung:}
\begin{itemize}
	\item Gl. (1) wegen Einfachheit und Symmetrie auch \underline{kanonische} Gl. genannt
	\item  Die $2 f$ Variablen $ q_i $ und $ p_i $ sind völlig \underline{gleichberechtigt},
	\begin{itemize}
		\item $ p_i $ heißt auch zu \underline{,, $ q_i $ konjugierter Impuls``}
		\item $ q_i , p_i $ heißt \underline{,,Paar konjugierter Variablen``}
	\end{itemize}
	\item Wichtig: $ \ham $ darf keine Geschwindigkeit enthalten
	\item In Kapitel I.6 wurde bereits gezeigt, dass Energie erhalten ist für $ \frac{d\ham}{dt} = \frac{\partial \ham}{\partial t} = 0 $ da $ \ham $ nicht explizit von der Zeit abhängt.
	\item \underline{Zyklische Koordinaten:}\\
	Hängt $ \ham(p,q) $ nicht von $ q_i $ ab, $\frac{\partial \ham}{\partial q_i} = 0$\\
	$ \rightarrow p_i = \const $ Erhaltungsgröße
\end{itemize}
Für ein konservatives System mit
\begin{equation*}
\lag = T - U = \sum_i \frac{m_i}{2} \dot{q}_i^2 - U(t)
\end{equation*}
mit den kanonischen Impulsen $ p_i = \frac{\partial \lag}{\partial \dot{q}_i} = m_i \dot{q}_i $ entspricht
\begin{equation*}
\ham = \sum_ii \frac{p_i^2}{m_i} - \frac{p_i^2}{2 m_i} + U(t)
\end{equation*}
\begin{equation*}
\rmbox{\ham(p,q) = \sum_i^{\phantom{i}} \frac{p_i^2}{2 m_i} + U(t) = T + U } \tag{3}
\end{equation*}
also der Gesamtenergie.\\
Hier ist der kanonische Impuls $ p_i = \frac{\partial \lag}{\partial \dot{q}_i} $ gleich dem kinetischen Impuls $ m_i \dot{q}_i $. Gilt für zeitunabhängige, holonome Zwangsbedingungen ruhenden Koordinaten und konservativen Kräften
\begin{equation*}
\sum_i \dot{q}_i \frac{\partial \lag}{\partial \dot{q}_i} = \sum_i \dot{q}_i \frac{\partial T}{\partial \dot{q}_i} = 2 T
\end{equation*}
$\bigg[$ \underline{Bsp:} harm. Oszillator: $ \ham = \frac{p^2}{2 m} + \frac{m}{2} \omega^2 q^2 $\\
$\phantom{\quad} \dot{q} = \frac{\partial \ham}{\partial p} = \frac{p}{m} \quad , \quad \dot{p} = - \frac{\partial \ham}{\partial q} = - \frac{\partial U}{\partial q} = - m \omega^2 q = F$\\
$\phantom{\quad} F = \dot{p} = m \ddot{q} = - \frac{\partial U}{\partial q} \qquad \quad \ \: \bigg] $
\begin{itemize}
	\item Geht aber nicht, z.B. bei:
	\begin{itemize}
		\item geschw. abhängigen Kräften (Lorenz-Kraft)
		\item zeitabhängigen Zwangsbedingungen
	\end{itemize}
\end{itemize}
\subsubsection{\underline{Standartfall von f Freiheitsgraden $ q_i $}}
\begin{itemize}
	\item erhalten durch die Elimination von zeitunabhängigen holonomen Zwangsbedingungen (oder ohne diese)
	\item die nicht explizit zeitabhängigen sind (z.B. externer Antrieb)
	\item die konservativen Kräften genügen ist:
\end{itemize}
\begin{align*}
\lag(q,\dot{q}) &= T  - U = \sum_i \frac{m_i}{2} \dot{q}^2_i - U(q)\\
\ham(q,p) &= T + U = \sum_i \frac{p_i^2}{2 m_i} + U(q)
\end{align*}
mit: Bewegungsgleichungen sind äquivalent
$$ \frac{d}{dt} \frac{\partial \lag}{\partial \dot{q}_i} = m_i \ddot{q}_i = \frac{\partial \lag}{\partial q_i} = - \frac{\partial U}{\partial q_i} $$
$$ \dot{q}_i = \frac{\partial \ham}{\partial p_i} = \frac{p_i}{m_i} \quad , \quad \dot{p}_i = - \frac{\partial \ham}{\partial q_i} = - \frac{\partial U}{\partial q_i} \ \rightarrow \ m_i \ddot{q}_i = - \frac{\partial U}{\partial q_i}$$
$$F_i = m_i \ddot{q}_i = -\frac{\partial U}{\partial q_i}$$
\subsubsection{Mechanik nach:}
\begin{itemize}
	\item[\underline{Newton:}] \begin{itemize}
		\item über Def. der Kraft
		\item einfach und anschaulich
	\end{itemize}
	\item[\underline{Lagrange:}] \begin{itemize}
		\item über Def. von $ \lag(q,\dot{q},t) $
		\item  berücksichtigung von Zwangsbedingungen
		\item Zwangskräfte (Lag. Gl. 1. Art)
		\item Konzept von verallg. Koord. $ q_i $
		\item Konzept von zyklischen Variablen, $ \frac{\partial \lag}{\partial q_i} = 0 $\\
		$ \rightarrow $ Erhaltung von $ p_i = \frac{\partial \lag}{\partial \dot q_i} $
		\item Hamilton-Prinzip
		\item Ableitung von Feldtheorien
	\end{itemize}
	\item[\underline{Hamilton:}] \begin{itemize}
		\item über Def. von $ \ham(q,p,t) $
		\item Zwangsbedingungen nur implizit
		\item Konzept des Phasenraums
		\item[$ \rightarrow $] Ausgangspunkt für \underline{statistische Mechanik} und \underline{QM}
	\end{itemize}
\end{itemize}
\section{Phasenraum}

%%% Hier ist ein Beispiel für minipage und tikz

\begin{minipage}{.65\textwidth}
	\begin{itemize}
		\item $(q,p)$ bilden einen 2$f$-dim. Phasenraum (PR)
		\item Zustand ist im Phasenraum eindeutig beschrieben, d.h. ( $\frac{\partial \mathcal H}{\partial t}= 0$ ) schneiden sich Bahnen im Phasenraum nicht.
		\item $\mathcal H(p,q) = E =$ const. $\quad$ entspricht einer 2$f - 1$ - dim. Fläche im PR, welche das $2f$-dim. \textit{Phasenraumvolumen}
	\end{itemize}
\end{minipage}%
\begin{minipage}{.35\textwidth}%
	\begin{tikzpicture}
	\draw[->] (-2,0) -- (2,0) node[right] {$x$}; 
	\draw[->] (0,-2) -- (0,2) node[above] {$P$};
	\draw (0,0) circle [radius=1.5];
	\node [below=2cm, align=flush center,text width=5cm]
	{Harm. Oszill.};
	\end{tikzpicture}%
\end{minipage}  
$$V_{PR} (E) = \int dq_1 \dots dq_f\ \int dp_1 \dots dp_f$$
$$\mathcal H (p,q) < E$$
\begin{itemize}
	\item Klassisch entspricht ein endliches Phasenraumvolumen $\infty$ System zustände
	\item QM entspricht ein endliches Phasenraumvolumen endlich viele System zustände
\end{itemize}

\paragraph{Bsp:}
$$\mathcal H = \frac{p^2}{2m} + \frac{m\omega^2}{2} q^2 = \mathcal H(q,p)$$
$\cong$ Ellipse mit Halbachsen
$$a = \sqrt{\frac{2E}{m \omega^2}} \quad , \quad b = \sqrt{2mE}$$
d.h. PR-Volumen ist Fläche der Ellipse 
$$V_{PR} (E) = \pi a b = \frac{2 \pi E}{\omega}$$
Der WM Oszillator hat die dikrete Energiezustände [siehe Theo Phys III]
$$E_n = \hbar \omega ( n+ 1/2 ) \quad n = 0,1,2,3$$
Damit ist Anzahl der Zustände mit Energie $< E$
$$N_E = \sum_{E_n < E} 1 \simeq \frac{E}{\hbar \omega} = \frac{V_{PR} (E)}{2 \pi \hbar} \qquad ( N_E \gg 1)$$
d.h. wir messen das PR Volumen in Einheiten des Plankschen Wirkungsquantum $2 \pi \hbar = h$ eund erhalten somit die Anzahl der energetisch erreichbaren Zustände\\
Für $f$ Freihaitsgrade ist $$N_E \simeq \frac{V_{PE}(E)}{(2\pi \hbar) f}$$
Durch Einführung von \underline{abzählbaren} Zuständen liefert die PR-Beschreibung die Grundlage für die \underline{Statische Mechanik}

% neue Vorlesung

\subsection{Zeitentwicklung im Phasenraum (PR)}
\subsubsection{Poissonklammer}
Zeitentwicklung von $ A(q(t) , p(t) , t) $ ist gegeben
\begin{align*}
\frac{d}{dt} A(q(t),p(t),t) &= \sum_i \bigg( \frac{\partial A}{\partial q_i} \dot{q}_i + \frac{\partial A}{\partial p_i} \dot{p}_i \bigg) - \frac{\partial A}{\partial t}\\
&= \sum_i \bigg( \frac{\partial A}{\partial q_i} \frac{\partial \ham}{\partial p_i} - \frac{\partial A}{\partial p_i} \frac{\partial \ham}{\partial q_i} \bigg) + \frac{\partial A}{\partial t}
\end{align*}
\underline{Def:} Poissonklammer zweier PR-Funktion $ f(p,q,t) $ und $ g(p,q,t) $ ist

\begin{equation*}
\rmbox{ \{ f,g \} = \sum_i \bigg( \frac{\partial f}{\partial q_i} \frac{\partial g}{\partial p_i} - \frac{\partial f}{\partial p_i} \frac{\partial g}{\partial q_i} \bigg) } \tag{1}
\end{equation*}
Ist A explizit zeitabhängig $ \frac{\partial A}{\partial t} = 0 $, gilt
\begin{equation*}
\frac{d}{dt} A(q(t),p(t)) = \{A,\ham\} \tag{2}
\end{equation*}
D.h. wenn $ \{A,\ham\} = 0 \leftrightarrow A $ ist erhaltene Größe\\
\underline{Bsp:}
\begin{itemize}
	\item für radialsymmetrisches Potential ist Drehimpuls $ l_i $ erhalten\\
	$ \{l_i , \ham \} = 0 $
	\item Bewegungs-Geichung:\\
	\begin{align*}
	\{q_j,\ham \} &= \sum_i\bigg( \ub{\frac{\partial q_j}{\partial q_i}}_{\delta_{ij}} \frac{\partial \ham}{\partial p_i} - \ub{\frac{\partial q_j}{\partial p_i}}_{= 0} \frac{\partial \ham}{\partial q_i} \bigg) = \frac{\partial \ham}{\partial p_j} = \dot{q}_j\\
	\{p_j,\ham \} &= \sum_i\bigg( \ub{\frac{\partial p_j}{\partial q_i}}_{= 0} \frac{\partial \ham}{\partial p_i} - \ub{\frac{\partial p_j}{\partial p_i}}_{\delta_{ij}} \frac{\partial \ham}{\partial q_i} \bigg) = -\frac{\partial \ham}{\partial q_i} = \dot{p}_j\\
	\{q_i,p_i\} &= \sum_k \bigg( \ub{\frac{\partial q_i}{\partial q_k}}_{\delta_{ik}} \ub{\frac{\partial p_j}{\partial p_k}}_{\delta_{jk}} - \ub{\frac{\partial q_j}{\partial p_k}}_{= 0} \frac{\partial p_j}{\partial q_k} \bigg) = \delta_{ij}\\
	\end{align*}
\end{itemize}
\underline{Korrespondenz zur QM}
\begin{equation*}
\ub{[g,f]}_{\tx{klass. Poissonklammer}} \quad \longrightarrow \quad \ub{\frac{1}{i \hbar} [g,f]}_{\tx{QM kommutator}} = \frac{1}{i\hbar} (gf-fg) \tag{4}
\end{equation*}
Aus Gl (3) folgt, dann die QM Unschärferelation.\\
Zeitentwicklung einer QM Größe A
\begin{equation*}
\frac{d A}{d t} = \frac{1}{i \hbar} [A,\ham] + \frac{\partial A}{\partial t} 
\end{equation*}

\subsection{\underline{PR-Dichte}}
\underline{Bsp:} gedämpfter harmonischer Oszillator.\\[5pt]
(\underline{Ortsraum} $ p,t $ Graph abklingende Cosinus Schwingung, im \underline{Phasenraum} $ p,q $ kleiner werdende Spirale)\\[5pt]
Wir betrachten viele Teilchen $ (N \gg 1) $ mit kontinuierlich verteilten Anfangsbedingungen $ q_i(t_0) ,\  p_i(t_0) $ wie z.B. in exp. Messung eines Ensembles von Teilchen\\[10pt]
\underline{Def:} PR Dichte $ \rho(q,p,t) $ entspricht der Wahrscheinlichkeit, dass sich zur Zeit t am Phasenraumpunkt $ (p,p) $ ein Teilchen befindet. Ist die Teilchenzahl erhalten, gilt:
\begin{equation*}
\int dq \ dp \ \rho(q,p,t) = N \qquad \tx{,,Normierungsbedingung``}
\end{equation*}
\begin{equation*}
\frac{d \rho}{d t} = \sum_i \bigg( \frac{\partial \rho}{\partial q_i} \dot{q}_i + \frac{\partial \rho}{\partial p_i} \dot{p_i} \bigg) + \frac{\partial \rho}{\partial t} = 0
\end{equation*}
d.h. die Phasenraumdichte ist Zeitlich konstant. ,,Liouville Theorem``\\[5pt]
Wir erhalten die Liouville-Gl.
\begin{equation*}
\rmbox{\frac{\partial \rho}{\partial t} = \{\ham , \rho\}} \tag{6}
\end{equation*}
Mit Ersetzung (4) wird daraus in der Quantenmechanik die Liouville von Neumann Gleichung:
\begin{equation*}
i \hbar \frac{\partial \rho}{\partial t} = [\ham, \rho] \tag{7}
\end{equation*}

\chapter{Relativistische Mechanik}
\begin{tikzpicture}
\draw[->] (0,0) -- (2,0) node[right] {$x$}; 
\draw[->] (0,0) -- (0,2) node[above] {$y$};

\node at (0.7,2) [align=flush center,text width=5cm]
{IS};
\end{tikzpicture}%
\begin{tikzpicture}
\draw[->] (0,0) -- (2,0) node[right] {$x'$}; 
\draw[->] (0,0) -- (0,2) node[above] {$y'$};
%\draw[->] (0.1,0.5) -- (0.5,0.5) node[right] {$v = \vec{e}_s$}

\node at (1.5,0.8) [align=flush center,text width=5cm]
{$\longrightarrow \vec{v} = v\vec{e_x}$};
\node at (0.7,2) [align=flush center,text width=5cm]
{IS'};
\node at (4,1.8) [align=flush center,text width=5cm]
{$*$ Ereignis};
\end{tikzpicture}

IS stillstehend und IS' in Bewegung $ \vec{v} = v \vec{e}_x $\\[10pt]
\underline{Galilei Trafo:}
\begin{equation*}
x' = x + vt , \quad y' = y , \quad z' = z, \quad t' = t \tag{1}
\end{equation*}
\underline{Def:} Ein \underline{Ereignis} ist definiert durch Raum-Zeit-Koord. $ (x,y,z,t) $ und hat in IS und IS' verschiedene Koordinatenwerte.\\[10pt]
\underline{Bsp:} Schallwellen\\
Luft ist Träger für Schallwellen. Bewegt sich die Luft mit $ \vec{v} = v \vec{e}_x $ , dann breitet sich der Schall
\begin{itemize}
	\item bei ruhender Luft $ (v=0) $ mit $ \frac{dx}{dt} = c $ aus
	\item in Richtung von $ \vec{v} $ mit $ \frac{dx}{dt} = v+c $ aus
	\item in Richtung entgegen $ \vec{v} $ mit $ \frac{dx}{dt} = c-v $ aus
\end{itemize}
\underline{Bsp:} Elektromagnetische Wellen\\
Photon wird bei $ t = t' = 0 $, $ x = x' = 0 $ in x-Richtung ausgesendet und bewegt sich in IS mit der Geschwindigkeit c. (hier IS mit v in x und IS' stillstehend)\\
Galilei Trafo.:
\begin{tikzpicture}
\draw[->] (0,0) -- (1,0); 
\draw[->] (0,0) -- (0,1);

\node at (1.3,0.5) [align=flush center,text width=2cm]
{$\buildrel{\vec{v}} \over \longrightarrow$};
\node at (0.7,1) [align=flush center,text width=2cm]
{IS};
\end{tikzpicture}%
\begin{tikzpicture}
\draw[->] (0,0) -- (1,0); 
\draw[->] (0,0) -- (0,1);
\node at (0.7,1) [align=flush center,text width=2cm]
{IS'};
\end{tikzpicture}

\begin{equation*}
\frac{dx}{dt} = c \longrightarrow \frac{dx'}{dt'} = c+v
\end{equation*}
\subsubsection{\underline{Michelson Experiment (1885)}}
Interferenz Exp zum Nachweis des ,,Äthers`` als Träger der Lichtwellen, gemessen mit und gegen die Erdbewegung\\
$ \rightarrow $ Lichtgeschwindigkeit ist konstant !

% neue Vorlesung

\section{Relativistische Mechanik}
Michelson: (Vakuum-) Lichtgeschwindigkeit $ c = \const $
\begin{itemize}
	\item die von Galilei-Trafo vorhergesagte Addition von Geschwindigkeiten gilt nicht allg, obwohl gut bestätigt für $ v \ll c $
	\item Naturgesetze hängen nicht von der Wahl des Inertialsystems ab\\
	$ \rightarrow $ es können nur relative Bewegungen gemessen werden also keine absoluten Geschwindigkeiten
	\item Maxwell-Gl. enthalten Lichtgeschw. als Konstante $ c $, e.m. Wellen breiten sich (im Vakuum) immer mit $ c $ aus
	\begin{itemize}
		\item Mit Galilei-Trafo wären damit Maxwell-Gl. in unterschiedlichen IS verschieden
		\item Gemäß Michelson-Exp wären Maxwell-Gl. in allen IS gültig ,,relativistische Gl.``
	\end{itemize}
\end{itemize}
\subsection{\underline{Einsteinsches Relativitätsprinzip} (1905)}
\begin{itemize}
	\item Konzept von Äther falsch
	\item Mechanik und Edynamik sollen unter gleiche Trafos form-invariant sein
\end{itemize}
\begin{enumerate}
	\item[$ \rightarrow $ 1.)] Alle IS sind gleichwertig
	\item[2.)] Licht breitet sich in allen IS mit Geschw. $ c $ aus
\end{enumerate}
\begin{enumerate}
	\item[$ \rightarrow $ \phantom{1.)}] 	Dann muss Galilei Trafo (1) in eine allgemeine Form bringen: ,,Lorenz Trafo`` (1904)\\
	Für $ v \ll c $ muss Gl.(1) als Grenzfall enthalten sein
\end{enumerate}

\paragraph{Bem:}
\begin{itemize}
	\item Einsteinsche Rel. prinzip und die daraus folgende Lorentz-Trafo sind unschwer nachzuvollziehen
	\item Die Konsequenzen daraus, insbesondere die Relativität von Raum und Zeit sind auch heute nicht leicht zu verstehen, da sie alltäglich Erfahrungen wiedersprechen und zu Paradoxien führen.
\end{itemize}

\paragraph{Def: Längenmessung}
Länge eines in IS ruhenden Objekts kann durch ruhende geeichte Maßstäbe bestimmt werden. Die sogenannte \underline{Eichlänge} hängt nicht vom IS ab, ist also Lorenz-invariant.

\subsubsection{Def: Zeitmessung: Synchronisierte Uhren}

\begin{tikzpicture}
\draw[->] (0,0) -- (5,0) node[right] {$x$}; 
\draw[->] (0,0) -- (0,2) node[above] {$y$};
\draw (0,0) circle(0.5cm);
\draw (3,2) circle(0.5cm);
\draw[thick,->] (0,0) -- (-0.3,0.3);
\draw[thick,->] (3,2) -- (2.7,1.7);
\draw[thick,->] (0,0) -- (-0.3,-0.3);
\draw[thick,->] (3,2) -- (2.7,2.3);
\end{tikzpicture}

\begin{itemize}
	\item Standartuhr ist im Ursprung
	\item synchronisierte Uhr an andrem Ort soll gleiche Zeit anzeigen\\
	Synchronisation erfolgt durch Austausch von Signalen\\
	d.h. zur Zeit $ t $ sendet Standartuhr ein Signal zur anderen Uhr , das sofort wieder zurückgeschickt wird und zur Zeit $ t+\Delta t $ ankommt $ \rightarrow $ Bei Empfangen des Signals von der anderen Uhr war Zeitpunkt $ t + \frac{\Delta t}{2} $
\end{itemize}

\subsubsection{Gleichzeitigkeit}
$$
\begin{tikzpicture}
\draw[thick] (0,0) -- (5,0);
\draw (0.2,0) -- (0.2,1);
\draw (0.2,1) -- (4,1);
%\draw (4,1) edge[bend left=30,looseness=1.5] (4.5,0);  alte kantige version
\draw[-] (4,1) to [out=0,in=90] (4.8,0);
\draw[red,text=red] (1.2,1) node[above]{$A'$};
\fill[red] (1.2,0.5)  circle[radius=1.5pt];
\draw[red,text=red] (2.4,1) node[above]{$M'$};
\fill[red] (2.4,0.5)  circle[radius=1.5pt];
\draw[red,text=red] (3.6,1) node[above]{$B'$};
\fill[red] (3.6,0.5)  circle[radius=1.5pt];
\draw[red,text=red] (6,1) node[above]{IS' Zug};
\draw[blue,text=blue] (1.2,0) node[below]{$A$};
\fill[blue] (1.2,0)  circle[radius=1.5pt];
\draw[blue,text=blue] (2.4,0) node[below]{$M$};
\fill[blue] (2.4,0)  circle[radius=1.5pt];
\draw[blue,text=blue] (3.6,0) node[below]{$B$};
\fill[blue] (3.6,0)  circle[radius=1.5pt];
\draw[blue,text=blue] (6.58,0) node[below]{IS Bahnsteig};
\node at (5.5,0.5) [align=flush center,text width=3cm]
{$\longrightarrow \vec{v}$};
\end{tikzpicture}$$
Betrachte Bahnsteig ,,IS`` und Zug mit konstanter Geschw. $ \vec{v} $ ,,IS'``\\
A und B sind zwei punkte im IS, in der Mitte M Steht der Beobachter. Dazu Gehören die Gleichen Punkte im IS' A', B' und M'
\begin{enumerate}[1.)]
	\item Zur Zeit $ t_1 $ werden bei A und B \underline{gleichzeitigkeit} 2 Lichtquellen eingeschalten.
	\item Im Zug haben Lichtquellen zu $ t_1 $ die Position A' und B'. Ein Zugreisender bei M' sieht zuerst das von B' kommende Licht (der fahrende Zug verkürzt die Strecke $ \overline{\tx{M'B}} $) \\
	Er weis, dass A' und B' gleich weit entfernt sind und dass Licht isotrop ausbreitet.\\
	$ \rightarrow $ Für ihn wurde das Licht in B' früher eingeschaltet als in A' und damit \underline{nicht gleichzeitig}.
	\item Etwas später erreichen Beobachter M \underline{gleichzeitig} die beiden Lichtsignale.
	\item Zuletzt sieht M' das von A ausgesandte Lichtsignal.
\end{enumerate}
\rbox{Gleichzeitigkeit hängt vom Bezugssystem ab, es ist also ein relativer Begriff.}
Dieser Effekt verschwindet, wenn Lichtgeschw. $ c \rightarrow \infty $, also bei \underline{instantaner} Signalübertragung.

\section{Lorenz-Trafo}
Zur Konstruktion verwenden wir die Symmetrien
\begin{itemize}
	\item Homogenität von Raum und Zeit\\
	d.h. alle Raumzeitpunkte sind äquivalent, man kann also seinen Ursprung beliebig Wählen
	\item Isotropie des Raumes\\
	d.h. alle Raumrichtungen sind äquivalent
\end{itemize}
\begin{tikzpicture}
\draw[->] (0,0) -- (2,0) node[right] {$x$}; 
\draw[->] (0,0) -- (0,1) node[above] {$y$};
\node at (0.7,1) [align=flush center,text width=2cm]
{IS};
\end{tikzpicture}%
\begin{tikzpicture}
\draw[->] (0,0) -- (2,0) node[right] {$x$}; 
\draw[->] (0,0) -- (0,1) node[above] {$y$};
\node at (1.3,0.5) [align=flush center,text width=3cm]
{\color{blue}$\longrightarrow \vec{v} = v\vec{e_x}$};
\node at (0.7,1) [align=flush center,text width=2cm]
{IS'};
\end{tikzpicture}

O.B.d.A. betrachten wir Bewegung entlang x-Achse (IS' in x-Richtung mit $ \vec{v} = v \vec{e}_x $) d.h. am Anfang $ y = y' \quad z = z' $\\
Zur Zeit $ z=0 $ fallen IS und IS' zusammen.\\[5pt]
Wegen Homogenität von Raum und Zeit muss Trafo \underline{linear} sein
\begin{itemize}
	\item sonst könnte Koord. Ursprung nicht beliebig gewählt werden
	\item sonst wäre ein gleichförmig bewegter Körper in IS beschleunigt in IS' 
\end{itemize}
\underline{Ansatz:} 
$$x' = a_{11} x + a_{12} t + b_1$$
$$y' = y \ \ , \ \ z' = z$$
$$z' = a_{21} x + a_{22} t + b_2$$
Aufgrund der Anfangsbed. für $ t=0 $ ist:
$$x_0 = x'_0 = 0 \ \ , \ \ t_0 = t'_0 = 0$$
ist: $b_1 = b_2 = 0$ , d.h. :
\begin{align}
x' &= a_{11} x + a_{12} t \tag{1}\\
t' &= a_{21} x + a_{22} t \tag{2}
\end{align}
wobei $ a_{ij} = a_{ij}(v) $\\[5pt]
Betrachte Bewegung des Ursprungs in IS' im IS d.h. $ x' = 0 $. Gl. (1) ergibt:
\begin{equation*}
0 = a_{11} x + a_{12} t \rightarrow - \frac{a_{12}}{a_{11}} = \frac{x}{t} = v \tag{3}
\end{equation*}
Eingesetzt in (1):
\begin{equation*}
x' = a_{11}(v) (x-vt) \tag{4}
\end{equation*}
Analog: Bewegung von Ursprung von IS in IS': $ x = 0 $
\begin{equation*}
x = a_{11}(-v) [x' - (-v) t'] \tag{5}
\end{equation*}
Wegen Isotropie des Raumes ist
\begin{equation*}
a_{11}(v) = a_{11}(-v) \ \ \tx{oder} \ \ a_{ij} = a_{ij}(v^2)
\end{equation*}
Betrachten wir nun eine Lichtquelle, die in IS bei $ x=0 $ ruht und sich daher in IS' mit $ -\vec{v} $ bewegt, und zur Zeit $ t_0 = t_0' $ einen kurzen Lichtblitz aussendet.\\
Wegen $ c = c' $ gilt für den Ort des Photons:
\begin{equation*}
x = ct \quad x' = ct' \tag{6}
\end{equation*}
Setze $ t = \frac{x}{c} $ und $ t' = \frac{x'}{c} $ in (4) und (5) ein:
\begin{align*}
x' &= a_{11}(v^2) x(1-\frac{v}{c}) \tag{7} \\
x &= a_{11}(v^2) x'(1+\frac{v}{c}) \tag{7}
\end{align*}
Ineinander eingesetzt und geteilt durch $ x' $:
\begin{align}
1 &= a_{11}^2 (a-\frac{v}{c}) (1+\frac{v}{c})\\
a_{11}^2 &= \frac{1}{1-(\frac{v}{c})^2}
\end{align}
Da Galilei-Trafo als Grenzfall sein soll, nur positive Wurzel
\begin{equation*}
a_{11} = \frac{1}{\sqrt{1 - \frac{v^2}{c^2}}} \tag{8}
\end{equation*}
Mit (3) ist $$a_{12} = \frac{-v}{\sqrt{1 - \frac{v^2}{c^2}}}$$
\begin{align*}
(2):\ t' &= a_{21} x + a_{12} t\\
(5):\ x &= a_{11} x' + a_{11} v t'
\end{align*}



% check index


\begin{align*}
t' &= \frac{x}{a_{11} v} - \frac{a_{12}}{a_{11} v} \frac{(x-vt)}{\sqrt{1-\frac{v^2}{c^2}}}\\
&= \frac{x}{v} \sqrt{1-\frac{v^2}{c^2}} - \frac{(x-vt)}{v \sqrt{1-\frac{v^2}{c^2}}}\\
&= \frac{x (1-\frac{v^2}{c^2}) - x + vt}{v \sqrt{1-\frac{v^2}{c^2}}}\\
&= \frac{t - \frac{v^2}{c^2} x}{1 - v'}
\end{align*}

%%% hier fehlt vielleicht? shit


\subsubsection{Lorentz trafo}
Mit dem Lorentz faktor $\gamma  = \frac{1}{\sqrt{1- \frac{v^2}{c^2}}}$ erhalten wir
\rbox{\begin{align*} \tag 9
	x' &= \frac{x-ct}{\sqrt{1-\frac{v^2}{c^2}}} = \gamma (x-ct)\\
	t' &= \frac{t-\frac{v^2}{c^2} x}{\sqrt{1 - \frac{v^2}{c^2}}} = \gamma (t-\frac{v}{c^2}x) 
	\end{align*}}
$$a_{11} = a_{22} = \gamma = \coth \psi$$
$$a_{12} = a_{21} = \gamma \frac{v}{c} = - \sinh \psi$$
mit dem sog. Rapidität $\psi = \tx{arctanh} \frac{v}{c}$

%\begin{tikzpicture}
%\draw [-,black,thick] (-0.5,-1) to [out=30,in=150] (0.5,-1) to [out=-30,in=-150] (1.5,-1) to [out=30,in=150] (2.5,-1);
%\end{tikzpicture}





\end{document}