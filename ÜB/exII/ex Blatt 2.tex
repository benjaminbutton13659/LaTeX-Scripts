
\documentclass[12pt]{article}
 
\usepackage[utf8]{inputenc}
\usepackage[T1]{fontenc}
\usepackage[german]{babel}
\usepackage{graphicx}
\usepackage{cleveref}
\usepackage{wrapfig}
\usepackage{amsmath}
\usepackage{amsfonts}
\usepackage{amssymb}
\usepackage{tikz}
\usepackage{nicefrac}
\usepackage{mathtools}
\usepackage[margin=1in]{geometry}

\newcommand{\N}{\mathbb{N}}
\newcommand{\Z}{\mathbb{Z}}
\newcommand{\kq}{\frac{1}{4 \pi \epsilon_0}}
\renewcommand{\Phi}{\phi}

\newenvironment{theorem}[2][Theorem]{\begin{trivlist}
\item[\hskip \labelsep {\bfseries #1}\hskip \labelsep {\bfseries #2.}]}{\end{trivlist}}
\newenvironment{lemma}[2][Lemma]{\begin{trivlist}
\item[\hskip \labelsep {\bfseries #1}\hskip \labelsep {\bfseries #2.}]}{\end{trivlist}}
\newenvironment{exercise}[2][Aufgabe]{\begin{trivlist}
\item[\hskip \labelsep {\bfseries #1}\hskip \labelsep {\bfseries #2.}]}{\end{trivlist}}
\newenvironment{problem}[2][Problem]{\begin{trivlist}
\item[\hskip \labelsep {\bfseries #1}\hskip \labelsep {\bfseries #2.}]}{\end{trivlist}}
\newenvironment{question}[2][Question]{\begin{trivlist}
\item[\hskip \labelsep {\bfseries #1}\hskip \labelsep {\bfseries #2.}]}{\end{trivlist}}
\newenvironment{corollary}[2][Corollary]{\begin{trivlist}
\item[\hskip \labelsep {\bfseries #1}\hskip \labelsep {\bfseries #2.}]}{\end{trivlist}}

\newenvironment{solution}{\begin{proof}[Solution]}{\end{proof}}
 
\begin{document}
 
\title{Blatt 2}
\author{Andréz Gockel\\ 
Experimental Physik II\\
Gruppe: 6}
 
\date{07.05.18}
\maketitle


\begin{exercise}{1a}
Berechnen Sie das elektrische Feld des Drahtes, indem Sie das Superpositionsprinzip anwenden.\\
%Superposition:
%\begin{itemize}
%	\item Superpositionsprinzip: $$\vec{E}_{Q_1 + Q_2} = \vec{E}_{Q_1} + \vec{E}_{Q_2}$$
%	\item N Punktladungen $Q_i, i = 1, \dots , N
%\qquad \vec{F} = \sum _{i=1}^{N} \vec{F}_i = \frac{1}{4 \pi \epsilon_{0}} \sum _{i=1}^{N} \frac{Q_i}{r^2_i} \hat{r}_i$
%$$ \Rightarrow \vec{E} = \sum _{i=1}^{N} \vec{E}_i = \frac{1}{4 \pi \epsilon_{0}}  \sum _{i=1}^{N} \frac{Q_i}{r^2_i} \hat{r}_i$$
%\item kontinuierliche Ladungsverteilung $\rho(\vec{r})$\\
%Gesamtladung $Q = \int dV \rho(\vec{r}') \quad (dV = d \vec{r}^{'3})$
%
%$$\vec{E} (\vec{r}) = \frac{1}{4 \pi \epsilon_0} \int d^3 \vec{r}' \frac{\rho (\vec{r}')}{\vert \vec{r}-\vec{r}' \vert ^2} \frac{\vec{r}-\vec{r}'}{\vert \vec{r}-\vec{r}' \vert}$$
%
%\end{itemize}
%
%Linenladungsdichte auf einem Ring $\lambda = \frac{Q}{2 \pi R} \Rightarrow Q = 2 \pi R \lambda$\\
%Internet gibt uns: $\vec{E}=\frac{L \lambda}{L 2 \pi \epsilon_0 R}$ wobei R die Distanz von dem Draht ist, und L die länge des drahtes bzw. die höhe des zylinders ist.
%
%als unendliche summe: $$\vec{E}=\kq \sum_{i=-\infty}^{\infty} \frac{Q_i}{r^2_i}$$
%Die gesamt Ladung des Drahtes beträgt:  $L \lambda$
%$$E(R)=\kq \int_{L}\lambda(r) dL$$
%$$d\vec{F} = \frac{q}{4 \pi \epsilon_0} \frac{\sigma \cdot dA}{b^2} \hat b$$
%$$dQ=\lambda d L$$
$$b = \frac{a}{\cos \alpha}$$
$$L = R \tan \alpha \ \Rightarrow dL = \frac{R}{\cos^2 \alpha}$$
$$\Rightarrow dE=\kq \frac{\lambda dL}{b^2} \Rightarrow E = \kq \frac{\lambda dL}{(\frac{R}{\cos \alpha})^2} = \kq \frac{\lambda dL}{R^2}\cos^2 \alpha$$
$$\Rightarrow E = \kq \int_{-\frac{\pi}{2	}}^{\frac{\pi}{2	}}\frac{\lambda}{R^2} \cos^2 \alpha \frac{R}{\cos^2 \alpha} \ d\alpha = \kq 2 \int_{0}^{\frac{\pi}{2}}\frac{\lambda}{R^2} \cos^2 \alpha \frac{R}{\cos^2 \alpha} \ d\alpha = \frac{\lambda}{2 \pi \epsilon_0 R}$$
\end{exercise}
\pagebreak

\begin{exercise}{1b}
Berechnen Sie den Fluß des elektrischen Feldes durch die Oberfläche eines Zylinders mit Durchmesser $D$ und Höhe $H$ unter Benutzung des Ergebnisses aus Teil a). Zylinder und Draht seien koaxial. Warum ist dieses Resultat zu erwarten?\\[10pt]
	Elektrische Fluss für beliebige Fläche A:
	$$\phi_A = \int_{A}\vec{E}d\vec{A}$$
	mit: $$E = \frac{\lambda}{2 \pi \epsilon_0 R} \ , R = D/2$$
	$$\phi = \int_{A} \frac{\lambda}{\pi \epsilon_0 D}dA$$
	mit: $$dA = 2\pi rh$$
	$$\Rightarrow \phi = \int_{0}^{H}dh \frac{2\pi D \lambda}{2 \pi \epsilon_0 D}h =  \frac{\lambda}{\epsilon_0} \int_{0}^{H}dh\ h =  \frac{\lambda}{2 \epsilon_0}H^2$$
\end{exercise}
\pagebreak

\begin{exercise}{1c}
Der Draht wird durch ein Koaxialkabel ersetzt. Der Draht im Inneren hat Radius $R_1$ (Seele des Kabels) und ist homogen positiv geladen. Der Draht ist in einem Abstand $R_2$ durch eine negativ geladene Abschirmung ummantelt. Bestimmen Sie die elektrische Feldstärke für alle möglichen Werte des Radius r.\\[10pt]
	für: $R_1 < r < R_2$ : $$\textrm{aus dem Gauß'schen Gesetz folgt: } \phi = \oint_A E_ndA = \frac{q_{innen}}{\epsilon_0}$$
	mit: 
	\begin{align*}	
	\oint_A E_n dA &= \int_{links} E_n dA + \int_{Mantel} E_n dA + \int_{rechts} E_n dA\\[10pt] 
	&= 0 + \int_{Mantel} E_r dA + 0\\[10pt] 
	&= E_r \int_{Mantel} dA \\[10pt] 
	&= E_r 2 \pi r h
	\end{align*}
	
	$\oint_A E_n dA$ einsetzen: $$E_r 2 \pi r h = \frac{q_{innen}}{\epsilon_0}$$
	mit: $$q_{innen} = \frac{h}{L}q$$
	$$E_r 2 \pi r h = \frac{h}{\epsilon_0 L}q \Rightarrow E_r = \frac{q}{2 \pi L \epsilon_0 r}$$
	
\end{exercise}
\pagebreak

\begin{exercise}{2}
Betrachten Sie einen Würfel der Kantenlänge $2d$, in dessen Mittelpunkt sich eine Punktladung der Grösse $q$ befindet. Berechnen Sie den Fluss durch eine der Oberflächen und zeigen Sie, dass sich in der Summe der Oberflächen $q/\epsilon_0$ ergibt. Hinweis: Sie können das Integral in einer Integrationstabelle nachschlagen.\\[10pt]
	$$\phi = \oint_A E_n dA$$

\end{exercise}
\pagebreak

\begin{exercise}{5}
Sie reihen abwechselnd Anionen mit negativer Elementarladung und Kationen mit positiver Elementarladung auf einem unendlich ausgedehnten, eindimensionalen Gitter auf. Der Abstand zwischen zwei Ladungen beträgt $a = 5 \cdot 10^{-10}$m. Bestimmen Sie die potenzielle Energie eines
Kations. Hinweis: $\sum_{n=1}^\infty \frac{1}{2} (−1)^{n+1}x^n = \ln(1 + x)$ für $−1 < x < 1$.\\[10pt]
%$$dE_{pot}=-\Vec{F} \cdot d\Vec{s}$$
%Wobei $d\vec{s}$ der verschiebungs vektor ist, und $d\Phi$ die Potentialdifferenz ist.
%$$d\Phi = \frac{dE_{el}}{q_0} = -\vec{E} \cdot ds$$
%$$\Delta \Phi = \Phi_b - \Phi_a = \frac{\Delta E_{el}}{q_0} = - \int_a^b \vec{E} \cdot ds$$
%Falls bei dem gewählten Nullpunkt die Elektrische Energie und das elektrische Potenzial 0 gilt: $$E_{el} = q_0 \Phi$$
%Das Elektrische Feld einer Punktladung ist: $$\vec{E} = \frac{1}{4 \pi \epsilon_0}\frac{q}{r^2}\hat{\vec{r}}$$
%Potential einer punktladung ist (Das Coulomb Potential):
%$$\Phi = \kq \frac{q}{r}$$
%Dabei wird der Abstand als unendlich gewählt damit $E_{el} = 0$.
%Die elektrische Energie von zwei ladungen:
%$$E_{el} = q_0 \Phi = q_0 \kq \frac{q}{r} = \kq \frac{q_0 q}{r}$$

$$E_{pot} = a \cdot F_c = a \cdot \kq \frac{Q_1 \cdot Q_2}{a^2} = \kq \frac{Q_1 \cdot Q_2}{a}$$
$n = 1$: $$ E_{pot1} = - \kq \frac{e^2}{a}$$
$n = 2$: $$ E_{pot2} = \kq \frac{e^2}{2a}$$
$n = 3$: $$ E_{pot} = - \kq \frac{e^2}{3a}$$
\begin{align*}
	\Rightarrow E^{tot}_{pot} = \kq \frac{e^2}{a}\sum_{-\infty}^{\infty}\frac{1}{n}(-1)^n &= -\kq \frac{e^2}{a}\sum_{-\infty}^{\infty}\frac{1}{n}(-1)^{n+1}\\
	&= -2 \kq \frac{e^2}{a}\sum_{n=1}^{\infty}\frac{1}{n}(-1)^{n+1}\\
	&= -2 \kq \frac{e^2}{a}\ln(2)\\
	&= -2 \cdot 899\cdot 10^7 \cdot \frac{(1,6 \cdot 10^{-19})^2}{5 \cdot 10^{-10}}\ln(2)\\
	&= -6 \cdot 10^{-19}
\end{align*}

\end{exercise}
 

\end{document}