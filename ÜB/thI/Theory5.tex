\documentclass[12pt,a4paper,ngerman]{article}
\usepackage[utf8]{inputenc}
\usepackage[T1]{fontenc}
\usepackage[german]{babel}
\usepackage{graphicx}
\usepackage{cleveref}
\usepackage{wrapfig}
\usepackage{amsmath}
\usepackage{amsfonts}
\usepackage{amssymb}
\usepackage{tikz}
\usepackage{nicefrac}
\usepackage{mathtools}
\usepackage[margin=1in]{geometry}
\usepackage{tcolorbox}
\usepackage{amsthm}
\usepackage{savetrees}
\setlength{\parskip}{1em}

\usepackage{eulervm}

\pagecolor{black}
\color{white}

\newenvironment{exercise}[2][Aufgabe]{\begin{trivlist}
\item[\hskip \labelsep {\bfseries #1}\hskip \labelsep {\bfseries #2.}]}{\end{trivlist}}

\newcommand{\verteq}{\rotatebox{90}{$\,=$}}
\newcommand{\equalto}[2]{\underset{\scriptstyle\overset{\mkern4mu\verteq}{#2}}{#1}}
\newcommand{\equaltoup}[2]{\overset{\scriptstyle\underset{\mkern4mu\verteq}{#2}}{#1}}
\newcommand{\tx}[1]{\textrm{#1}}
\newcommand{\ov}[1]{\overline{#1}}
\newcommand{\ub}[1]{\underbrace{#1}}
\newcommand{\ob}[1]{\overbrace{#1}}
\newcommand{\im}{\tx{im}}
\newcommand{\spa}{\tx{span}}
\newcommand{\adj}{\tx{adj}}
\newcommand{\grad}{\tx{grad }}
\newcommand{\zz}{\fontfamily{cmss} \selectfont{Z\kern-.61em\raise-0.7ex\hbox{Z}:}}
\newcommand{\const}{\tx{const.}}
\newcommand{\summ}[2]{\sum_{#1}^{#2}}
\newcommand{\intt}[2]{\int_{#1}^{#2}}
\newcommand{\vc}[1]{\boldsymbol{#1}}
\newcommand{\vl}{\vc{\varLambda}}
\begin{document}

\title{Blatt 5}
\author{Andréz Gockel\\ 
Theoretische Physik I\\
Gruppe: 4}
\date{29.05.18}
\maketitle
\paragraph{Aufgabe 2}
$$V(r) = -\frac{\alpha}{r}, \quad \vl = \frac{\vc{p} \times \vc{L}}{\alpha m} -  \frac{\vc{r}}{r}, \quad \vc{p}=m\vc{\dot{r}}, \quad \vc{L}=\vc{r}\times \vc{p}, \quad \alpha = \gamma m M$$ 
$$\frac{d}{dx} \sqrt{x} = \frac{1}{2\sqrt{x}}$$
$$\vc{r} = \begin{pmatrix} x \\ y \\ z \end{pmatrix} ,\quad r = \sqrt{x^2+y^2+z^2}, \quad \vec{\nabla} \vc{r} = \frac{\vc{r}}{r} = \vc{e}_r$$
\begin{equation}
	\vec{\nabla} V(r) = \frac{\partial V(r)}{\partial r} \vec{\nabla} \vc{r} = -\frac{\partial}{\partial r} \frac{\alpha}{r}\vc{e}_r = \frac{\alpha}{r^2}\vc{e}_r = -\frac{V(r)}{r}\vc{e}_r
\end{equation}
\begin{equation}
	\vc{a} \times (\vc{b} \times \vc{c}) = \vc{b} \cdot (\vc{a} \cdot \vc{c}) - \vc{c} \cdot ( \vc{a}\cdot \vc{b})
\end{equation}
\paragraph{a)} \zz \ 
$\dot{\vc{\varLambda}} = 0.$
\begin{proof}
	$$\vl = (\vc{r} \times \vc{L}) + V(r)\vc{r}$$
	$$\dot{\vl}=(\vc{\ddot{r}}\times\vc{L}) + \underbrace{(\vc{r} \times \vc{\dot{L}})}_{=0, \tx{ da } \vc{L}= \tx{const.}} + \frac{d}{dt}(V(r) \cdot \vc{r})$$
	Mit: $\frac{d}{dt}V(r)=\vec{\nabla}V(r)\cdot\vc{\dot r}$ \textit{(aus vorlesung)} ist
	$$\dot{\vl}= (\vc{\ddot{r}}\times\vc{L}) + V(r)\cdot\vc{\dot r} + \vc{r} \cdot (\vec{\nabla}V(r)\cdot\vc{\dot r})$$
	Da die Kräfte konservativ sind: $m\vc{\ddot r} = -\grad V(r) \quad \Rightarrow \vc{\ddot r} = -\frac{\vec{\nabla V(r)}}{m}$
	\begin{align*} 
		\Rightarrow \dot \vl &= (\bigg(-\frac{\vec{\nabla V(r)}}{m}\bigg)\times\vc{L}) + V(r)\cdot\vc{\dot r} + \vc{r} \cdot (\vec{\nabla}V(r)\cdot\vc{\dot r})\\
		&=(-\vec{\nabla} V(r)\times(\vc{r}\times\vc{\dot r})) + V(r)\cdot\vc{\dot r} + \vc{r} \cdot (\vec{\nabla}V(r)\cdot\vc{\dot r})\\
		\buildrel{\tx{mit (1)}}\over{\Rightarrow} &= \frac{V(r)}{r^2}(\vc{r}\times(\vc{r}\times\vc{\dot r})) + V(r)\cdot\vc{\dot r} + \vc{r} \cdot (-\frac{V(r)}{r^2}\vc{r}\cdot\vc{\dot r})\\
		\buildrel{\tx{mit (2)}}\over{\Rightarrow} &= \frac{V(r)}{r^2}(\vc{r} \cdot (\vc{r} \cdot \vc{\dot r}) - \vc{\dot r} \cdot ( \vc{r}\cdot \vc{r})) + V(r)\cdot\vc{\dot r} + \vc{r} \cdot (-\frac{V(r)}{r^2}\vc{r}\cdot\vc{\dot r})\\
		&= \frac{V(r)}{r^2}(\vc{r} \cdot (\vc{r} \cdot \vc{\dot r})) - \frac{V(r)}{r^2} \cdot \vc{\dot r} \cdot ( \vc{r}\cdot \vc{r}) + V(r)\cdot\vc{\dot r} + \vc{r} \cdot (-\frac{V(r)}{r^2}\vc{r}\cdot\vc{\dot r})\\
		&= -V(r) \cdot \vc{\dot r} + V(r) \cdot \vc{\dot r} = 0
	\end{align*}
\end{proof}
\pagebreak
\paragraph{b)}
\zz \ $\vl$ zeigt zu dem Perihel.
\begin{proof}
	Wir wählen $r=$ Perihel, d.h. $\vc{r}= -r\vc{e}_y$, $\vc{L}=L\vc{e}_x,\ \vc{p}=-p\vc{e}_z$, daraus folg das $\vl$ in Richtung $-\vc{e}_x$ zeigt (Richtung des Perihel) und da $\vl$ konstant ist zeigt es immer in Richtung Perihel.
\end{proof}
\paragraph{c)}
\zz \ $|\vl|=\epsilon$
\begin{proof}
	Trivial. ;)
\end{proof}





























\end{document}